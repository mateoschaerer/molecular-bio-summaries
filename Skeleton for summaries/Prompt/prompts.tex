\usepackage{nopageno} %Seitenzahlen auf richtiger Seite 

\usepackage[left=2cm, right=2cm, top=2cm, includehead, includefoot, headheight=17pt]{geometry}

\usepackage[utf8x]{inputenc}
\usepackage[english]{babel}
\usepackage{amsmath,amssymb,amsthm}
\usepackage{framed}
\usepackage{wasysym}
\usepackage[T1]{fontenc} %Silbentrennung 
\usepackage{color} %Farbe
\usepackage{graphicx}
\usepackage{float}%Grafik am gleichen Ort plazieren
%pdf. png. einfach eingliedern
\usepackage{subfigure} %Grafiken nebeneinander
\usepackage{pdfpages}
\usepackage{ulem} 	%\uuline{urgent}    % doppelt unterstreichen
%\uwave{boat}      % unterschlängeln
%\sout{wrong}       % durchstreichen
%\xout{removed}     % ausstreichen mit //////.

\usepackage{tikz}
\usetikzlibrary{trees}
\usetikzlibrary{plotmarks}
\usetikzlibrary{angles,quotes,babel}
\usetikzlibrary{shadings}
\usetikzlibrary{patterns}
\usetikzlibrary{matrix}
\usetikzlibrary{arrows}
\usetikzlibrary{calc}

\usepackage{pgfplots}
\usepackage{pgf-pie}
\pgfplotsset{compat=1.10}
\usepgfplotslibrary{statistics}
\usepgfplotslibrary{fillbetween}

\usepackage{tkz-euclide}
\usepackage{enumerate}
\usepackage{stmaryrd}
\usepackage{tabularx}
\usepackage{wrapfig}
\usepackage{epsdice}
\usepackage{multirow}
\usepackage{rotating}
\usepackage{pdflscape}
\usepackage{fancyhdr}

\pagestyle{fancy} %eigener Seitenstil
\fancyhf{} %alle Kopf- und Fußzeilenfelder bereinigen
\fancyhead[L]{} %Kopfzeile links
\fancyhead[C]{} %zentrierte Kopfzeile
\fancyhead[R]{} %Kopfzeile rechts
\renewcommand{\headrulewidth}{0.4pt} %obere Trennlinie
\fancyfoot[C]{\thepage} %Seitennummer
\renewcommand{\footrulewidth}{0.4pt} %untere Trennlinie

% Number spaces 
\newcommand{\CC}{\ensuremath{\mathbb{C}}}
\newcommand{\RR}{\ensuremath{\mathbb{R}}}
\newcommand{\QQ}{\ensuremath{\mathbb{Q}}}
\newcommand{\ZZ}{\ensuremath{\mathbb{Z}}}
\newcommand{\NN}{\ensuremath{\mathbb{N}}}
\newcommand{\LL}{\ensuremath{\mathbb{L}}}
\newcommand{\DD}{\ensuremath{\mathbb{D}}}
\newcommand{\WW}{\ensuremath{\mathbb{W}}}

%draw chemestry molecules 
\usepackage{chemfig} % https://mirror.ox.ac.uk/sites/ctan.org/macros/generic/chemfig/

\newcommand\vv[1]{%
	\begin{tikzpicture}[baseline=(arg.base)]
		\node[inner xsep=0pt] (arg) {$#1$};
		\draw[line cap=round,line width=0.45,->,shorten >= 0.2pt, shorten <= 0.7pt] (arg.north west) -- (arg.north east);
	\end{tikzpicture}%
} %command will render \vv{x} with an arrow aboth 

\renewcommand{\labelenumi}{\roman{enumi})}

\DeclareMathOperator{\ggT}{ggT}
\DeclareMathOperator{\sign}{sign}

%sections
\theoremstyle{plain}
\newtheorem{Thm}{Theorem}[section]
\newtheorem{Def}[Thm]{Definition}
\newtheorem{Prop}[Thm]{Proposition}

\theoremstyle{definition}
\newtheorem{lemma}[Thm]{Lemma}
\newtheorem{corollary}[Thm]{Corollary}
\newtheorem{claim}[Thm]{Claim}
\newtheorem{Proof}[Thm]{Proof}
\newtheorem{Ex}[Thm]{Example}

\newtheorem{Exercise}{ex}[section] %follow proper enum
\newtheorem{ex}[Exercise]{Exercise}
\newtheorem{Solution}{sol}[section]
\newtheorem{sol}[Solution]{Solution}

\theoremstyle{remark}
\newtheorem{remark}[Thm]{Remark} % follows thm enum

\newtheorem{comment}{Comment}[section] %follow comment enum
\newtheorem{notation}[comment]{Notation}
\newtheorem{reasoning}[comment]{Reasoning}
\newtheorem{Intpr}[comment]{Interpretation}

%some premmade with title (uterwise use \textbf{Title} ...)
\newenvironment{ThmWithTitle}[1]{%
	\begin{Thm}[\textbf{#1}]}{\end{Thm}}
\newenvironment{PropWithTitle}[1]{%
	\begin{Prop}[\textbf{#1}]}{\end{Prop}}
\newenvironment{ExWithTitle}[1]{%
	\begin{Ex}[\textbf{#1}]}{\end{Ex}}
\newenvironment{DefWithTitle}[1]{%
	\begin{Def}[\textbf{#1}]}{\end{Def}}
\newenvironment{RemarkWithTitel}[1]{%
	\begin{remark}[\textbf{#1}]}{\end{remark}}

%format of paragraph 
\renewcommand\paragraph{\@startsection{paragraph}{4}{\z@}%
	{-2.5ex\@plus -1ex \@minus -.25ex}%
	{1.25ex \@plus .25ex}%
	{\normalfont\normalsize\bfseries}}
\makeatother
\setcounter{secnumdepth}{4} % how many sectioning levels to assign numbers to
\setcounter{tocdepth}{4}    % how many sectioning levels to show in ToC

\newcounter{row} 
\renewcommand\therow{\alph{row}} %hier a,b,c etc. def und mit therow abrufbar

\newenvironment{aufz}
{\setcounter{row}{0}%
	\par\noindent\tabularx{\linewidth}[t]
	{\cdot{20}{>{\stepcounter{row}\makebox[1.5em][l]{\therow)\hfill}}X}} %bis max 20 Elemente nebeinander
}
{\endtabularx}


%biblio
\usepackage[]{biblatex}
\addbibresource{referenzenma.bib} 

%glossary
\usepackage{glossaries}
\usepackage{import}
