\newglossaryentry{exocytosis}{
	name=exocytosis,
	description={A cellular process in which substances contained in vesicles are released from the cell to the extracellular environment by fusion of the vesicle with the plasma membrane}
}

\newglossaryentry{endocytosis}{
	name=endocytosis,
	description={A cellular process in which the cell membrane folds inward to form a vesicle that encloses extracellular material for internalization into the cell. Or it can also describe the pathway by which material is transported inwards from the plamsa membrane}
}

\newglossaryentry{AP2}{
	name=AP2,
	description={An adaptor protein complex involved in clathrin-mediated endocytosis. AP2 binds to specific phosphorylated phosphoinositides in the plasma membrane}
}

\newglossaryentry{BARdomain}{
	name=BAR domain,
	description={A structural domain found in proteins that bind to and stabilize curved membranes. BAR domains can sense or induce membrane curvature and are involved in various trafficking pathways, including clathrin-mediated endocytosis}
}

\newglossaryentry{dynamin}{
	name=Dynamin,
	description={A large GTPase involved in clathrin-mediated endocytosis. Dynamin assembles around the neck of budding vesicles and, through GTP hydrolysis, facilitates membrane scission to release the vesicle into the cytosol}
}

\newglossaryentry{Rab}{
	name=Rab protein,
	description={A family of small GTPases that regulate vesicle transport by ensuring specificity in vesicle targeting. Rab proteins recruit effector molecules that help guide vesicles to the correct membrane compartment}
}

\newglossaryentry{SNARE}{
	name=SNARE protein,
	description={A group of membrane-associated proteins that mediate the fusion of vesicle and target membranes. SNAREs on the vesicle (v-SNAREs) and on the target membrane (t-SNAREs) form complexes that bring membranes close enough to fuse}
}

\newglossaryentry{NSF}{
	name=NSF,
	description={N-ethylmaleimide-sensitive factor; an ATPase that disassembles SNARE complexes after membrane fusion. NSF uses energy from ATP hydrolysis to recycle SNARE proteins for further rounds of vesicle fusion}
}

\newglossaryentry{homotypicfusion}{
	name=Homotypic membrane fusion,
	description={A type of membrane fusion in which two membranes of the same type or origin (e.g., two endosomes) fuse together. This process is important for organelle maturation and requires matching sets of SNARE proteins on both membranes}
}

\newglossaryentry{VTC}{
	name=Vesicular tubular clusters,
	description={Homotypic membrane fusio of ER-derived vesicles. They function as intermediate sorting stations that mediate cargo transport from the endoplasmic reticulum to the Golgi apparatus}
}

\newglossaryentry{KDEL}{
	name=KDEL sequence,
	description={A C-terminal amino acid sequence (Lys-Asp-Glu-Leu) that serves as a retrieval signal for soluble proteins that reside in the endoplasmic reticulum (ER). Proteins with a KDEL sequence are recognized by KDEL receptors in the Golgi and returned to the ER via retrograde transport}
}

\newglossaryentry{EndoH}{
	name=Endo H,
	description={Short for Endoglycosidase H, an enzyme that cleaves high-mannose and some hybrid N-linked oligosaccharides from glycoproteins}
}

\newglossaryentry{GAGs}{
	name=Glycosaminoglycans (GAGs),
	description={Long, unbranched polysaccharides composed of repeating disaccharide units, typically including an amino sugar and a uronic acid. Often sulfated, they are highly negatively charged and play key roles in the extracellular matrix, providing structural support, hydration, and participating in signaling processes}
}

\newglossaryentry{PAPS}{
	name=PAPS,
	description={Short for 3'-Phosphoadenosine-5'-phosphosulfate, a universal sulfate donor in biological sulfation reactions. It is synthesized in the cytosol and used by sulfotransferases in the Golgi apparatus to add sulfate groups to proteins, lipids, and carbohydrates}
}

\newglossaryentry{lectin}{
	name=Lectin,
	description={Proteins that interact with sugar groups }
}

\newglossaryentry{pinocytosis}{
	name={Pinocytosis},
	description={A form of endocytosis in which cells nonspecifically engulf extracellular fluid and solutes through small vesicles, often referred to as “cell drinking”}
}

\newglossaryentry{macropinocytosis}{
	name={Macropinocytosis},
	description={A form of endocytosis involving the nonspecific uptake of extracellular fluid and membrane through large vesicles called macropinosomes. It is often triggered by growth factors and involves actin-driven plasma membrane ruffling}
}

\newglossaryentry{LDL}{
	name={Low-Density Lipoprotein (LDL)},
	description={A cholesterol-carrying particle in the bloodstream. LDL delivers cholesterol to cells via receptor-mediated endocytosis and is often referred to as "bad cholesterol" due to its association with atherosclerosis.}
}

\newglossaryentry{ESCRT}{
	name={ESCRT complex},
	description={A set of cytosolic protein complexes (ESCRT-0, -I, -II, -III) involved in sorting ubiquitylated membrane proteins into intralumenal vesicles of multivesicular bodies. They recognize ubiquitin tags and phosphoinositide signals, enabling membrane invagination and vesicle formation for lysosomal degradation}
}

\newglossaryentry{transcytosis}{
	name=transcytosis,
	description={A process where molecules are transported across a cell by endocytosis on one side and exocytosis on the other. Common in epithelial cells, e.g., transport of antibodies in newborns}
}

\newglossaryentry{constitutive-secretory-pathway}{
	name={Constitutive secretory pathway},
	description={A continuous secretory route used by all cells, in which vesicles deliver proteins and lipids from the Golgi apparatus to the plasma membrane for immediate secretion or membrane insertion}
}

\newglossaryentry{regulated-secretory-pathway}{
	name={Regulated secretory pathway},
	description={A secretion route in specialized cells where proteins are stored in secretory vesicles and released by exocytosis only in response to specific signals, such as hormones or neurotransmitters}
}
