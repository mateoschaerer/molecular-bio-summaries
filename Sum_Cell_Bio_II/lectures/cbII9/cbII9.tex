\documentclass[../main.tex]{subfiles}

\usepackage{nopageno} %Seitenzahlen auf richtiger Seite 

\usepackage[left=2cm, right=2cm, top=2cm, includehead, includefoot, headheight=17pt]{geometry}

\usepackage[utf8x]{inputenc}
\usepackage[english]{babel}
\usepackage{amsmath,amssymb,amsthm}
\usepackage{framed}
\usepackage{wasysym}
\usepackage[T1]{fontenc} %Silbentrennung 
\usepackage{color} %Farbe
\usepackage{graphicx}
\usepackage{float}%Grafik am gleichen Ort plazieren
%pdf. png. einfach eingliedern
\usepackage{subfigure} %Grafiken nebeneinander
\usepackage{pdfpages}
\usepackage{ulem} 	%\uuline{urgent}    % doppelt unterstreichen
%\uwave{boat}      % unterschlängeln
%\sout{wrong}       % durchstreichen
%\xout{removed}     % ausstreichen mit //////.

\usepackage{tikz}
\usetikzlibrary{trees}
\usetikzlibrary{plotmarks}
\usetikzlibrary{angles,quotes,babel}
\usetikzlibrary{shadings}
\usetikzlibrary{patterns}
\usetikzlibrary{matrix}
\usetikzlibrary{arrows}
\usetikzlibrary{calc}

\usepackage{pgfplots}
\usepackage{pgf-pie}
\pgfplotsset{compat=1.10}
\usepgfplotslibrary{statistics}
\usepgfplotslibrary{fillbetween}

\usepackage{tkz-euclide}
\usepackage{enumerate}
\usepackage{stmaryrd}
\usepackage{tabularx}
\usepackage{wrapfig}
\usepackage{epsdice}
\usepackage{multirow}
\usepackage{rotating}
\usepackage{pdflscape}
\usepackage{fancyhdr}

\pagestyle{fancy} %eigener Seitenstil
\fancyhf{} %alle Kopf- und Fußzeilenfelder bereinigen
\fancyhead[L]{} %Kopfzeile links
\fancyhead[C]{} %zentrierte Kopfzeile
\fancyhead[R]{} %Kopfzeile rechts
\renewcommand{\headrulewidth}{0.4pt} %obere Trennlinie
\fancyfoot[C]{\thepage} %Seitennummer
\renewcommand{\footrulewidth}{0.4pt} %untere Trennlinie

% Number spaces 
\newcommand{\CC}{\ensuremath{\mathbb{C}}}
\newcommand{\RR}{\ensuremath{\mathbb{R}}}
\newcommand{\QQ}{\ensuremath{\mathbb{Q}}}
\newcommand{\ZZ}{\ensuremath{\mathbb{Z}}}
\newcommand{\NN}{\ensuremath{\mathbb{N}}}
\newcommand{\LL}{\ensuremath{\mathbb{L}}}
\newcommand{\DD}{\ensuremath{\mathbb{D}}}
\newcommand{\WW}{\ensuremath{\mathbb{W}}}

%draw chemestry molecules 
\usepackage{chemfig} % https://mirror.ox.ac.uk/sites/ctan.org/macros/generic/chemfig/

\newcommand\vv[1]{%
	\begin{tikzpicture}[baseline=(arg.base)]
		\node[inner xsep=0pt] (arg) {$#1$};
		\draw[line cap=round,line width=0.45,->,shorten >= 0.2pt, shorten <= 0.7pt] (arg.north west) -- (arg.north east);
	\end{tikzpicture}%
} %command will render \vv{x} with an arrow aboth 

\renewcommand{\labelenumi}{\roman{enumi})}

\DeclareMathOperator{\ggT}{ggT}
\DeclareMathOperator{\sign}{sign}

%sections
\theoremstyle{plain}
\newtheorem{Thm}{Theorem}[section]
\newtheorem{Def}[Thm]{Definition}
\newtheorem{Prop}[Thm]{Proposition}

\theoremstyle{definition}
\newtheorem{lemma}[Thm]{Lemma}
\newtheorem{corollary}[Thm]{Corollary}
\newtheorem{claim}[Thm]{Claim}
\newtheorem{Proof}[Thm]{Proof}
\newtheorem{Ex}[Thm]{Example}

\newtheorem{Exercise}{ex}[section] %follow proper enum
\newtheorem{ex}[Exercise]{Exercise}
\newtheorem{Solution}{sol}[section]
\newtheorem{sol}[Solution]{Solution}

\theoremstyle{remark}
\newtheorem{remark}[Thm]{Remark} % follows thm enum

\newtheorem{comment}{Comment}[section] %follow comment enum
\newtheorem{notation}[comment]{Notation}
\newtheorem{reasoning}[comment]{Reasoning}
\newtheorem{Intpr}[comment]{Interpretation}

%some premmade with title (uterwise use \textbf{Title} ...)
\newenvironment{ThmWithTitle}[1]{%
	\begin{Thm}[\textbf{#1}]}{\end{Thm}}
\newenvironment{PropWithTitle}[1]{%
	\begin{Prop}[\textbf{#1}]}{\end{Prop}}
\newenvironment{ExWithTitle}[1]{%
	\begin{Ex}[\textbf{#1}]}{\end{Ex}}
\newenvironment{DefWithTitle}[1]{%
	\begin{Def}[\textbf{#1}]}{\end{Def}}
\newenvironment{RemarkWithTitel}[1]{%
	\begin{remark}[\textbf{#1}]}{\end{remark}}

%format of paragraph 
\renewcommand\paragraph{\@startsection{paragraph}{4}{\z@}%
	{-2.5ex\@plus -1ex \@minus -.25ex}%
	{1.25ex \@plus .25ex}%
	{\normalfont\normalsize\bfseries}}
\makeatother
\setcounter{secnumdepth}{5} % how many sectioning levels to assign numbers to
\setcounter{tocdepth}{4}    % how many sectioning levels to show in ToC

\newcounter{row} 
\renewcommand\therow{\alph{row}} %hier a,b,c etc. def und mit therow abrufbar

\newenvironment{aufz}
{\setcounter{row}{0}%
	\par\noindent\tabularx{\linewidth}[t]
	{\cdot{20}{>{\stepcounter{row}\makebox[1.5em][l]{\therow)\hfill}}X}} %bis max 20 Elemente nebeinander
}
{\endtabularx}


%biblio
\usepackage[]{biblatex}
\addbibresource{referenzenma.bib} 

%glossary
\usepackage{glossaries}
\usepackage{import}

%\newglossaryentry{AdenylateCyclase}{
	name={Adenylate Cyclase},
	description={Also called adenylyl cyclase, this membrane-bound enzyme converts ATP to cAMP in response to stimulation by G-proteins. It is a key player in many GPCR-mediated pathways.}
}



\newglossaryentry{Allornothingsignal}{
	name={All or nothing signal},
	description={A type of cellular response that occurs only after a threshold level of signal is reached, resulting in a binary, digital-like outcome.}
}


\newglossaryentry{AlphaHelix}{
	name={Alpha Helix (AH)},
	description={A common structural motif in proteins, including in G-proteins and GPCRs, where it can play a role in conformational change upon activation.}
}



\newglossaryentry{AlphaSubunit}{
	name={Alpha Subunit},
	description={The component of a heterotrimeric G-protein that binds GDP/GTP and dissociates upon activation to regulate downstream effectors such as adenylate cyclase or phospholipase C.}
}



\newglossaryentry{Arrestins}{
	name={Arrestins},
	description={Proteins that bind phosphorylated GPCRs, blocking further G-protein activation and targeting receptors for internalization or alternate signaling.}
}



\newglossaryentry{AutocrineSignaling}{
	name={Autocrine Signaling},
	description={A form of signaling in which a cell secretes signaling molecules that bind to receptors on its own surface, allowing it to regulate itself.}
}



\newglossaryentry{Autophosphorylation}{
	name={Autophosphorylation},
	description={A process in which a kinase adds a phosphate group to itself, often leading to sustained activation independent of the original signal, as seen in CaM-Kinase II.}
}



\newglossaryentry{BetaComplex}{
	name={Beta Complex},
	description={Part of the G-protein beta-gamma dimer, it remains membrane-associated and contributes to the regulation of ion channels and other signaling proteins.}
}



\newglossaryentry{Calcium}{
	name={Ca\textsuperscript{2+}},
	description={A ubiquitous intracellular second messenger that regulates a wide range of cellular processes including muscle contraction, secretion, metabolism, and gene expression. Its release is often triggered by IP\textsubscript{3} in response to upstream signaling events.}
}


\newglossaryentry{Calmodulin}{
	name={Calmodulin},
	description={A small calcium-binding protein that undergoes conformational change upon Ca\textsuperscript{2+} binding, enabling it to activate target enzymes such as CaM-Kinase II.}
}



\newglossaryentry{CaMKII}{
	name={CaM-Kinase II},
	description={Short for calcium/calmodulin-dependent protein kinase II, an important serine/threonine kinase that decodes calcium oscillations via autophosphorylation and regulates memory, gene expression, and metabolism.}
}



\newglossaryentry{cAMP}{
	name={cAMP},
	description={Short for cyclic adenosine monophosphate, a second messenger synthesized by adenylate cyclase that activates downstream targets like PKA and regulates cellular responses.}
}



\newglossaryentry{cAMPPhosphodiesterase}{
	name={cAMP Phosphodiesterase},
	description={An enzyme that degrades cAMP into AMP, thereby terminating the cAMP signaling pathway. It is a key modulator of signal duration.}
}



\newglossaryentry{CellSignaling}{
	name={Cell Signaling},
	description={A fundamental process by which cells detect, interpret, and respond to external or internal cues through molecular signals. It involves extracellular signaling molecules binding to specific receptors, triggering intracellular signaling cascades that regulate cellular functions such as gene expression, metabolism, division, or apoptosis. Cell signaling enables coordination in multicellular organisms and is essential for development, immune response, and homeostasis.}
}



\newglossaryentry{CellSurfaceReceptor}{
	name={Cell-Surface Receptor},
	description={A transmembrane protein located on the cell membrane that binds extracellular signaling molecules (ligands), such as hormones or neurotransmitters. Upon ligand binding, it initiates an intracellular signaling cascade without the ligand entering the cell. Major classes include ion-channel-coupled receptors, G-protein-coupled receptors, and enzyme-coupled receptors.}
}



\newglossaryentry{cGMP}{
	name={cGMP},
	description={Short for cyclic guanosine monophosphate, a second messenger similar to cAMP that is produced by guanylate cyclase and regulates processes like phototransduction and vasodilation.}
}



\newglossaryentry{cGMPPhosphodiesterase}{
	name={cGMP Phosphodiesterase},
	description={An enzyme activated in visual transduction that hydrolyzes cGMP to GMP, leading to the closing of ion channels in photoreceptor cells.}
}



\newglossaryentry{ConstitutivelyActive}{
	name={Constitutively Active},
	description={Describes a receptor or signaling protein that is active without the need for ligand binding, often due to mutations or abnormal expression.}
}


\newglossaryentry{ContactDependentSignaling}{
	name={Contact Dependent Signaling},
	description={A signaling mechanism that requires direct membrane-to-membrane contact between cells, typically involving membrane-bound ligands and receptors.}
}


\newglossaryentry{CREB}{
	name={CREB},
	description={Short for cAMP response element-binding protein, a transcription factor phosphorylated by PKA that regulates genes involved in memory, survival, and metabolism.}
}



\newglossaryentry{DAG}{
	name={DAG},
	description={Short for diacylglycerol, a lipid-derived second messenger produced by PLC that activates protein kinase C and regulates membrane-associated signaling.}
}



\newglossaryentry{DownstreamCascade}{
	name={Downstream Cascade},
	description={A sequence of biochemical events triggered by receptor activation, involving multiple intermediates and amplifying the original signal to produce a cellular response.}
}



\newglossaryentry{Effectorproteins}{
	name={Effector proteins},
	description={Proteins that execute the final cellular response to a signal, such as changes in gene expression, metabolism, or cytoskeletal structure.}
}


\newglossaryentry{EndocrineSignaling,hormonalsignaling}{
	name={Endocrine Signaling},
	description={A.k.a. hormonal signaling, Long-range signaling in which hormones are secreted into the bloodstream and act on distant target cells.}
}


\newglossaryentry{Enzymecoupledreceptors}{
	name={Enzyme coupled receptors},
	description={Transmembrane receptors that have intrinsic enzymatic activity or are associated with enzymes activated by ligand binding.}
}


\newglossaryentry{ExtracellularSignalingMolecule}{
	name={Extracellular Signaling Molecule},
	description={A molecule, such as a hormone or neurotransmitter, that is released from one cell to bind receptors on another and initiate signaling.}
}


\newglossaryentry{G-protein-coupledreceptors}{
	name={G-protein-coupled receptors},
	description={A large family of membrane receptors (GPCRs) that activate intracellular G-proteins upon ligand binding to transmit signals.}
}


\newglossaryentry{GammaComplex}{
	name={Gamma Complex},
	description={Forms a functional dimer with the beta subunit in heterotrimeric G-proteins, anchoring the complex to membranes and participating in downstream signaling.}
}



\newglossaryentry{GAP}{
	name={GAP},
	description={GTPase-activating proteins that enhance the intrinsic GTPase activity of G-proteins, leading to signal termination.}
}


\newglossaryentry{GAPJunctions}{
	name={GAP Junctions},
	description={Specialized intercellular connections that allow direct chemical communication between adjacent cells via diffusion of small molecules and ions.}
}



\newglossaryentry{GEF}{
	name={GEF},
	description={Guanine nucleotide exchange factors, proteins that activate GTP-binding proteins by promoting the exchange of GDP for GTP.}
}


\newglossaryentry{GPCR}{
	name={GPCR},
	description={Short for G-protein-coupled receptor, a family of 7-pass transmembrane receptors that activate G-proteins in response to extracellular ligands. They are among the most abundant and versatile signaling receptors in eukaryotic cells.}
}



\newglossaryentry{GPCRKinases}{
	name={GPCR Kinases (GRKs)},
	description={A family of kinases that phosphorylate activated GPCRs, initiating their desensitization by promoting arrestin binding.}
}



\newglossaryentry{GProteins}{
	name={G-Proteins},
	description={Short for guanine nucleotide-binding proteins, these molecular switches relay signals from receptors (like GPCRs) to intracellular effectors by cycling between GDP-bound (inactive) and GTP-bound (active) states.}
}



\newglossaryentry{Gq}{
	name={Gq},
	description={A subclass of heterotrimeric G-proteins that activates phospholipase C, leading to intracellular calcium release and activation of protein kinase C.}
}



\newglossaryentry{GTPases}{
	name={GTPases},
	description={Enzymes that hydrolyze GTP to GDP and phosphate, acting as molecular switches in signaling pathways.}
}


\newglossaryentry{GTPbinding}{
	name={GTP binding},
	description={A regulatory mechanism by which proteins, especially G-proteins, toggle between active and inactive states depending on GTP or GDP binding.}
}


\newglossaryentry{GuanylateCyclase}{
	name={Guanylate Cyclase},
	description={An enzyme that converts GTP to cGMP upon activation by nitric oxide or natriuretic peptides, initiating cGMP-mediated signaling pathways.}
}


		
		
		
\newglossaryentry{HeterotrimericGProtein}{
	name={Heterotrimeric G Protein},
	description={A type of G-protein composed of three distinct subunits—alpha, beta, and gamma—that relay signals from GPCRs to downstream effectors.}
}



\newglossaryentry{Hyperbolicsignal}{
	name={Hyperbolic signal},
	description={A graded signal response that increases steadily with ligand concentration and eventually plateaus, resembling Michaelis-Menten kinetics.}
}


\newglossaryentry{Inhibitorysignals}{
	name={Inhibitory signals},
	description={Signals that suppress or diminish cellular responses, often balancing excitatory pathways for proper cell regulation.}
}


\newglossaryentry{InsulinReceptorSubstrate(IRS)}{
	name={Insulin Receptor Substrate (IRS)},
	description={A docking protein phosphorylated by the insulin receptor, serving as a scaffold for downstream signaling molecules.}
}



\newglossaryentry{IntracellularReceptor}{
	name={Intracellular Receptor},
	description={A receptor located within the cytoplasm or nucleus that binds small, hydrophobic signaling molecules (e.g., steroid hormones) that cross the plasma membrane. Upon activation, many intracellular receptors function as transcription factors that directly modulate gene expression.}
}



\newglossaryentry{Ion-channel-coupledreceptors}{
	name={Ion-channel-coupled receptors},
	description={Receptors that open or close ion channels in response to ligand binding, converting chemical signals into electrical ones.}
}


\newglossaryentry{Lipidrecruitment}{
	name={Lipid recruitment},
	description={The process of signaling molecules being recruited to specific membrane lipids, such as PIP3, for spatial activation.}
}


\newglossaryentry{Longfeedbackdelay}{
	name={Long feedback delay},
	description={A feedback loop that acts over a longer time scale, potentially leading to oscillations or long-term regulation.}
}


\newglossaryentry{ModularInteractionDomain}{
	name={Modular Interaction Domain},
	description={Protein domains that mediate specific interactions with phosphorylated or lipid-modified partners in signaling complexes.}
}


\newglossaryentry{Molecularswitches}{
	name={Molecular switches},
	description={Molecules, often proteins, that toggle between 'on' and 'off' states to propagate or terminate signals.}
}


\newglossaryentry{NegativeFeedback}{
	name={Negative Feedback},
	description={A regulatory mechanism in which a signaling output inhibits an earlier step, stabilizing the pathway.}
}


\newglossaryentry{Neurotransmitter}{
	name={Neurotransmitter},
	description={A chemical messenger that transmits signals across synapses from one neuron to another.}
}

%Section - Cell Signaling: the World of G-Proteins


\newglossaryentry{Oscillation}{
	name={Oscillation},
	description={In cell signaling, a periodic fluctuation in the concentration or activity of signaling molecules (such as Ca\textsuperscript{2+}) that conveys dynamic information to control gene expression or cellular responses.}
}



\newglossaryentry{ParacrineSignaling}{
	name={Paracrine Signaling},
	description={Short-range signaling where secreted molecules affect nearby target cells without entering the bloodstream.}
}


\newglossaryentry{Phosphatidylserine}{
	name={Phosphatidylserine},
	description={A negatively charged phospholipid found on the inner leaflet of the plasma membrane that helps localize signaling proteins like PKC through electrostatic interactions.}
}



\newglossaryentry{PhospholipaseC}{
	name={Phospholipase C (PLC)},
	description={A membrane-associated enzyme activated by certain G-proteins (like Gq), which hydrolyzes phosphoinositides to generate DAG and IP\textsubscript{3}, initiating calcium signaling and PKC activation.}
}



\newglossaryentry{Phosphorylation}{
	name={Phosphorylation},
	description={The addition of a phosphate group to a protein or other molecule, often regulating activity or interactions.}
}


\newglossaryentry{PhosphotyrosineBinding(PTB)}{
	name={Phosphotyrosine Binding (PTB)},
	description={A domain that binds phosphorylated tyrosines on target proteins, mediating recruitment in signaling pathways.}
}


\newglossaryentry{PI}{
	name={PI},
	description={Short for phosphatidylinositol, a membrane phospholipid that can be phosphorylated to form various signaling lipids like PI(4,5)P\textsubscript{2}, which are substrates for PLC and involved in many signaling pathways.}
}



\newglossaryentry{PKA}{
	name={PKA},
	description={Short for protein kinase A, a serine/threonine kinase activated by cAMP that phosphorylates various substrates to regulate metabolism, gene expression, and other cellular processes.}
}



\newglossaryentry{PKC}{
	name={PKC},
	description={Short for protein kinase C, a family of serine/threonine kinases activated by DAG and calcium that phosphorylate a variety of cellular proteins involved in growth, metabolism, and differentiation.}
}



\newglossaryentry{PLC}{
	name={PLC},
	description={Short for phospholipase C, a key enzyme in Gq-mediated signaling that generates second messengers DAG and IP\textsubscript{3} from membrane phospholipids. See also Phospholipase C.}
}



\newglossaryentry{PleckstrinHomology(PH)}{
	name={Pleckstrin Homology (PH)},
	description={A protein domain that binds phosphoinositides in membranes, targeting proteins to specific locations.}
}


\newglossaryentry{PositiveFeedback}{
	name={Positive Feedback},
	description={A mechanism in which a signaling output enhances an earlier step, amplifying the signal.}
}


\newglossaryentry{ProteinRecruitment}{
	name={Protein Recruitment},
	description={The assembly of signaling complexes at specific membrane sites or proteins through binding domains.}
}


\newglossaryentry{ProteinResponse}{
	name={Protein Response},
	description={The cellular outcome of a signaling event, typically involving activation or repression of specific proteins.}
}


\newglossaryentry{RasDomain}{
	name={Ras Domain},
	description={A conserved GTP-binding domain found in small GTPases like Ras, involved in signal transduction and downstream activation of pathways such as MAPK.}
}



\newglossaryentry{Receptor}{
	name={Receptor},
	description={A protein, usually on the cell surface or in the cytoplasm, that binds a specific signaling molecule and initiates a response.}
}


\newglossaryentry{Rhodopsin}{
	name={Rhodopsin},
	description={A light-sensitive GPCR found in photoreceptor cells of the retina that activates the visual transduction pathway via transducin and cGMP breakdown.}
}



\newglossaryentry{Scaffoldingprotein}{
	name={Scaffolding protein},
	description={A protein that binds multiple signaling components, organizing them into functional complexes to enhance efficiency and specificity.}
}


\newglossaryentry{Shortfeedbackdelay}{
	name={Short feedback delay},
	description={A feedback loop that acts rapidly after signal initiation, often stabilizing or fine-tuning the signal.}
}


\newglossaryentry{SigmoidalSignal}{
	name={Sigmoidal Signal},
	description={A type of cellular response curve characterized by a slow initiation, followed by a steep increase, and then saturation—forming an “S” shape. It often reflects cooperative binding or multi-step signaling cascades, allowing cells to respond sensitively to threshold changes in stimulus concentration.}
}



\newglossaryentry{Signalingcascade}{
	name={Signaling cascade},
	description={A series of biochemical events, often involving sequential activation of enzymes, leading to a cellular response.}
}


\newglossaryentry{SignalIntegration}{
	name={Signal Integration},
	description={The cellular process of combining inputs from multiple signaling pathways to generate a unified response.}
}


\newglossaryentry{SrcHomology(SH)}{
	name={Src Homology (SH)},
	description={A family of protein domains (e.g., SH2, SH3) involved in recognizing phosphorylated tyrosines or proline-rich motifs.}
}


\newglossaryentry{SynapticSignaling}{
	name={Synaptic Signaling},
	description={A specialized form of signaling in neurons where neurotransmitters are released at synapses to stimulate adjacent cells.}
}


\newglossaryentry{Transcriptionalresponse}{
	name={Transcriptional response},
	description={Changes in gene expression triggered by signaling pathways reaching the nucleus.}
}


\newglossaryentry{Transducin}{
	name={Transducin},
	description={A heterotrimeric G-protein specifically involved in visual signaling, activated by rhodopsin to stimulate cGMP phosphodiesterase.}
}

\newglossaryentry{HalfLife}{
	name={Half-Life},
	description={The time required for the concentration of a substance—such as a signaling molecule, mRNA, or protein—to decrease to half of its initial value.}
}


\newglossaryentry{AKT}{
	name={AKT},
	description={A serine/threonine-specific protein kinase also known as Protein Kinase B, involved in promoting cell survival and growth through downstream effects of PI3K signaling.}
}

\newglossaryentry{Cholesterol}{
	name={Cholesterol},
	description={A lipid molecule essential for membrane structure and function; it also serves as a precursor for steroid hormones and plays a role in modulating signaling pathways such as Hedgehog.}
}

\newglossaryentry{CK1}{
	name={CK1},
	description={Short for Casein Kinase 1, a serine/threonine kinase that phosphorylates signaling components in the Wnt and Hedgehog pathways.}
}

\newglossaryentry{Costal2}{
	name={Costal2},
	description={A kinesin-like protein in the Hedgehog pathway that forms a complex with Smoothened and regulates the processing of Cubitus Interruptus.}
}

\newglossaryentry{CubitusInterruptus}{
	name={Cubitus Interruptus (Ci)},
	description={A transcription factor regulated by the Hedgehog pathway in Drosophila; acts as a repressor or activator depending on Hedgehog signal presence.}
}

\newglossaryentry{Delta}{
	name={Delta},
	description={A membrane-bound ligand for the Notch receptor that plays a critical role in lateral inhibition during development.}
}

\newglossaryentry{Disheveled}{
	name={Disheveled},
	description={A cytoplasmic protein activated by Frizzled in the Wnt pathway; it inhibits the degradation complex to stabilize β-catenin.}
}

\newglossaryentry{Dimerization}{
	name={Dimerization},
	description={The process by which two receptor molecules associate, often as a prerequisite for activation, especially in receptor tyrosine kinases (RTKs).}
}

\newglossaryentry{EGFKinase}{
	name={EGF Kinase},
	description={A kinase domain found in the Epidermal Growth Factor Receptor (EGFR), involved in autophosphorylation upon ligand binding and dimerization.}
}

\newglossaryentry{EGFR}{
	name={EGF-R},
	description={Epidermal Growth Factor Receptor, a receptor tyrosine kinase (RTK) that activates downstream pathways like Ras/MAPK upon EGF binding.}
}

\newglossaryentry{Endocytosis}{
	name={Endocytosis},
	description={A cellular process by which extracellular materials are internalized via vesicles; in signaling, it helps regulate receptor availability and pathway duration.}
}

\newglossaryentry{epithelialcells}{
	name={Epithelial cells},
	description={Cells that line surfaces and cavities of organs, involved in barrier function, absorption, and signaling processes like lateral inhibition.}
}

\newglossaryentry{Erk}{
	name={Erk},
	description={Extracellular signal-regulated kinase, part of the MAPK signaling pathway downstream of Ras, involved in cell proliferation and differentiation.}
}

\newglossaryentry{FRET}{
	name={FRET},
	description={Short for Förster Resonance Energy Transfer, a technique used to study molecular interactions based on energy transfer between two fluorescent molecules.}
}

\newglossaryentry{FGFR}{
	name={FGFR},
	description={Fibroblast Growth Factor Receptor, a type of RTK that triggers signaling cascades involved in development and cell differentiation.}
}

\newglossaryentry{frizzled}{
	name={Frizzled},
	description={A family of G-protein-coupled receptors that bind Wnt proteins and initiate the Wnt/β-catenin signaling pathway.}
}

\newglossaryentry{Grb2}{
	name={Grb2},
	description={An adaptor protein that links RTKs like EGFR to Ras activation via interaction with SOS, part of the Ras/MAPK pathway.}
}

\newglossaryentry{Groucho}{
	name={Groucho},
	description={A transcriptional co-repressor that inhibits Wnt target gene expression by binding LEF1 in the absence of β-catenin.}
}

\newglossaryentry{GSK3}{
	name={GSK3},
	description={Short for Glycogen Synthase Kinase 3 (often GSK3), involved in phosphorylating β-catenin to promote its degradation in the Wnt pathway.}
}

\newglossaryentry{iHog}{
	name={iHog},
	description={Short for Interference Hedgehog, a co-receptor in the Hedgehog signaling pathway that facilitates ligand reception and signaling.}
}

\newglossaryentry{IKKalpha}{
	name={IKK alpha},
	description={A kinase that is part of the IκB kinase complex; phosphorylates IκB, promoting its degradation and thereby activating NFκB signaling.}
}

\newglossaryentry{IKKbeta}{
	name={IKK beta},
	description={A key catalytic subunit of the IKK complex, required for the canonical NFκB pathway activation through phosphorylation of IκB.}
}

\newglossaryentry{JAK}{
	name={JAK},
	description={Janus Kinase, a family of non-receptor tyrosine kinases associated with cytokine receptors, activating STAT proteins upon phosphorylation.}
}

\newglossaryentry{kinasecytokinereceptors}{
	name={Kinase cytokine receptors},
	description={Receptors that lack intrinsic kinase activity but associate with tyrosine kinases like JAK to transduce signals from extracellular cytokines.}
}

\newglossaryentry{kinaseinsertregion}{
	name={Kinase insert region},
	description={A flexible loop within the kinase domain of some RTKs involved in substrate specificity or regulation of kinase activity.}
}

\newglossaryentry{LEF1}{
	name={LEF1},
	description={Lymphoid enhancer-binding factor 1, a transcription factor activated by β-catenin in the Wnt signaling pathway.}
}

\newglossaryentry{LateralInhibition}{
	name={Lateral Inhibition},
	description={A process during development where a cell inhibits its neighbors from adopting the same fate, often mediated by Notch-Delta signaling.}
}

\newglossaryentry{LRP}{
	name={LRP},
	description={Low-density lipoprotein receptor-related protein, a Wnt co-receptor that cooperates with Frizzled to activate downstream signaling.}
}

\newglossaryentry{MAP}{
	name={MAP},
	description={Short for Mitogen-Activated Protein; involved in cascades such as Ras-MAPK that control gene expression, cell division, and survival.}
}

\newglossaryentry{mTORC1}{
	name={mTORC1},
	description={Mechanistic target of rapamycin complex 1, regulates protein synthesis, metabolism, and cell growth; activated downstream of AKT.}
}

\newglossaryentry{mTORC2}{
	name={mTORC2},
	description={A signaling complex that phosphorylates AKT and regulates the cytoskeleton and survival pathways.}
}

\newglossaryentry{NEMO}{
	name={NEMO},
	description={NFκB essential modulator, a regulatory subunit of the IKK complex that is crucial for NFκB activation.}
}

\newglossaryentry{Nemo}{
	name={Nemo},
	description={Alternative name for NEMO, required for assembling the IKK complex and regulating NFκB signaling.}
}

\newglossaryentry{NFKB}{
	name={NFkB},
	description={A transcription factor that regulates immune and inflammatory responses, activated upon degradation of its inhibitor IκB.}
}

\newglossaryentry{Notch}{
	name={Notch},
	description={A membrane-bound receptor that, upon binding Delta, undergoes proteolytic cleavage releasing the Notch intracellular domain (NICD) to regulate gene transcription.}
}

\newglossaryentry{Notchtail}{
	name={Notch tail},
	description={Also known as NICD (Notch Intracellular Domain), it translocates to the nucleus and associates with Rbpsuh to influence transcription.}
}

\newglossaryentry{Patched}{
	name={Patched},
	description={A receptor in the Hedgehog pathway that inhibits Smoothened in the absence of Hedgehog ligand.}
}

\newglossaryentry{PDK1}{
	name={PDK1},
	description={Phosphoinositide-dependent kinase-1, activates AKT by phosphorylation following PI3K signaling.}
}

\newglossaryentry{Phosphotyrosin}{
	name={Phosphotyrosine},
	description={A phosphorylated tyrosine residue that serves as a binding site for SH2 domain-containing proteins in signaling pathways.}
}

\newglossaryentry{PI3K}{
	name={PI3K},
	description={Phosphoinositide 3-kinase, an enzyme activated by RTKs that produces PIP3, leading to AKT activation.}
}

\newglossaryentry{protooncogene}{
	name={Proto-oncogene},
	description={A normal gene that can become an oncogene through mutation or overexpression, promoting cell proliferation or survival.}
}

\newglossaryentry{PTEN}{
	name={PTEN},
	description={A phosphatase that antagonizes PI3K signaling by dephosphorylating PIP3 to PIP2, acting as a tumor suppressor.}
}

\newglossaryentry{Ras}{
	name={Ras},
	description={A small GTPase that transmits signals from RTKs to MAPK cascades, promoting proliferation and differentiation.}
}

\newglossaryentry{RasMPK}{
	name={Ras MPK},
	description={Refers to the Ras-MAPK pathway, a cascade where Ras activates RAF, MEK, and ERK, leading to gene regulation and cell proliferation.}
}

\newglossaryentry{Rbpsuh}{
	name={Rbpsuh},
	description={Recombination signal-binding protein for immunoglobulin kappa J region, a transcription factor that partners with NICD in Notch signaling.}
}

\newglossaryentry{RTK}{
	name={RTK},
	description={Short for Receptor Tyrosine Kinase, a class of cell-surface receptors that activate intracellular signaling via tyrosine phosphorylation.}
}

\newglossaryentry{Smad}{
	name={Smad},
	description={Intracellular proteins that transmit signals from TGF-beta receptors to the nucleus to regulate transcription.}
}

\newglossaryentry{Smoothened}{
	name={Smoothened},
	description={A transmembrane protein activated in the Hedgehog pathway upon relief of Patched inhibition, triggering downstream signaling.}
}

\newglossaryentry{STAT}{
	name={STAT},
	description={Signal Transducer and Activator of Transcription; phosphorylated by JAKs and then dimerizes to regulate gene expression.}
}

\newglossaryentry{TGFbeta}{
	name={TGF beta},
	description={Transforming Growth Factor beta, a family of cytokines involved in cell proliferation, differentiation, and immune regulation via Smad signaling.}
}

\newglossaryentry{TNF}{
	name={TNF},
	description={Tumor Necrosis Factor, a cytokine involved in systemic inflammation, apoptosis, and immune system signaling through receptors like TNFR.}
}

\newglossaryentry{typeIreceptor}{
	name={Type I receptor},
	description={A component of the TGF-beta receptor complex that is phosphorylated by type II receptors to propagate Smad signaling.}
}

\newglossaryentry{typeIIreceptor}{
	name={Type II receptor},
	description={A receptor that binds TGF-beta ligand and phosphorylates the associated type I receptor to initiate downstream signaling.}
}

\newglossaryentry{Wnt}{
	name={Wnt},
	description={A family of secreted glycoproteins that activate Frizzled receptors and regulate β-catenin stabilization, crucial for development and cell fate.}
}

\newglossaryentry{androgen}{
	name={Androgen},
	description={A group of steroid hormones like testosterone that regulate male traits and reproductive activity.}
}

\newglossaryentry{androgenreceptor}{
	name={Androgen receptor (AR)},
	description={A type of intracellular receptor that binds androgens, then translocates to the nucleus to regulate target gene transcription.}
}

\newglossaryentry{Transphosphorylation}{
	name={Transphosphorylation},
	description={A process in which one kinase phosphorylates another kinase, often occurring during receptor activation, such as with receptor tyrosine kinases (RTKs) where two adjacent receptors phosphorylate each other upon dimerization.}
}

\newglossaryentry{TumorSuppressor}{
	name={Tumor Suppressor},
	description={A gene or protein that prevents uncontrolled cell growth by regulating the cell cycle, promoting apoptosis, or repairing DNA. Loss-of-function mutations in tumor suppressors can contribute to cancer.}
}

\newglossaryentry{ProteolyticCleavage}{
	name={Proteolytic Cleavage},
	description={A biochemical process where specific peptide bonds in a protein are broken by proteases, activating or deactivating signaling molecules or receptors, such as in Notch or Hedgehog signaling pathways.}
}

\newglossaryentry{BetaCatenin}{
	name={Beta Catenin},
	description={A multifunctional protein involved in the Wnt signaling pathway and in cell adhesion. In Wnt signaling, its stabilization leads to nuclear translocation and activation of Wnt target genes.}
}

\newglossaryentry{Axin}{
	name={Axin},
	description={A scaffold protein that forms part of the destruction complex in Wnt signaling. It promotes degradation of beta-catenin in the absence of Wnt signals.}
}

\newglossaryentry{Hedgehog}{
	name={Hedgehog},
	description={A secreted signaling molecule that regulates cell growth and patterning during development. Binding of Hedgehog to the Patched receptor activates Smoothened, initiating downstream signaling through proteins like Cubitus interruptus (Ci).}
}
\newglossaryentry{MEK}{
	name={MEK},
	description={Mitogen-activated protein kinase kinase (MAPKK), a dual-specificity kinase that phosphorylates and activates ERK in the MAPK signaling cascade. MEK acts downstream of Raf and plays a key role in transmitting growth signals from the cell membrane to the nucleus.}
}

\newglossaryentry{Raf}{
	name={Raf},
	description={A serine/threonine-specific protein kinase (MAPKKK) that is activated by Ras in the MAPK pathway. Raf phosphorylates and activates MEK, initiating a kinase cascade involved in cell division and differentiation.}
}




\makeglossaries
\newglossaryentry{4,5-dihydroototic acid}{
	name={4,5-dihydroototic acid},
	description={An intermediate in pyrimidine biosynthesis formed by the cyclization of carbamoyl aspartic acid via dihydroorotase.}
}

\newglossaryentry{5'-nucleotidase}{
	name={5'-nucleotidase},
	description={An enzyme that hydrolyzes phosphate from nucleotides like AMP during purine degradation.}
}

\newglossaryentry{ATP}{
	name={ATP},
	description={Adenosine triphosphate, the primary energy currency of the cell used to drive many biosynthetic reactions.}
}

\newglossaryentry{Adenosine}{
	name={Adenosine},
	description={A nucleoside formed from adenine and ribose, a breakdown product of AMP that is deaminated to inosine.}
}

\newglossaryentry{Alanine}{
	name={Alanine},
	description={Formed by transamination of pyruvate, catalyzed by alanine transaminase, and involved in nitrogen transport.}
}

\newglossaryentry{Alanine transaminase (ALT)}{
	name={Alanine transaminase (ALT)},
	description={An enzyme that transfers an amino group from glutamate to pyruvate to form alanine, also a liver marker.}
}

\newglossaryentry{Arginine}{
	name={Arginine},
	description={A urea cycle amino acid synthesized from glutamate via a pathway involving acylation, reduction, and conversion through ornithine.}
}

\newglossaryentry{Asparagine}{
	name={Asparagine},
	description={Synthesized from aspartate via amidation by asparagine synthase using glutamine as NH3 donor and ATP.}
}

\newglossaryentry{Asparagine Synthase}{
	name={Aspargine Synthase},
	description={An enzyme that catalyzes the ATP-dependent conversion of aspartate and glutamine into asparagine.}
}

\newglossaryentry{Aspartate}{
	name={Aspartate},
	description={Formed from oxaloacetate by transamination, serves as a precursor for nucleotides and participates in the urea cycle.}
}

\newglossaryentry{Aspartate carbamoyltransferase}{
	name={Aspartate carbamoyltransferase},
	description={An enzyme that catalyzes the condensation of aspartate with carbamoyl phosphate in pyrimidine biosynthesis.}
}

\newglossaryentry{Aspartate transaminase (AST)}{
	name={Aspartate transaminase (AST)},
	description={An enzyme that transfers an amino group to oxaloacetate forming aspartate, used as a liver function marker.}
}

\newglossaryentry{Biogenic amines}{
	name={Biogenic amines},
	description={Low molecular weight amines derived from amino acids, acting as neurotransmitters or hormones.}
}

\newglossaryentry{CTP}{
	name={CTP},
	description={Cytidine triphosphate, synthesized from UTP and used in RNA synthesis and lipid metabolism.}
}

\newglossaryentry{CTP synthetase}{
	name={CTP synthetase},
	description={An enzyme that converts UTP to CTP using glutamine as the nitrogen donor and ATP for energy.}
}

\newglossaryentry{Chemotherapics}{
	name={Chemotherapics},
	description={Drugs that target rapidly dividing cells, often by inhibiting nucleotide biosynthesis (e.g., thymidylate synthesis).}
}

\newglossaryentry{CoA}{
	name={CoA},
	description={Coenzyme A, a cofactor involved in acyl group transfer reactions, derived in part from nucleotide structures.}
}

\newglossaryentry{Creatine}{
	name={Creatine},
	description={A compound derived from glycine, arginine, and methionine, converted into phosphocreatine for ATP regeneration.}
}

\newglossaryentry{Cysteine}{
	name={Cysteine},
	description={An amino acid synthesized from serine and homocysteine via cystathionine, involved in antioxidant synthesis (e.g., glutathione).}
}

\newglossaryentry{DNA}{
	name={DNA},
	description={Deoxyribonucleic acid, the carrier of genetic information built from deoxyribonucleotides.}
}

\newglossaryentry{FAD}{
	name={FAD},
	description={Flavin adenine dinucleotide, a redox cofactor involved in many metabolic reactions.}
}

\newglossaryentry{FMN}{
	name={FMN},
	description={Flavin mononucleotide, a coenzyme derived from riboflavin, functioning in redox reactions.}
}

\newglossaryentry{GTP}{
	name={GTP},
	description={Guanosine triphosphate, a nucleotide used as an energy source and a precursor for RNA.}
}

\newglossaryentry{Glutahione}{
	name={Glutahione},
	description={An antioxidant tripeptide composed of glutamate, cysteine, and glycine, protecting cells from oxidative damage.}
}

\newglossaryentry{Glutamate}{
	name={Glutamate},
	description={An amino acid that acts as a nitrogen donor in biosynthesis and is produced in bacteria from $\alpha$-ketoglutarate and glutamine via glutamate synthase.}
}

\newglossaryentry{Glutamate synthase}{
	name={Glutamate synthase},
	description={An enzyme found in bacteria that converts $\alpha$-ketoglutarate and glutamine into two molecules of glutamate.}
}

\newglossaryentry{Glutamine}{
	name={Glutamine},
	description={A nitrogen-carrying amino acid formed from glutamate and NH4+ by glutamine synthase; serves as a nitrogen donor in many biosynthetic reactions.}
}

\newglossaryentry{Glutamine amidotransferase}{
	name={Glutamine amidotransferase},
	description={An enzyme with two domains that transfers NH3 from glutamine to other molecules, using a glutamyl-enzyme intermediate and ATP-activated acceptors.}
}

\newglossaryentry{Glutamine synthase}{
	name={Glutamine synthase},
	description={An enzyme that catalyzes the ATP-dependent conversion of glutamate and ammonium into glutamine.}
}

\newglossaryentry{Glutathione synthetase}{
	name={Glutathione synthetase},
	description={A biochemical term relevant to amino acid or nucleotide metabolism: Glutathione synthetase.}
}

\newglossaryentry{Glycine}{
	name={Glycine},
	description={A derivative of serine, synthesized by removing a carbon via serine hydroxymethyltransferase using tetrahydrofolate and PLP as cofactors.}
}

\newglossaryentry{Heme}{
	name={Heme},
	description={A biochemical term relevant to amino acid or nucleotide metabolism: Heme.}
}

\newglossaryentry{Methionine}{
	name={Methionine},
	description={An essential amino acid that donates a methyl group in the biosynthesis of creatine and also forms homocysteine in cysteine synthesis.}
}

\newglossaryentry{NAD+}{
	name={NAD+},
	description={Nicotinamide adenine dinucleotide, a coenzyme involved in redox reactions.}
}

\newglossaryentry{NADP+}{
	name={NADP+},
	description={A phosphorylated form of NAD+ used in anabolic redox reactions.}
}

\newglossaryentry{Nucleotides}{
	name={Nucleotides},
	description={A biochemical term relevant to amino acid or nucleotide metabolism: Nucleotides.}
}

\newglossaryentry{PLP}{
	name={PLP},
	description={Pyridoxal phosphate, an active form of vitamin B6 and a coenzyme involved in amino acid metabolism, including transamination and decarboxylation.}
}

\newglossaryentry{Phosphocreatine (Pcr)}{
	name={Phosphocreatine (Pcr)},
	description={A biochemical term relevant to amino acid or nucleotide metabolism: Phosphocreatine (Pcr).}
}

\newglossaryentry{Phosphoribosyltransferases}{
	name={Phosphoribosyltransferases},
	description={A biochemical term relevant to amino acid or nucleotide metabolism: Phosphoribosyltransferases.}
}

\newglossaryentry{Polyamines}{
	name={Polyamines},
	description={Cationic molecules derived from ornithine involved in DNA binding and translational regulation.}
}

\newglossaryentry{Porphyria}{
	name={Porphyria},
	description={A group of diseases caused by defective enzymes in the heme biosynthetic pathway, leading to accumulation of porphyrin intermediates.}
}

\newglossaryentry{Porphyrins}{
	name={Porphyrins},
	description={Heterocyclic macrocycles that coordinate metal ions; precursors to heme.}
}

\newglossaryentry{Proline}{
	name={Proline},
	description={A cyclic derivative of glutamate formed through phosphorylation, reduction, and spontaneous cyclization of glutamate semialdehyde.}
}

\newglossaryentry{Purine}{
	name={Purine},
	description={A nitrogenous base composed of fused imidazole and pyrimidine rings; a building block for DNA and RNA.}
}

\newglossaryentry{Pyrimidine}{
	name={Pyrimidine},
	description={A six-membered nitrogen-containing ring found in cytosine, uracil, and thymine.}
}

\newglossaryentry{RNA}{
	name={RNA},
	description={Ribonucleic acid, a nucleic acid involved in gene expression, composed of ribonucleotides.}
}

\newglossaryentry{Serine}{
	name={Serine},
	description={An amino acid derived from 3-phosphoglycerate (glycolysis intermediate) via oxidation, transamination, and dephosphorylation steps.}
}

\newglossaryentry{Thymidine}{
	name={Thymidine},
	description={A nucleoside component of DNA formed from thymine and deoxyribose, used in salvage pathways.}
}

\newglossaryentry{Thymidine kinase}{
	name={Thymidine kinase},
	description={An enzyme that phosphorylates thymidine to TMP in nucleotide salvage pathways.}
}

\newglossaryentry{Thymidine phosphorylase}{
	name={Thymidine phosphorylase},
	description={An enzyme that salvages thymine by converting it to thymidine using sugar phosphates.}
}

\newglossaryentry{Thymidylate (dTMP)}{
	name={Thymidylate (dTMP)},
	description={A nucleotide synthesized from dUMP by thymidylate synthase; essential for DNA replication.}
}

\newglossaryentry{Uridine phosphorylase}{
	name={Uridine phosphorylase},
	description={An enzyme that adds ribose-1-phosphate to uracil in pyrimidine salvage.}
}

\newglossaryentry{Uridine-cytidine kinase}{
	name={Uridine-cytidine kinase},
	description={An enzyme that phosphorylates uridine and cytidine in salvage pathways.}
}

\newglossaryentry{adenine phosphoribosyltransferases (APRT)}{
	name={adenine phosphoribosyltransferases (APRT)},
	description={A biochemical term relevant to amino acid or nucleotide metabolism: adenine phosphoribosyltransferases (APRT).}
}

\newglossaryentry{adenosine monophosphate (AMP)}{
	name={adenosine monophosphate (AMP)},
	description={A biochemical term relevant to amino acid or nucleotide metabolism: adenosine monophosphate (AMP).}
}

\newglossaryentry{adenylosuccinate lyase}{
	name={adenylosuccinate lyase},
	description={A biochemical term relevant to amino acid or nucleotide metabolism: adenylosuccinate lyase.}
}

\newglossaryentry{alpha-ketobutyrate}{
	name={alpha-ketobutyrate},
	description={A by-product in the synthesis of cysteine from homocysteine and serine, ultimately entering the TCA cycle.}
}

\newglossaryentry{cAMP}{
	name={cAMP},
	description={Cyclic AMP, a second messenger derived from ATP involved in signal transduction pathways.}
}

\newglossaryentry{cGMP}{
	name={cGMP},
	description={Cyclic GMP, a signaling molecule and second messenger involved in cellular responses to hormones.}
}

\newglossaryentry{carbamoyl aspartic acid}{
	name={carbamoyl aspartic acid},
	description={A biochemical term relevant to amino acid or nucleotide metabolism: carbamoyl aspartic acid.}
}

\newglossaryentry{carbamoyl phosphate}{
	name={carbamoyl phosphate},
	description={A biochemical term relevant to amino acid or nucleotide metabolism: carbamoyl phosphate.}
}

\newglossaryentry{cytidine deaminase}{
	name={cytidine deaminase},
	description={A biochemical term relevant to amino acid or nucleotide metabolism: cytidine deaminase.}
}

\newglossaryentry{d-aminolevulinate}{
	name={d-aminolevulinate},
	description={A biochemical term relevant to amino acid or nucleotide metabolism: d-aminolevulinate.}
}

\newglossaryentry{deoxyuridine}{
	name={deoxyuridine},
	description={A biochemical term relevant to amino acid or nucleotide metabolism: deoxyuridine.}
}

\newglossaryentry{dihydroorotase}{
	name={dihydroorotase},
	description={A biochemical term relevant to amino acid or nucleotide metabolism: dihydroorotase.}
}

\newglossaryentry{dihydroorotate oxidase}{
	name={dihydroorotate oxidase},
	description={A biochemical term relevant to amino acid or nucleotide metabolism: dihydroorotate oxidase.}
}

\newglossaryentry{essential amino acids}{
	name={essential amino acids},
	description={Amino acids that cannot be synthesized by the human body and must be obtained from the diet.}
}

\newglossaryentry{folate}{
	name={folate},
	description={A biochemical term relevant to amino acid or nucleotide metabolism: folate.}
}

\newglossaryentry{glutamate-cysteine ligase (GCL)}{
	name={glutamate-cysteine ligase (GCL)},
	description={A biochemical term relevant to amino acid or nucleotide metabolism: glutamate-cysteine ligase (GCL).}
}

\newglossaryentry{guanosine monophosphate}{
	name={guanosine monophosphate},
	description={A biochemical term relevant to amino acid or nucleotide metabolism: guanosine monophosphate.}
}

\newglossaryentry{hypoxanthine}{
	name={hypoxanthine},
	description={A biochemical term relevant to amino acid or nucleotide metabolism: hypoxanthine.}
}

\newglossaryentry{hypoxanthine-guanine phosphoribosyltransferases (HGPRT)}{
	name={hypoxanthine-guanine phosphoribosyltransferases (HGPRT)},
	description={A biochemical term relevant to amino acid or nucleotide metabolism: hypoxanthine-guanine phosphoribosyltransferases (HGPRT).}
}

\newglossaryentry{inosine}{
	name={inosine},
	description={A biochemical term relevant to amino acid or nucleotide metabolism: inosine.}
}

\newglossaryentry{inosine monophosphate (IMP)}{
	name={inosine monophosphate (IMP)},
	description={A biochemical term relevant to amino acid or nucleotide metabolism: inosine monophosphate (IMP).}
}

\newglossaryentry{kinases}{
	name={kinases},
	description={A biochemical term relevant to amino acid or nucleotide metabolism: kinases.}
}

\newglossaryentry{methylmalonyl semialdehyde}{
	name={methylmalonyl semialdehyde},
	description={A biochemical term relevant to amino acid or nucleotide metabolism: methylmalonyl semialdehyde.}
}

\newglossaryentry{neurotransmitter}{
	name={neurotransmitter},
	description={A biochemical term relevant to amino acid or nucleotide metabolism: neurotransmitter.}
}

\newglossaryentry{pyrimidine-nucleoside phosphorylase}{
	name={pyrimidine-nucleoside phosphorylase},
	description={A biochemical term relevant to amino acid or nucleotide metabolism: pyrimidine-nucleoside phosphorylase.}
}

\newglossaryentry{ribonucleotide reductase (RNR)}{
	name={ribonucleotide reductase (RNR)},
	description={A biochemical term relevant to amino acid or nucleotide metabolism: ribonucleotide reductase (RNR).}
}

\newglossaryentry{salvage pathways}{
	name={salvage pathways},
	description={A biochemical term relevant to amino acid or nucleotide metabolism: salvage pathways.}
}

\newglossaryentry{serine hydroxymethyltransferase}{
	name={serine hydroxymethyltransferase},
	description={An enzyme that converts serine to glycine by removing a carbon, requiring tetrahydrofolate and PLP as cofactors.}
}

\newglossaryentry{severe combined immune deficiency (ADA-SCID)}{
	name={severe combined immune deficiency (ADA-SCID)},
	description={A biochemical term relevant to amino acid or nucleotide metabolism: severe combined immune deficiency (ADA-SCID).}
}

\newglossaryentry{succinyl-CoA}{
	name={succinyl-CoA},
	description={A biochemical term relevant to amino acid or nucleotide metabolism: succinyl-CoA.}
}

\newglossaryentry{transamination}{
	name={transamination},
	description={A biochemical process where an amino group is transferred from one molecule (usually glutamate or glutamine) to another, forming new amino acids.}
}

\newglossaryentry{uric acid}{
	name={uric acid},
	description={A biochemical term relevant to amino acid or nucleotide metabolism: uric acid.}
}

\newglossaryentry{uridine}{
	name={uridine},
	description={A biochemical term relevant to amino acid or nucleotide metabolism: uridine.}
}

\newglossaryentry{uridine monophosphate (UMP)}{
	name={uridine monophosphate (UMP)},
	description={A biochemical term relevant to amino acid or nucleotide metabolism: uridine monophosphate (UMP).}
}

\newglossaryentry{uridine triphosphate (UTP)}{
	name={uridine triphosphate (UTP)},
	description={A biochemical term relevant to amino acid or nucleotide metabolism: uridine triphosphate (UTP).}
}

\newglossaryentry{beta-aminoisobutyrate}{
	name={beta-aminoisobutyrate},
	description={A biochemical term relevant to amino acid or nucleotide metabolism: $\beta$-aminoisobutyrate.}
}

\newglossaryentry{beta-ureidopropionase}{
	name={beta-ureidopropionase},
	description={A biochemical term relevant to amino acid or nucleotide metabolism: $\beta$-ureidopropionase.}
}

\newglossaryentry{gamma-glutamylcysteine}{
	name={gamma-glutamylcysteine},
	description={A biochemical term relevant to amino acid or nucleotide metabolism: $\gamma$-glutamylcysteine.}
}

\newglossaryentry{bilirubin}{
    name={bilirubin},
    description={A yellow compound that occurs in the normal catabolic pathway that breaks down heme in red blood cells. It is excreted in bile and urine, and elevated levels may indicate liver dysfunction or disease}
}

\newglossaryentry{tyms}{
    name={thymidylate synthase (TYMS)},
    description={An essential enzyme that catalyzes the conversion of deoxyuridine monophosphate (dUMP) to deoxythymidine monophosphate (dTMP), a key step in DNA synthesis and repair. It is a target for certain anticancer drugs}
}

\newglossaryentry{dihydrofolate_reductase}{
    name={dihydrofolate reductase},
    description={An enzyme that reduces dihydrofolate (DHF) to tetrahydrofolate (THF), a form required for the synthesis of purines, thymidylic acid, and certain amino acids. It plays a critical role in DNA synthesis and cell replication, and is a target of drugs like methotrexate}
}





\begin{document}
	
\section{Cell Junctions and the Extracellular Matrix}

The organization of our cell is a pretty complex system. It can be split into two major components: The \textbf{\gls{epithelialtissue}} and the \textbf{\gls{connectivetissue}}. The \textbf{\gls{basallamina}} separates the two, providing stability and also selective permeability between the two tissue types. The \textbf{\gls{extracellularmatrix}} with the \textbf{\gls{collagenfibers}}, bears a lot of the mechanical stress on the tissue. This chapter will be looking at the connections between cells, cells and the ecm, and with \textbf{\gls{cytoskeletalproteins}}, focusing also how a cell can sense its surroundings.

\begin{figure}[H]
	\centering
	\includegraphics[width=0.7\linewidth]{tiss_overview}
	\caption{An overview of the tissue structure between the epithelial tissue and the extracelullar matrix.}
	\label{fig:tissoverview}
\end{figure}


\subsection{Overview of the Types and Groups of \gls{celljunctions}}

There are four main types of \gls{celljunctions}, which each have specific functions, as well as subgroups:
\begin{enumerate}
	\item \textbf{\gls{anchoringjunctions}}: adhesion of cell-cell (c-c) or cell-matrix(c-m) connections.
	\begin{itemize}
		\item Actin related: \gls{adherensjunctions} for c-c and \gls{actinlinkedjunctions} for c-m
		\item \gls{intermediatefilamentattachmentsites}: \gls{desmosomes} for c-c and \gls{hemidesmosomes} for c-m.
	\end{itemize}
	\item \textbf{\gls{occludingjunctions}}: Block inter-cellular passage, causing impermeable or selectively permeable barriers
	\begin{itemize}
		\item \gls{tightjunctions} (vertebrates)
		\item \gls{septatejunctions} (in non-vertebrates, not covered in course)
	\end{itemize}
	\item \textbf{\gls{channelformingjunctions}}: Enable communication between two cell interiors (ex. GAP junctions - electrical conductance between two cells).
	\begin{itemize}
		\item \gls{Gapjunctions} (animals)
		\item \gls{plasmodesmata} (plants, not covered in course)
	\end{itemize}
	\item \textbf{\gls{signalrelayingjunctions}}: To enable communication between two adjacent cells (ex. neurological synapse).
	\begin{itemize}
		\item \gls{chemicalsynapses} (nervous system)
		\item \gls{immunologicalsynapses} (immune system, not covered in course)
		\item \gls{transmembraneligandreceptorcellcellsignalingcontacts} (Delta-Notch, ephrin-Eph, etc.) 
		\item Anchoring, occluding, and channel-forming junctions can all have signaling functions in addition to their structural roles.
	\end{itemize}
\end{enumerate}

\begin{figure}[H]
	\centering
	\includegraphics[width=0.7\linewidth]{junct_types}
	\caption{The four main types of junctions between cell-cell and cell-matrix.}
	\label{fig:juncttypes}
\end{figure}

The same cell will have many different types of junctions. A \textbf{\gls{junctioncomplex}} is a tight junction, at the most apical position, followed by an adheren junction, followed by a desmosome. These three "glue" a cell together. The figure \ref{fig:junctoverex} shows all the different types of junctions, based off of an epithelial cell of the small intestine: 

\begin{figure}[H]
	\centering
	\includegraphics[width=0.7\linewidth]{junct_overex}
	\caption{An example of the types of junctions, based off of an epithelial cell in the small intestine.}
	\label{fig:junctoverex}
\end{figure}


There are two types cells bind to each other:
\begin{enumerate}
	\item \textbf{\textbf{\gls{homophilicbinding}}}: both cells have identical extracellular proteins which bind together.
	\item \textbf{\textbf{\gls{heterophilicbinding}}}: two different proteins attach to each other. Common for signaling (e.g., Delta-Notch).
\end{enumerate}

\subsection{Cell-Cell Anchoring Junctions}

There are four main different anchoring junctions, two for cell-cell and two for cell-matrix. Here is an overview (fig: \ref{fig:anchover}) of each type and what there main actors are:

\begin{figure}[H]
	\centering
	\includegraphics[width=0.7\linewidth]{anch_over}
	\caption{An overview of the different types of anchoring junctions.}
	\label{fig:anchover}
\end{figure}


\subsubsection{Cadherins and Adherens Junction}

There are a lot of different \gls{cadherin} superfamily members. All of them are structurally related in the ecm domain, as they all have the cadherin domains. The number of \gls{cadherindomain} varies strongly (5 for classical, 4 or 5 for \gls{desmogleins} and \gls{desmocollins}, and at time over 30 for \gls{nonclassicalcadherins}. In addition, their functions, and intracelullar proteins vary strongly. Some even lack the transmembrane element (e.g., T-cadherin who is attached through a GPI anchor). All this means a strong variety of cadherins in the superfamily. Here (fig: \ref{fig:cadsuper}) are some examples: 
\begin{figure}[H]
	\centering
	\includegraphics[width=0.3\linewidth]{cad_super}
	\caption{Shows some members of cadherin superfamily.}
	\label{fig:cadsuper}
\end{figure}

Cadherins mostly bind \textbf{homophically}. The extracellular of a classical cadherin contain five copies of the cadherin domain, separated by five flexible hinge regions. 

(see fig: \ref{fig:cadCaInt}) \gls{Ca2} ions bind in the neighborhood of each hinge, preventing it from flexing. As a result cadherin forms a rigid, curved structure, which allows it to enter in binding with another rigid cadherin. In the absence of Ca2+ the cadherin will be more flexible resulting in a floppy molecule that can't interact with a different cadherin. Leading to a failure of adhesion. This means that a \textbf{sufficiently high concentration of Ca2+ is essential for cell adhesion}.

(see fig: \ref{fig:cadnter}) To generate the cell-cell adhesion the cadherin domain at the N-terminal tip of one cadherin binds to the domain of the other cadherin.

(see fig: \ref{fig:cadCaInt}) At a typical cell junction, an organized array of cadherin molecules functions like Velcro. Cadherins on the same cell are thought to be coupled by side-to-side interactions between their N-terminals, resulting in a linear array. Each cadherin (green) will bind to a cadherin on the other cell (blue) that is in a perpedicular array to it. This will lead to a tight-knit structure.

\begin{figure}[H]
	\centering
	\includegraphics[width=0.6\linewidth]{cad_nter}
	\caption{The interaction between two different cadherins at their N-terminals}
	\label{fig:cadnter}
\end{figure}
\begin{figure}[H]
	\centering
	\subfigure[The interaction between cadherins and Ca2 and how it relates to stiffness of the molecule.]{
		\includegraphics[width=0.4\linewidth]{cad_Ca2}
	}
	\hfill
	\subfigure[How the cadherins chain together like velcro in multiple directions. The arrays of interaction are perpendicular to each other. Having multiple arrays, then gives us a tight knit mat.]{
		\includegraphics[width=0.5\linewidth]{cad_move}
	}
	\caption{}
	\label{fig:cadCaInt}
\end{figure}

\textbf{Classical cadherins interact with the cytoskeleton}. The interaction between cadherin and the \textbf{\gls{actin}} filaments is indirect and mediated by an adaptor complex, which includes \textbf
{\gls{betacatenin}}, which we saw in Wnt signaling. Further it contains \textbf{\gls{p120catenin}} and \textbf{\gls{alphacatenin}}. Further proteins such as \textbf{\gls{vinculin}} associate with $\alpha$-catenin and provide further actin links. This mean that \textbf{multiple actins will interact with one cadherin, but through different mediator proteins}. \\

The adaptor complex will undergo a conformational change when cadherin is attached to another cadherin. The tension can also be increased through a \textbf{\gls{myosinII}}. The \textbf{increased tension on cadherin, causes $\alpha$-catenin to extend}, which in turn allows proteins like vinculin to attach associate and recruit further actins, strengthening the link between cytoskeleton and the junction. In essence, the \textbf{higher the tension, the more the cell strengthens the junction}.

\begin{figure}[H]
	\centering
	\includegraphics[width=0.4\linewidth]{cad_tens}
	\caption{How increased tension causes more recruitment of actin, strengthening the cytoskeleton and junction.}
	\label{fig:cadtens}
\end{figure}

(see fig: \ref{fig:cadexpa})This combined strengthening by cytoskeleton and junctions can organize the cells and cytoskeleton. When the first cadherin attaches it becomes easier for the neighboring ones to attach, which leads to the formation of clusters. This leads to the activation of an actin regulator, \textbf{\gls{GTPaseRac}}, which promotes further cadherin binding. This causes a more widespread junction. Eventually it is inhibited and replaced by the related \textbf{\gls{GTPaseRho}}, which moves the actins into a more linear form and promotes myosin II recruitment. That allows contractile move along the cell membrane allows cadherins up and down stream to be activated too. This further expands the junction.

\begin{figure}[H]
	\centering
	\includegraphics[width=0.6\linewidth]{cad_expa}
	\caption{The expansion of a cadherin junction.}
	\label{fig:cadexpa}
\end{figure}

\begin{RemarkWithTitel}{Sorting of cells with the help of cadherins}
	Thanks to the number of different cadherins and homophylic nature of cadherins, cells tend to associate better with some cell than others and sort themselves accordingly. In an experiment this was shown, by purposefully mixing up cells and then seeing them re-associate.
	
	\begin{figure}[H]
		\centering
		\includegraphics[width=0.5\linewidth]{cad_reas}
		\caption{This experiment shows how cells reassociate thanks to having specific cadherin bonds. This allows cells to sort themselves to the correct cell types.}
		\label{fig:cadreas}
	\end{figure}
	
\end{RemarkWithTitel}

\subsubsection{Adhesion Belts by Adherens Junctions}

The actins in the cell between two cadherins can form a direct line from one adherens junction to one on the other side of the cell. If this line is continued between a number of cells, it forms an \textbf{\gls{adhesionbelt}}. This gives a lot of stability and allows for the formation of pretty set structures between epithelial cells. 

\begin{figure}[H]
	\centering
	\includegraphics[width=0.4\linewidth]{cad_micr}
	\caption{Shows an example of an adhesion belt, in the case of \gls{microvilli}.}
	\label{fig:cadmicr}
\end{figure}

\begin{RemarkWithTitel}{Microvilli in the small intestine:}
	In the small intestine it is important theat all the microvilli are tightly packed to maximize surface are. Adhesion belts, keep the cells in place.
\end{RemarkWithTitel}

\begin{RemarkWithTitel}{Use Case: Developmental Biology}
	The adhesion belt helps in cell development, by providing structure and connectivity between cell. For example, when creating an the neural tube in early vertebrate development, it helps the cells to narrow at their apex and roll into a tube.
	
	\begin{figure}[H]
		\centering
		\includegraphics[width=0.4\linewidth]{cad_deve}
		\caption{How adhesions belts assist the development of cell groups.}
		\label{fig:caddeve}
	\end{figure}
	
	Looking more at the development of the neural tube, we can also observe that at different parts of the adhesion belt, we will have different cadherins. This will have the effect that they will segregate to each other, making it easier to break away form the \gls{ectoderm}.
	
	\begin{figure}[H]
		\centering
		\includegraphics[width=0.7\linewidth]{cad_neur}
		\caption{How having different cadherins causes certain shapes to form, looking at the use case of neural development.}
		\label{fig:cadneur}
	\end{figure}
	
\end{RemarkWithTitel}

\subsubsection{EMT and MET}

Some key terms and their meaning:
\begin{itemize}
	\item \gls{mesenchymal}: multipotent stromal (connective tissue) cells that differentiate into a variety of cells including: fibroblasts, osteoblasts, chondrocytes, adipocytes, myocytes.
	\item \gls{EMT}: epithelial-to-mesenchymal transition
	\item \gls{MET}: Mesenchymal-to-epithelial transition
\end{itemize}

Now, looking at the \textbf{EMT} first: epithelial cells loose their polarity, as well as their cell adhesion, through the dissolution of tight junctions, adherence junctions (with that E-cadherin and cytoskeletal reorganization), and desmosomes. \\

Next up \textbf{MET}:  starting with the initial adhesive contact, then the adherens junction (with it cytoskeletal reorganization), and then the desmosome associateion. Finally tight junctions will form.

\begin{figure}[H]
	\centering
	\subfigure[Shows the cycle of met and emt epithelial and mesenchymal cells can go through.]{
		\includegraphics[width=0.5\linewidth]{met_emt}
		}
		\subfigure[Shows how cancer uses EMT and MET]{
		\includegraphics[width=0.4\linewidth]{met_canc}
	}
	\caption{}
	\label{fig:metemt}
\end{figure}

\begin{RemarkWithTitel}{The body applying EMT and MET}
	\textbf{EMT} is used in:
	\begin{itemize}
		\item \textbf{embryonic development}: helps cell move, adapt to for new cell gorups;
		\item \textbf{Wound healing}: helps cells migrate to wound;
		\item \textbf{Cancer metastasis}: allows cells to migrate.
	\end{itemize}
	\textbf{MET} is used in:
	\begin{itemize}
		\item \textbf{embryonic development}: helps tissue grow together, form, and specialize (less talked about);
		\item \textbf{Wound healing}: crucial in repair of damage (fills the gaps);
		\item \textbf{Cancer metastasis}: allows it to settle into new area.
	\end{itemize}
\end{RemarkWithTitel}

\begin{RemarkWithTitel}{Some transcription factors which regulate EMT}
	\textbf{\gls{Twist}, \gls{Snail}, \gls{Slug}, and \gls{Zeb}} are transcription factors that \textbf{drive EMT}. They do this by repressing epithelial genes (mainly cadherins) or activating mesenchymal genes (\gls{fibronectin} and \gls{vimentin}).
\end{RemarkWithTitel}

\subsubsection{Desmosomes and Hemidesmosomes}
\label{sec:demso}

The structure of a desmosome is as follows:
\begin{itemize}
	\item On the cytoplasmic surface is a dense plaque composed of a mix of intracellular adaptor proteins. Some of these components are:
	\begin{itemize}
		\item \textbf{\gls{desmogleins}} and \textbf{\gls{desmocollins}} are \gls{nonclassicalcadherins}. Their tails bind to \textbf{\gls{plakiglobin}} ($\gamma$-catenin) amd \textbf{plakophillin} (distant relative of p120-catenin). Together they turn into a \textbf{desmoplakin}.
		\item Desmoplakin binds to the sides of intermediate filaments, tying the \gls{desmosomes} to the filaments.
	\end{itemize}
	\item To this plaque a bunch of keratin intermediate filaments are attached.
	\item On the other side of the plaque a lot of \gls{nonclassicalcadherins} bind to the plaque, whose extracellular domains interact with the \gls{cadherin} of another molecule.
\end{itemize}

\begin{figure}[H]
	\centering
	\subfigure[Shows a rough overview of how a desmosome anchor junction looks like.]{
		\includegraphics[width=0.4\linewidth]{desm_over}
	}
	\subfigure[Shows the plaque in more detail.]{
		\includegraphics[width=0.55\linewidth]{desmo_dets}
	}
	\caption{}
	\label{fig:desm}
\end{figure}

While a desmosome is cell-cell a hemidesomosome is cell-matrix (hence hemi = half). For the cell side of things the structure is exactly the same. Both types of junctions give rigidity to the cell.

\subsection{Tight Junctions}

\gls{tightjunctions} are everywhere, like tracts in the urinary system. 

Tight junctions hold adjacent membranes very close together. The strands are composed of transmembrane proteins that make contact across the intercellular space, creating a seal. They due this multiple times in high numbers, which creates a very large surface area and with that a strong seal. The sealing strand is composed mainly of proteins with four transmembrane elements. The main one is \textbf{\gls{claudin}}, secondary \textbf{\gls{occludin}} have less of an important role in determining \textbf{junction permeability}. The two termini for both this proteins are on the cytoplasmic side of the membrane where they interact with scaffolding proteins and link to actin to organize the sealing strands.

\begin{figure}[H]
	\centering
	\includegraphics[width=0.6\linewidth]{tight_over}
	\caption{Shows a model of a tight junction. Highlights how sealing strands exist around the molecule. Also shows the main two sealing proteins: claudin and occludin.}
	\label{fig:tightover}
\end{figure}


\begin{RemarkWithTitel}{3D-thinking of tight junctions}
	In 3D these tight junctions will wrap around the cell forming a band. As otherwise they wouldn't actually block anything if they just existed at selected spots
\end{RemarkWithTitel}

\subsubsection{Tight Junctions in transcellular transport: Intestine}

tight junctions \textbf{seal off different parts of the tissue}. This allows for the body to create transfers from one part to the other in a more controlled version. Tight junctions also confine transport proteins to their part of the membrane, working as a \textbf{fence}, within the lipid bilayer. They also \textbf{block the backflow} of unwanted molecules.

\begin{figure}[H]
	\centering
	\includegraphics[width=0.6\linewidth]{tight_glu}
	\caption{Shows the intake of glucose in the small intestinge. For simplicity only tight junctions are shown of the anchor junctions. Glucose is actively transported into the cell through Na+-driven glucose transporters and leaves passively through glucose transporters.}
	\label{fig:tightglu}
\end{figure}


\subsection{Channel-Forming Junctions}

The essence: \textbf{Gap junctions decide which molecules are shared between cells}. However, there is a limit at around \textbf{1000 Daltons}.

The gap junction is seen as a cluster of homogeneous intramembrane particles. Each intramembrane particle is a protein assembly called a \textbf{\gls{connexon}}, consisting of \textbf{6 \gls{connexin} subunits}, which penetrate the lipid bilayer. Connecting two connexons then creates a channel between two cells. These connexons can be \textbf{\gls{homotypic}} or \textbf{\gls{heterotypic}}, depending on the usage of different conenxins. Each connexin consists mainly of $\alpha$-helix, with the whole connexon ending up having a pore size of around 1.4nm which matches with the molecule size permitted (around 1000 Daltons).

\begin{figure}[H]
	\centering
	\includegraphics[width=0.8\linewidth]{gap_over}
	\caption{(A) Shows the view on the membrane, (B) the components of a connexon, (C) and the 3D structure of a connexon.}
	\label{fig:gapover}
\end{figure}


\subsection{The Extracellular Matrix}

The major components of the extracellular matrix (ecm) are the following:
\begin{itemize}
	\item \gls{glycoprotein}: \gls{laminin}, \gls{nidogen}, and \gls{fibronectin}
	\item \gls{fibrous}: \gls{typeIVcollagen}, \gls{fibrillarcollagen}
	\item \gls{proteoglycan} and \gls{glycosaminoglycanGAGs}: \gls{hyaluronan}, \gls{perlecan}, \gls{decorin}, \gls{aggrecan}
\end{itemize} 

\begin{figure}[H]
	\centering
	\includegraphics[width=0.6\linewidth]{ecm_comp}
	\caption{Shows the major components of the ecm. Green is protein and red is GAG.}
	\label{fig:ecmcomp}
\end{figure}

The size of these molecules also varies very much:
\begin{itemize}
	\item globular protein (MW 50'000)
	\item glycogen (MW around 400'000)
	\item spectrin (MW 460'000)
	\item collagen (MW 290'000)
	\item hyaluronan (MW 8 x $10^{6}$ and 300nm in diameter)
\end{itemize}

\subsubsection{Gylcosylation and GAGs}

GAGs are a long chain of \textbf{typically sulfated repeating disaccharides}, this means that we will need a bunch of glycosylation bonds. Something about \textbf{GAGs is that they are often sulfated}. It will vary from 70\% (\gls{heparin}) to under 50\% (heparan) or none at all (\gls{hyaluronan}). Further the length can vary from 200 pairs of disaccharides to up to 25'000 sugar monomers, again highlighting the high variability. Leading to even higher variability is the fact that the disaccharide chain is also very variable: for \textbf{\gls{chondroitinsulfate}} it is D-glucuronic acid and N-acetyl-D-galactosamine, while for \textbf{\gls{heparansulfate}} it is N-acetyl-D-glucosamine with either D-glucuronic acid or L-iduronic acid, and finally for \textbf{\gls{keratansulfate}} it is D-galactose and N-actyl-D-glucosamine. 


\begin{figure}[H]
	\centering
	\includegraphics[width=0.8\linewidth]{gag_sulf}
	\caption{The picture on the left shows a GAG with 100\% sulfation (not really a thing). On the right we can see hyaluronan, a rather simple but long GAG (no sulfation).}
	\label{fig:gagsulf}
\end{figure}
 

\paragraph{Synthesis proteoglycan: adding GAGs to proteins}

GAGs are added to their core protein via a special link tetrasaccharide of the GAG and a serine on the protein. Once this linkage has happened the rest of the GAG repeating disaccharide chain can be added one sugar at a time. 

\begin{figure}[H]
	\centering
	\includegraphics[width=0.7\linewidth]{gag_bond}
	\caption{How a proteoglycan bind the GAG and protein.}
	\label{fig:gagbond}
\end{figure}

\begin{RemarkWithTitel}{Proteoglycans varies a lot}
	The Glycosylation degree is very variable in size and absolute number. Aggrecan for example has 300 amino acids in its core protein and 30 keratan sulfate and 100 chondroitin sulfate chains linked to the protein. On the other hand decorin just has one GAG and "decorates" collagen fibrils (so it can't be all too large).
	
	\begin{figure}[H]
		\centering
		\includegraphics[width=0.7\linewidth]{gag_ex}
		\caption{Different proteoglycans and their varying sizes and numebrs of glycosylations.}
		\label{fig:gagex}
	\end{figure}
	
\end{RemarkWithTitel}

\subsubsection{Aggrecan aggregation}

Aggrecan is the name of the core protein. It has many keratan sulfate glycosylations. The N-terminal of aggrecan then binds noncovalently to a single hyaluronan molecule. A link protein, part of the hyaluronan-binding proteins (can also be cell surface proteins), then binds to both the aggrecan core and the hyaluronan stabilizing the bond. This aggregate can become huge north of $10^{8}$ daltons and occupy the volume of a bacterium (2x$10^{-12}cm^{3}$).

\begin{figure}[H]
	\centering
	\includegraphics[width=0.5\linewidth]{agg_comp}
	\caption{A visualization of the aggrecan aggregate on a hyaluronan molecule}
	\label{fig:aggcomp}
\end{figure}



\subsubsection{Collagen}

\paragraph{Structure of a typical collagen}
Collagen is composed of three $\alpha$ chains. One $\alpha$ chain is a long left-handed helix with a set pattern: every third amino acid is a \gls{glycine}. The other two can be anything but are commonly a \gls{hydroxyproline} (Y) (modified during collagen synthesis) and a \gls{proline} (X). The reason every third amino acid needs to be glycine is because for the three $\alpha$ chains to wrap into each other, one of the amino acids needs to fit in between and the only amino acid small enough for that is glycine. The entire collage will become up to 300nm long.

\begin{figure}[H]
	\centering
	\includegraphics[width=0.3\linewidth]{coll_struc}
	\caption{Structure of collagens. X is typically proline and y hydroxypriline}
	\label{fig:collstruc}
\end{figure}

\paragraph{The collagen amino acids}

Nearly one third of amino acids in collagen is glycine. 15-30\% are Proline and 4-Hydroxyprolyl (Hyp).  Then 3-Hydroxyprolyl and 5-\gls{hydroxylysyl} (Hyl) residues also occur in collagen, but in smaller amounts. All of these hydroxylation reactions are \textbf{\gls{VitaminC}} dependent, as it is a cofactor for the \textbf{enzymes \gls{lysylhydroxylase} and \gls{prolylhydroxylase}}.

\begin{RemarkWithTitel}{Collagen as connective Tissue: Fibrils}
	\gls{collagenfibers} are organized into bundles which run through the ecm. They are oriented in nearly a right angle, creating a net of fibrils. 
\end{RemarkWithTitel}

\paragraph{Collagen types}
\begin{figure}[H]
	\centering
	\subfigure[A bunch of different types of collagen]{
		\includegraphics[width=0.45\linewidth]{coll_type}
	}
	\subfigure[(a) shows how collagen forms its trimeric form, and then from it all the diverse forms it can take.]{
		\includegraphics[width=0.4\linewidth]{coll_typeI}
	}
\end{figure}


\paragraph{Synthesis of fibril Collagen I}

The synthesis of \textbf{\gls{collagenI}} happens in \textbf{\gls{fibroblasts}}. It happens in three parts: first \gls{procollagen} is assembled in the ER, then in the cytosol the fibril is assembled. The fiber is then assembled in the ecm. Breaking down the individual parts: \\

\textbf{Procollagen assembly}
\begin{enumerate}
	\item \textbf{\gls{procollagen}} assist folding into the left handed $\alpha$-helix
	\item \textbf{\gls{hydroxylation}} of Proline and Lysine
	\item N-linked glycosylation
	\item Beginning of quaterny structure through self-assembly of disulfide bonds.
	\item Proline bonds are also forced to be trans so they don't break apart the helical form.
	\item Formation of the triple helix
	\item transportation through \textbf{\gls{Golgiapparatus}}
	\item Modification of N- and O- linked sugars \\
	
	\textbf{Fibril/fiber assembly}
	\item Cleavage of \textbf{\gls{propeptides}}, which are the parts of the chain which didn't form the tight triple helix
	\item Self assembly of fibril
	\item Secretion
	\item Fiber assembly
\end{enumerate}

\begin{figure}[H]
	\centering
	\includegraphics[width=0.6\linewidth]{coll_synt}
	\caption{Synthesis of Collagen I a type of fibril.}
	\label{fig:collsynt}
\end{figure}


\paragraph{Defective collagen synthesis = bad news}

Having a defect in one of the proteins can be really bad really fast, as for the fibril fibers to do their job, we need everything to be packed very tightly in just the right way. For example a mutation in the gene for the \textbf{\gls{procollagenNproteinase}}, which is responsible for cutting the the parts of the gene which didn't fold properly. This will basically just mean that the collagen becomes useless.

\begin{figure}[H]
	\centering
	\includegraphics[width=0.6\linewidth]{coll_defe}
	\caption{How a defect in collagen synthesis is very very bad.}
	\label{fig:colldefe}
\end{figure}


\subsubsection{The ECM's Flexibility}

Unlike soccer (a.k.a. football) players the ecm needs to be able to stretch. For this it has a \gls{elastin} fiber. It is a bunch of elastin molecules bonded covalently to generate a cross-linked network. Each molecule can extend and coil, which allows the fiber as a whole to function as a rubber band. One elastin has a \textbf{long half life of around 40 years}. However, elastin is \textbf{not really regenerated after puberty}, which lead to Gesichtsfalten. 

\begin{figure}[H]
	\centering
	\includegraphics[width=0.5\linewidth]{ela_over}
	\caption{Elastin in its stretched vs. relaxed state.}
	\label{fig:elaover}
\end{figure}



\subsubsection{Complex Glycoproteins}

There are over 200 matrix glycoproteins in mammals. Many matrix glycoproteins are large \gls{scaffoldproteins} containing multiple copies of specific protein-interaction domains. Each domain is folded into a discrete globular structure, often having a bead like structure. Each protein contains multiple repeat domains. 
Example of Fibronectin which has a numerous copies of different \gls{fibronectin} repeats: \gls{FN1}, \gls{FN2}, and \gls{FN3}. Two type III at the end are crucial for integrin binding, while other position are important fibrin, collagen, or heparin binding. 

Other matrix proteins contain \gls{EGFepidermalgrowthfactor} (epidermal growth factor) like sequences, indicating that they might serve a similar signaling purpose. Others on the glycoprotein, like the \gls{IGFBPInsulingrowthbindingfactor} (IGFBP) regulate soluble growth factors. Many of these genes can be spliced, leading to even more diversity among glycoproteins.

Finally some of the domains are responsible for building multimeric forms. For example in fibronectin the C-termini builds dimers, in \gls{tenacin} and \gls{thrombospondin} form N-terminally linked hexamers and trimers, respectively.

\begin{figure}[H]
	\centering
	\includegraphics[width=0.6\linewidth]{glyc_comp}
	\caption{A bunch of complex glycoproteins in the ECM.}
	\label{fig:glyccomp}
\end{figure}


\subsubsection{Fibronectin}

\gls{fibronectin} plays a crucial role in guiding cell structure and behaviors.

\begin{figure}[H]
	\centering
	\includegraphics[width=0.6\linewidth]{fibr_struc}
	\caption{The structure of fibronectin. Note its minor differences between the chains.}
	\label{fig:fibrstruc}
\end{figure}

Looking at the fig: \ref{fig:fibrstruc} above we can see on the left that the two chains may be similar but not entirely the same, meaning they were spliced differently (as the same gene). They are joined by two disulfide bonds near the C-termini. Each chain is around 2'500 amino acids long and is folded into a bunch of domains. Some domains are specialized to binding to certain molecules. On the right the sequences in red are important for binding \gls{integrin}. \\

\begin{RemarkWithTitel}{Fibronectin under tension}
	Some type III fibronectin repeats can unfold when fibronectin is put under tension. That unfolding can expose cryptic binding sites resulting in multiple in the formation of multiple fibronectins.
\end{RemarkWithTitel}

\begin{figure}
	\centering
	\includegraphics[width=0.4\linewidth]{fibr_tens}
	\caption{Shows how some domains are exposed when fibronectin is pulled upon.}
	\label{fig:fibrtens}
\end{figure}



\begin{RemarkWithTitel}{Fibronectins and the cytoskeleton alinging}
	Fibronectin will accumulate at focal adhesions, making the organized in a paralell way to actin filaments. Integrin molecules link the fibronectin outside the cell to the actin filaments inside it (will be covered in more detail in section \ref{integrins}). Tension on the fibronectin exposes them exposing sites which promote fibril formation.
\end{RemarkWithTitel}



\subsection{Basal Lamina}

Depending on where we are in the body, the basal lamina will have a different organization. These differences in composition can also exist between tissues.
\begin{figure}[H]
	\centering
	\includegraphics[width=0.7\linewidth]{bl_diff}
	\caption{The basal lamina will look very different depending on where in the body it is.}
	\label{fig:bldiff}
\end{figure}

\subsubsection{Complexity of the basal lamina}

The \gls{basallamina} is formed by specific interaction between proteins, \gls{laminin}, \gls{typeIVcollagen}, and \gls{nidogen}, and the \gls{proteoglycan} \gls{perlecan}. Transmembrane laminin receptors, \gls{integrin} and \gls{dystroglycan} are thought to organize the assembly of the basal lamina.

\begin{figure}[H]
	\centering
	\includegraphics[width=0.8\linewidth]{bl_form}
	\caption{The assembly of the basal lamina is very complex. On the right, one can see the net of interaction. An arrow indicates who can bind to who.}
	\label{fig:blform}
\end{figure}


\subsection{Integrins}
\label{integrins}

\gls{integrin} are essential in connecting the two sides of the \gls{basallamina}. 
\begin{figure}[H]
	\centering
	\includegraphics[width=0.4\linewidth]{int_roug}
	\caption{A rough glance at integrins role for the basal lamina}
	\label{fig:introug}
\end{figure}


\subsubsection{The types of integrins}

\begin{figure}[H]
	\centering
	\includegraphics[width=0.6\linewidth]{int_type}
	\caption{Some types of integrins}
	\label{fig:inttype}
\end{figure}


\subsubsection{Integrin: The Major Activity States}

Integrin has to main states: inactive (folded) and active (extended). This switch happens spontaneously.
\begin{figure}[H]
	\centering
	\includegraphics[width=0.6\linewidth]{int_conf}
	\caption{The two different conformations of integrins.}
	\label{fig:intconf}
\end{figure}


\subsubsection{Integrins in hemidesomoses}

Hemidesomoses (see sec:\ref{sec:desmo}) glue epithelial cells to the basal lamina. They do this by linking keratin filaments on the inside and out outside of the cell. A specialized integrin($\alpha$6, $\beta$4) attaches to the \textbf{keratin} filaments, via adaptor proteins \textbf{\gls{plectin}} and \textbf{\gls{BP230}} and to the laminin extracellularly. The adhesive complex also contain a unusual collagen known as\textbf{\gls{collagenXVII}}, which has a membrane-spanning domain attached to its extracellular collagenous portion.

\begin{RemarkWithTitel}{Blisters due ot hemisdesomoses}
	Defects in any of these proteins may cause blistering of the skin. One such disease \textbf{\gls{bullouspemphigold}} is an autoimmune disease in which the immune system destroys its own collagen.
\end{RemarkWithTitel}


\subsubsection{Talin: tension sensor}

\gls{Talin} is an adaptor protein between integrins and actin filaments. Its long, flexible, C-terminus is divided into a series of folded domains, some of which are vinculin binding-sites, that are hidden when in a relaxed state. Then, once the Talin feels the tension through either the integrin or the actin it unwinds giving way to the vinculin-binding sites. This allows vinculin to attach and recruit more actin stabiliting the complex and relieving of tension.

\begin{figure}[H]
	\centering
	\includegraphics[width=0.6\linewidth]{int_tal}
	\caption{Shows where Talin has different binding domains and how once it gets started to be pulled apart those become uncovered. PSA: this is a pic from an experiment where they used a magnet to pull the molecule apart, we can ignore that.}
	\label{fig:inttal}
\end{figure}

\subsubsection{Activation of Integrin Signaling}
Signals received from outside the cell can activate integrin. In \gls{plateletsthrombocytes} (thrombocytes):
\begin{enumerate}
	\item \gls{thrombin} activates a GPCR on the cell surface.
	\item Which in turn activates \gls{Rap1}, a member of the GTPases. It should be said that \textbf{many other receptors can activate Rap1}!
	\item Rap1 interacts with \gls{RIAM}, which then recruits inactive talin and \gls{kindlin} to the membrane surface.
	\item Talin and kindlin interact with the \gls{integrinbeta} to trigger integrin activation.
	\item Talin hangs around to interact with vinculin and more.
\end{enumerate} 

\begin{figure}[H]
	\centering
	\includegraphics[width=0.9\linewidth]{int_path}
	\caption{The pathway to activate integrin.}
	\label{fig:intpath}
\end{figure}


\begin{RemarkWithTitel}{Talin activation}
	Talin is initially inactive due to a rod domain on the C-terminal and N-terminal which would contain the integrin-binding site but is now blocked. However when RIAM recruits Talin to the membrane, it interacts with a \gls{Phosphoinositide} (PI(4,5)P2) resulting in the dissociation of the \gls{roddomain}. Talin unfolds and binds to integrin.
\end{RemarkWithTitel}



\subsubsection{Integrins interacting with the ECM}

Different integrins interact with different components. Arginine-Glycine-Aspartic Acid, RGD are the three amino acids in fibronectin interacting with the integrin $\alpha$5$\beta$1

\begin{figure}[H]
	\centering
	\includegraphics[width=0.5\linewidth]{int_sign}
	\caption{The signal activation and progression from the perspective of integrin.}
	\label{fig:intsign}
\end{figure}


\subsubsection{Laminin}

\gls{Laminin} is very important for the basal lamina. Due to the binding sites with other proteins, laminin plays a central part in organizing and anchoring the basal lamina. Laminins are multidomain glycoproteins composed of three polypeptide ($\alpha$, $\beta$, $\gamma$), which are bonded through disulfide bonds, bonding them into an asymmetric crosslike structure. Each chain is over 1500 amino acids long. There are 5 $\alpha$, 4 $\beta$, and 3 $\gamma$ different types of chains known to us, which leads to a bunch of different combinations. \gls{Laminin111}, the most understood one, has $\alpha$1, $\beta$1, and you guessed it $\gamma$1 subunits. 

\begin{figure}[H]
	\centering
	\includegraphics[width=0.7\linewidth]{bl_111}
	\caption{laminin-111, the most understood of laminins, used as example.}
	\label{fig:bl111}
\end{figure}


\paragraph{The types of laminin}

The N-terminus is responsible for interactions with other extracellular matrix proteins, making it important for the assembly and stability of basement membranes. The C-terminus is responsible for interactions with cell surface receptors, making it crucial for adhesion vs. migration, survival vs. apoptosis, signaling, differentiation, and gene expression. This also shows that while the C-terminus is essential, some types laminin don't require a N-terminus.

Here are a bunch of laminin and where they are most commonly found:
\begin{itemize}
	\item 111 mostly in the embryo, rare in adults
	\item 511 and 521 are the most common isoforms in adults
	\item 211 and 221 present in skeletal and cardiac muscles.
	\item 411 and 421 endothelial cells of blood vessels.
	\item 332 is specific for the basal lamina under the epithelial cells (mainly skin).
\end{itemize}

\begin{figure}[H]
	\centering
	\includegraphics[width=0.6\linewidth]{bl_type}
	\caption{Different groups of laminin and in which types of basal lamina they will mostly be found (laminin looking up is its lamina type).}
	\label{fig:bltype}
\end{figure}




\end{document}