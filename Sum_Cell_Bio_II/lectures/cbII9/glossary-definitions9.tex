\newglossaryentry{BP230}{
	name={BP230},
	description={A hemidesmosomal protein that connects plectin and keratin filaments to integrins in epithelial cells.}
}

\newglossaryentry{Ca2}{
	name={Ca2+},
	description={Calcium ions essential for cadherin-mediated adhesion by stabilizing cadherin domains and preventing molecular flexibility.}
}

\newglossaryentry{Collagenfibrils}{
	name={Collagen fibrils},
	description={Bundles of collagen molecules aligned in a staggered fashion, forming the structural framework of connective tissues.}
}

\newglossaryentry{EGFepidermalgrowthfactor}{
	name={EGF},
	description={A cell biology concept related to EGF (epidermal growth factor), requiring further specification.}
}

\newglossaryentry{integrin}{
	name={Integrin},
	description={A family of transmembrane receptors that mediate cell adhesion to the extracellular matrix and other cells. They link the cytoskeleton to the ECM and transmit bidirectional signals to regulate cell shape, motility, and survival.}
}

\newglossaryentry{EMT}{
	name={EMT},
	description={Epithelial-to-mesenchymal transition, a process where epithelial cells lose adhesion and gain migratory, invasive properties.}
}

\newglossaryentry{FN1}{
	name={FN1},
	description={One of the repeating domains in fibronectin responsible for collagen and fibrin binding.}
}

\newglossaryentry{FN2}{
	name={FN2},
	description={A fibronectin domain that contributes to binding interactions within the ECM.}
}

\newglossaryentry{FN3}{
	name={FN3},
	description={A fibronectin domain that contains the RGD motif for integrin binding, crucial for cell adhesion.}
}

\newglossaryentry{GPCR}{
	name={GPCR},
	description={G-protein-coupled receptor; activates intracellular signaling cascades including those that lead to integrin activation.}
}

\newglossaryentry{GTPase}{
	name={GTPase},
	description={A family of enzymes that hydrolyze GTP and regulate many signaling pathways, including cell adhesion and migration.}
}

\newglossaryentry{GTPaseRac}{
	name={GTPase Rac},
	description={A small GTPase that promotes actin polymerization and junction expansion during early cell-cell adhesion.}
}

\newglossaryentry{GTPaseRho}{
	name={GTPase Rho},
	description={A GTPase that promotes actomyosin contractility and maturation of adherens junctions into linear actin bundles.}
}

\newglossaryentry{Gapjunctions}{
	name={Gap junctions},
	description={Specialized intercellular connections composed of connexons that allow direct cytoplasmic exchange of small molecules and ions between neighboring cells.}
}

\newglossaryentry{Golgiapparatus}{
	name={Golgi apparatus},
	description={A cell organelle where glycosylation and processing of ECM proteins like procollagen occur before secretion.}
}

\newglossaryentry{hydroxylysyl}{
	name={Hydroxylysyl (Hyl)},
	description={A hydroxylated derivative of lysine found in collagen. It plays a critical role in forming covalent cross-links between collagen molecules, enhancing tensile strength of the extracellular matrix.}
}

\newglossaryentry{IGFBPInsulingrowthbindingfactor}{
	name={IGFBP},
	description={Insulin growth binding factor, a protein domain found in ECM components that binds insulin-like growth factors, modulating their availability and activity.}
}

\newglossaryentry{Integrin}{
	name={Integrin},
	description={A family of transmembrane receptors that link ECM proteins to the cytoskeleton and mediate bidirectional signaling.}
}

\newglossaryentry{Laminin}{
	name={Laminin},
	description={A cell biology concept related to Laminin, requiring further specification.}
}

\newglossaryentry{Laminin111}{
	name={Laminin-111},
	description={A trimeric laminin isoform found mainly in embryonic tissues, composed of α1, β1, and γ1 chains, crucial for basement membrane structure.}
}

\newglossaryentry{MET}{
	name={MET},
	description={Mesenchymal-to-epithelial transition, where mesenchymal cells adopt epithelial characteristics and form structured tissues.}
}

\newglossaryentry{Phosphoinositide}{
	name={Phosphoinositide},
	description={A lipid component of the membrane that binds Talin and other proteins to regulate integrin activation.}
}

\newglossaryentry{RIAM}{
	name={RIAM},
	description={An adaptor protein that helps recruit Talin to the plasma membrane during integrin activation.}
}

\newglossaryentry{Rap1}{
	name={Rap1},
	description={A small GTPase involved in inside-out signaling to activate integrins during processes like platelet adhesion.}
}

\newglossaryentry{Slug}{
	name={Slug},
	description={Slug is a transcription factor that regulates EMT by repressing epithelial genes and promoting mesenchymal traits.}
}

\newglossaryentry{Snail}{
	name={Snail},
	description={Snail is a transcription factor that regulates EMT by repressing epithelial genes and promoting mesenchymal traits.}
}

\newglossaryentry{Talin}{
	name={Talin},
	description={A cytoskeletal adaptor protein that connects integrins to actin and unfolds under tension to recruit vinculin.}
}

\newglossaryentry{Twist}{
	name={Twist},
	description={Twist is a transcription factor that regulates EMT by repressing epithelial genes and promoting mesenchymal traits.}
}

\newglossaryentry{VitaminC}{
	name={Vitamin-C},
	description={A cofactor required for the hydroxylation of proline and lysine during collagen synthesis. Its deficiency impairs collagen stability.}
}

\newglossaryentry{Zeb}{
	name={Zeb},
	description={Zeb is a transcription factor that regulates EMT by repressing epithelial genes and promoting mesenchymal traits.}
}

\newglossaryentry{actin}{
	name={actin},
	description={A cytoskeletal protein that forms filaments involved in maintaining cell shape and enabling junction formation and contractility.}
}

\newglossaryentry{actinlinkedjunctions}{
	name={actin-linked junctions},
	description={Also known as actin filament attachment sites, these junctions connect actin filaments to cadherins or integrins, supporting adhesion.}
}

\newglossaryentry{adherensjunctions}{
	name={adherens junctions},
	description={Junctions that connect actin filaments between cells via cadherins and adaptor proteins like β-catenin and α-catenin.}
}

\newglossaryentry{adhesionbelt}{
	name={adhesion belt},
	description={A continuous band of adherens junctions linked to actin filaments, providing mechanical integrity and shaping epithelial sheets.}
}

\newglossaryentry{aggrecan}{
	name={aggrecan},
	description={A large proteoglycan with multiple GAG chains that binds hyaluronan to form large ECM aggregates, especially in cartilage.}
}

\newglossaryentry{alphacatenin}{
	name={alpha catenin},
	description={An adaptor protein connecting cadherin-bound β-catenin to actin filaments, enabling dynamic junction regulation.}
}

\newglossaryentry{anchoringjunctions}{
	name={anchoring junctions},
	description={Junctions that attach cells to each other or to the ECM, including adherens junctions, desmosomes, hemidesmosomes, and actin-linked adhesions.}
}

\newglossaryentry{apicalposition}{
	name={apical position},
	description={Refers to the topmost region of an epithelial cell, facing the lumen or external environment. Tight junctions typically occupy this position.}
}

\newglossaryentry{basallamina}{
	name={basal lamina},
	description={A specialized form of extracellular matrix separating epithelial and connective tissue. Composed of laminin, type IV collagen, nidogen, and perlecan.}
}

\newglossaryentry{betacatenin}{
	name={beta-catenin},
	description={A cytoplasmic adaptor protein linking cadherins to actin and also involved in Wnt signaling when stabilized.}
}

\newglossaryentry{bullouspemphigold}{
	name={bullous pemphigold},
	description={An autoimmune skin disease where antibodies target hemidesmosomal proteins, causing blistering.}
}

\newglossaryentry{cadherin}{
	name={cadherin},
	description={A calcium-dependent adhesion molecule mediating homophilic binding between cells. Crucial in adherens junctions and desmosomes.}
}

\newglossaryentry{cadherindomain}{
	name={cadherin domain},
	description={Repeating extracellular domains of cadherins that bind calcium ions and mediate adhesion via N-terminal interactions.}
}

\newglossaryentry{celljunctions}{
	name={cell junctions},
	description={Structures that link adjacent cells or cells to the ECM. Include anchoring, occluding, channel-forming, and signal-relaying junctions.}
}

\newglossaryentry{channelformingjunctions}{
	name={channel-forming junctions},
	description={Junctions that create pores between adjacent cells, allowing the exchange of ions and small molecules. Gap junctions are the primary example in animal cells.}
}

\newglossaryentry{chaperonins}{
	name={chaperonins},
	description={Proteins that assist in the proper folding of procollagen triple helices during synthesis in the endoplasmic reticulum.}
}

\newglossaryentry{chemicalsynapses}{
	name={chemical synapses},
	description={Junctions in the nervous system where neurotransmitters are released to relay signals between neurons or to target cells.}
}

\newglossaryentry{chondroitinsulfate}{
	name={chondroitin sulfate},
	description={A sulfated GAG composed of repeating disaccharides that provides compressive strength to cartilage and connective tissues.}
}

\newglossaryentry{claudin}{
	name={claudin},
	description={A four-pass transmembrane protein forming the backbone of tight junction strands and regulating paracellular permeability.}
}

\newglossaryentry{collagenI}{
	name={collagen I},
	description={The most abundant collagen type in the body, forming fibrils and fibers in skin, bone, tendons, and other tissues.}
}

\newglossaryentry{collagenXVII}{
	name={collagen XVII},
	description={A transmembrane collagen found in hemidesmosomes, anchoring epithelial cells to the basal lamina.}
}

\newglossaryentry{collagenfibers}{
	name={collagen fibers},
	description={Fibrous components of the ECM providing tensile strength. Composed of tightly packed collagen triple helices forming fibrils and bundles.}
}

\newglossaryentry{connectivetissue}{
	name={connective tissue},
	description={Tissue type that supports and anchors other tissues. It contains extracellular matrix rich in fibers like collagen and is separated from epithelium by the basal lamina.}
}

\newglossaryentry{connexin}{
	name={connexin},
	description={A family of proteins that assemble into connexons, enabling the formation of gap junction channels between cells.}
}

\newglossaryentry{connexon}{
	name={connexon},
	description={A hexameric protein assembly forming half of a gap junction channel, made up of six connexin subunits.}
}

\newglossaryentry{cytoskeletalproteins}{
	name={cytoskeletal proteins},
	description={Proteins like actin and intermediate filaments that provide structural integrity and connect to cell junctions for force transmission.}
}

\newglossaryentry{decorin}{
	name={decorin},
	description={A small proteoglycan with a single GAG chain, known to bind collagen fibrils and regulate ECM assembly.}
}

\newglossaryentry{desmocollins}{
	name={desmocollins},
	description={Nonclassical cadherins that function similarly to desmogleins in forming stable desmosomal junctions.}
}

\newglossaryentry{desmogleins}{
	name={desmogleins},
	description={Nonclassical cadherins found in desmosomes that mediate cell-cell adhesion through intermediate filaments.}
}

\newglossaryentry{desmoplakin}{
	name={desmoplakin},
	description={A key desmosomal protein that anchors intermediate filaments to the desmosomal plaque.}
}

\newglossaryentry{desmosomes}{
	name={desmosomes},
	description={Cell-cell anchoring junctions linking intermediate filaments via nonclassical cadherins such as desmogleins and desmocollins.}
}

\newglossaryentry{dystroglycan}{
	name={dystroglycan},
	description={A transmembrane receptor in muscle and epithelial cells that connects the cytoskeleton to the basal lamina via laminin.}
}

\newglossaryentry{ectoderm}{
	name={ectoderm},
	description={The outermost embryonic germ layer that gives rise to the epidermis, nervous system, and related structures.}
}

\newglossaryentry{elastin}{
	name={elastin},
	description={A protein that forms elastic fibers in the ECM, allowing tissues like lungs and skin to stretch and recoil.}
}

\newglossaryentry{epithelialtissue}{
	name={epithelial tissue},
	description={A tissue type that forms tightly connected cell layers, covering body surfaces and lining cavities. It plays a role in protection, secretion, and selective permeability.}
}

\newglossaryentry{extracellularmatrix}{
	name={extracellular matrix},
	description={A complex network of proteins and polysaccharides outside cells. It includes collagen fibers, proteoglycans, and glycoproteins and provides mechanical support.}
}

\newglossaryentry{fibrillarcollagen}{
	name={fibrillar collagen},
	description={Collagen types that assemble into rope-like fibers in connective tissues, including type I, providing tensile strength.}
}

\newglossaryentry{fibroblasts}{
	name={fibroblasts},
	description={Connective tissue cells that synthesize ECM components, especially fibrillar collagens such as collagen I.}
}

\newglossaryentry{fibronectin}{
	name={fibronectin},
	description={A cell biology concept related to fibronectin, requiring further specification.}
}

\newglossaryentry{fibrous}{
	name={fibrous},
	description={Refers to the structural proteins in the ECM, such as collagen, that form fiber-like assemblies to provide tensile strength and mechanical support.}
}

\newglossaryentry{gammacatenin}{
	name={gamma catenin},
	description={Another name for plakoglobin, a desmosomal protein connecting cadherins to intermediate filaments.}
}

\newglossaryentry{glycine}{
	name={glycine},
	description={The smallest amino acid, appearing every third residue in collagen helices to fit into the tightly packed structure.}
}

\newglossaryentry{glycoprotein}{
	name={glycoprotein},
	description={Proteins with covalently attached carbohydrate chains that play structural and signaling roles in the ECM. Many act as scaffold proteins with multiple interaction domains.}
}

\newglossaryentry{glycosaminoglycanGAGs}{
	name={glycosaminoglycan (GAGs)},
	description={Long unbranched polysaccharides composed of repeating disaccharide units, found in proteoglycans and contributing to ECM viscosity and charge.}
}

\newglossaryentry{glycosylation}{
	name={glycosylation},
	description={The enzymatic addition of sugar chains to proteins, critical for forming proteoglycans and glycoproteins in the ECM.}
}

\newglossaryentry{hemidesmosomes}{
	name={hemidesmosomes},
	description={Cell-ECM anchoring junctions connecting intermediate filaments to the basal lamina through integrins and collagen XVII.}
}

\newglossaryentry{heparansulfate}{
	name={heparan sulfate},
	description={A variably sulfated GAG attached to proteoglycans, involved in binding growth factors and regulating signaling.}
}

\newglossaryentry{heparin}{
	name={heparin},
	description={A highly sulfated GAG known for its anticoagulant properties, structurally related to heparan sulfate.}
}

\newglossaryentry{heterophilicbinding}{
	name={heterophilic binding},
	description={Binding interaction between different adhesion molecules on adjacent cells, often seen in signal-relaying junctions like Notch-Delta.}
}

\newglossaryentry{heterotypic}{
	name={heterotypic},
	description={Refers to interactions or gap junctions involving different types of proteins or connexins on adjacent cells.}
}

\newglossaryentry{homophilicbinding}{
	name={homophilic binding},
	description={A form of cell adhesion where identical molecules on adjacent cells bind to each other, common in cadherin-mediated adhesion.}
}

\newglossaryentry{homotypic}{
	name={homotypic},
	description={Describes gap junctions or adhesion interactions where the same type of protein is present on both sides of the junction.}
}

\newglossaryentry{hyaluronan}{
	name={hyaluronan},
	description={A large, non-sulfated GAG that forms a backbone for proteoglycan aggregates, important for tissue hydration and resilience.}
}

\newglossaryentry{hydroxylation}{
	name={hydroxylation},
	description={A post-translational modification essential for stabilizing collagen helices, involving hydroxylation of proline and lysine.}
}

\newglossaryentry{hydroxyproline}{
	name={Hydroxyproline},
	description={A post-translationally modified derivative of proline, formed by hydroxylation in collagen. It stabilizes the collagen triple helix via hydrogen bonding and requires vitamin C for synthesis.}
}


\newglossaryentry{immunologicalsynapses}{
	name={immunological synapses},
	description={Specialized junctions between immune cells that facilitate communication, antigen presentation, and activation. Not covered in this course.}
}

\newglossaryentry{integrinbeta}{
	name={integrin beta},
	description={The β-subunit of integrins that interacts with Talin and Kindlin to mediate inside-out signaling.}
}

\newglossaryentry{intermediatefilamentattachmentsites}{
	name={intermediate filament attachment sites},
	description={Sites where intermediate filaments anchor to junctions like desmosomes and hemidesmosomes, stabilizing tissue architecture.}
}

\newglossaryentry{junctioncomplex}{
	name={junction complex},
	description={A cluster of multiple junction types, typically including tight junctions, adherens junctions, and desmosomes, found in epithelial tissues.}
}

\newglossaryentry{keratansulfate}{
	name={keratan sulfate},
	description={A GAG with a different disaccharide composition, found in cartilage and cornea, contributing to ECM structure.}
}

\newglossaryentry{kindlin}{
	name={kindlin},
	description={A protein that cooperates with Talin to activate integrins by binding to their cytoplasmic tail.}
}

\newglossaryentry{laminin}{
	name={laminin},
	description={A cell biology concept related to laminin, requiring further specification.}
}

\newglossaryentry{lysylhydroxylase}{
	name={lysyl hydroxylase},
	description={An enzyme that hydroxylates lysine residues in collagen, enabling crosslinking and stability of collagen fibrils.}
}

\newglossaryentry{mesenchymal}{
	name={mesenchymal},
	description={Refers to loosely organized, migratory connective tissue cells capable of differentiating into various cell types.}
}

\newglossaryentry{microvilli}{
	name={microvilli},
	description={Finger-like projections on the apical surface of epithelial cells that increase surface area for absorption, especially in the intestine.}
}

\newglossaryentry{myosinII}{
	name={myosin II},
	description={A motor protein that generates contractile force at junctions, influencing cadherin tension and cytoskeletal rearrangement.}
}

\newglossaryentry{nidogen}{
	name={nidogen},
	description={A glycoprotein in the basal lamina that connects laminin networks to type IV collagen, aiding in the structural integrity of the ECM.}
}

\newglossaryentry{nonclassicalcadherins}{
	name={nonclassical cadherins},
	description={Variants of cadherins with diverse structures and roles, such as desmogleins and desmocollins, often involved in stronger cell adhesion.}
}

\newglossaryentry{occludin}{
	name={occludin},
	description={A transmembrane protein that supports tight junction integrity, working alongside claudins to create a selective barrier.}
}

\newglossaryentry{occludingjunctions}{
	name={occluding junctions},
	description={Junctions that seal the space between epithelial cells to regulate permeability. Tight junctions are the main type in vertebrates.}
}

\newglossaryentry{p120catenin}{
	name={p120-catenin},
	description={A component of the adherens junction adaptor complex that stabilizes cadherins and regulates their endocytosis.}
}

\newglossaryentry{perlecan}{
	name={perlecan},
	description={A large proteoglycan in the basal lamina involved in filtration, ECM organization, and signaling.}
}

\newglossaryentry{plakiglobin}{
	name={plakiglobin},
	description={Also known as γ-catenin, a component of desmosomes that links cadherins to intermediate filaments.}
}

\newglossaryentry{plakophilin}{
	name={plakophilin},
	description={A member of the armadillo protein family involved in stabilizing desmosomal junctions and linking them to the cytoskeleton.}
}

\newglossaryentry{plasmodesmata}{
	name={plasmodesmata},
	description={Plant cell structures analogous to gap junctions, forming channels that traverse cell walls. Not covered in this course.}
}

\newglossaryentry{plateletsthrombocytes}{
	name={platelets},
	description={A.k.a. thrombocytes, Small blood cells involved in clotting, where integrin activation plays a role in adhesion and aggregation.}
}

\newglossaryentry{thrombin}{
	name={Thrombin},
	description={A serine protease involved in blood coagulation. It converts fibrinogen to fibrin, leading to clot formation, and also activates platelets and integrin signaling via protease-activated receptors.}
}

\newglossaryentry{plectin}{
	name={plectin},
	description={An adaptor protein linking intermediate filaments to hemidesmosomal components like integrins and BP230.}
}

\newglossaryentry{procollagen}{
	name={procollagen},
	description={A soluble precursor of collagen with propeptides at both ends that prevent premature fibril formation inside the cell.}
}

\newglossaryentry{procollagenNproteinase}{
	name={procollagen N-proteinase},
	description={An enzyme that cleaves the N-terminal propeptides of procollagen, allowing fibril formation in the extracellular space.}
}

\newglossaryentry{proline}{
	name={proline},
	description={A cyclic amino acid that contributes to the rigidity of collagen helices. Often hydroxylated in collagen to hydroxyproline.}
}

\newglossaryentry{prolylhydroxylase}{
	name={prolyl hydroxylase},
	description={An enzyme that hydroxylates proline residues, stabilizing the collagen helix through hydrogen bonding.}
}

\newglossaryentry{propeptides}{
	name={propeptides},
	description={Terminal extensions on procollagen molecules that prevent premature fibril assembly inside the cell and are cleaved after secretion.}
}

\newglossaryentry{proteoglycan}{
	name={proteoglycan},
	description={A protein core heavily glycosylated with glycosaminoglycan (GAG) chains, contributing to ECM hydration and compression resistance.}
}

\newglossaryentry{roddomain}{
	name={rod domain},
	description={A segment of Talin's structure that blocks its binding sites in the inactive state and unfolds under tension.}
}

\newglossaryentry{scaffoldproteins}{
	name={scaffold proteins},
	description={Large, multidomain glycoproteins that organize other ECM components and cell-surface receptors into structured assemblies.}
}

\newglossaryentry{septatejunctions}{
	name={septate junctions},
	description={Occluding junctions found in invertebrates, functionally similar to tight junctions. Not covered in this course.}
}

\newglossaryentry{signalrelayingjunctions}{
	name={signal-relaying junctions},
	description={Cell junctions specialized for transmitting signals, such as neural synapses and receptor-ligand interactions like Notch-Delta.}
}

\newglossaryentry{tenacin}{
	name={tenacin},
	description={An ECM glycoprotein involved in modulating cell adhesion and migration during development and repair.}
}

\newglossaryentry{thrombospondin}{
	name={thrombospondin},
	description={A multifunctional ECM protein involved in cell-to-matrix communication, wound healing, and angiogenesis.}
}

\newglossaryentry{tightjunctions}{
	name={tight junctions},
	description={Junctions that form impermeable seals using claudins and occludins, helping compartmentalize tissues and maintain cell polarity.}
}

\newglossaryentry{transmembraneligandreceptorcellcellsignalingcontacts}{
	name={transmembrane ligand-receptor cell-cell signaling contacts},
	description={Junctions where ligand-bound receptors such as Notch and Delta span adjacent membranes to transmit signals directly.}
}

\newglossaryentry{typeIVcollagen}{
	name={type IV collagen},
	description={A network-forming collagen found primarily in the basal lamina, forming flexible, sheet-like structures.}
}

\newglossaryentry{vimentin}{
	name={vimentin},
	description={An intermediate filament protein typically expressed in mesenchymal cells and upregulated during EMT.}
}

\newglossaryentry{vinculin}{
	name={vinculin},
	description={A cytoskeletal protein that binds α-catenin under tension and promotes actin recruitment to strengthen adherens junctions.}
}