\documentclass[../main.tex]{subfiles}

\usepackage{nopageno} %Seitenzahlen auf richtiger Seite 

\usepackage[left=2cm, right=2cm, top=2cm, includehead, includefoot, headheight=17pt]{geometry}

\usepackage[utf8x]{inputenc}
\usepackage[english]{babel}
\usepackage{amsmath,amssymb,amsthm}
\usepackage{framed}
\usepackage{wasysym}
\usepackage[T1]{fontenc} %Silbentrennung 
\usepackage{color} %Farbe
\usepackage{graphicx}
\usepackage{float}%Grafik am gleichen Ort plazieren
%pdf. png. einfach eingliedern
\usepackage{subfigure} %Grafiken nebeneinander
\usepackage{pdfpages}
\usepackage{ulem} 	%\uuline{urgent}    % doppelt unterstreichen
%\uwave{boat}      % unterschlängeln
%\sout{wrong}       % durchstreichen
%\xout{removed}     % ausstreichen mit //////.

\usepackage{tikz}
\usetikzlibrary{trees}
\usetikzlibrary{plotmarks}
\usetikzlibrary{angles,quotes,babel}
\usetikzlibrary{shadings}
\usetikzlibrary{patterns}
\usetikzlibrary{matrix}
\usetikzlibrary{arrows}
\usetikzlibrary{calc}

\usepackage{pgfplots}
\usepackage{pgf-pie}
\pgfplotsset{compat=1.10}
\usepgfplotslibrary{statistics}
\usepgfplotslibrary{fillbetween}

\usepackage{tkz-euclide}
\usepackage{enumerate}
\usepackage{stmaryrd}
\usepackage{tabularx}
\usepackage{wrapfig}
\usepackage{epsdice}
\usepackage{multirow}
\usepackage{rotating}
\usepackage{pdflscape}
\usepackage{fancyhdr}

\pagestyle{fancy} %eigener Seitenstil
\fancyhf{} %alle Kopf- und Fußzeilenfelder bereinigen
\fancyhead[L]{} %Kopfzeile links
\fancyhead[C]{} %zentrierte Kopfzeile
\fancyhead[R]{} %Kopfzeile rechts
\renewcommand{\headrulewidth}{0.4pt} %obere Trennlinie
\fancyfoot[C]{\thepage} %Seitennummer
\renewcommand{\footrulewidth}{0.4pt} %untere Trennlinie

% Number spaces 
\newcommand{\CC}{\ensuremath{\mathbb{C}}}
\newcommand{\RR}{\ensuremath{\mathbb{R}}}
\newcommand{\QQ}{\ensuremath{\mathbb{Q}}}
\newcommand{\ZZ}{\ensuremath{\mathbb{Z}}}
\newcommand{\NN}{\ensuremath{\mathbb{N}}}
\newcommand{\LL}{\ensuremath{\mathbb{L}}}
\newcommand{\DD}{\ensuremath{\mathbb{D}}}
\newcommand{\WW}{\ensuremath{\mathbb{W}}}

%draw chemestry molecules 
\usepackage{chemfig} % https://mirror.ox.ac.uk/sites/ctan.org/macros/generic/chemfig/

\newcommand\vv[1]{%
	\begin{tikzpicture}[baseline=(arg.base)]
		\node[inner xsep=0pt] (arg) {$#1$};
		\draw[line cap=round,line width=0.45,->,shorten >= 0.2pt, shorten <= 0.7pt] (arg.north west) -- (arg.north east);
	\end{tikzpicture}%
} %command will render \vv{x} with an arrow aboth 

\renewcommand{\labelenumi}{\roman{enumi})}

\DeclareMathOperator{\ggT}{ggT}
\DeclareMathOperator{\sign}{sign}

%sections
\theoremstyle{plain}
\newtheorem{Thm}{Theorem}[section]
\newtheorem{Def}[Thm]{Definition}
\newtheorem{Prop}[Thm]{Proposition}

\theoremstyle{definition}
\newtheorem{lemma}[Thm]{Lemma}
\newtheorem{corollary}[Thm]{Corollary}
\newtheorem{claim}[Thm]{Claim}
\newtheorem{Proof}[Thm]{Proof}
\newtheorem{Ex}[Thm]{Example}

\newtheorem{Exercise}{ex}[section] %follow proper enum
\newtheorem{ex}[Exercise]{Exercise}
\newtheorem{Solution}{sol}[section]
\newtheorem{sol}[Solution]{Solution}

\theoremstyle{remark}
\newtheorem{remark}[Thm]{Remark} % follows thm enum

\newtheorem{comment}{Comment}[section] %follow comment enum
\newtheorem{notation}[comment]{Notation}
\newtheorem{reasoning}[comment]{Reasoning}
\newtheorem{Intpr}[comment]{Interpretation}

%some premmade with title (uterwise use \textbf{Title} ...)
\newenvironment{ThmWithTitle}[1]{%
	\begin{Thm}[\textbf{#1}]}{\end{Thm}}
\newenvironment{PropWithTitle}[1]{%
	\begin{Prop}[\textbf{#1}]}{\end{Prop}}
\newenvironment{ExWithTitle}[1]{%
	\begin{Ex}[\textbf{#1}]}{\end{Ex}}
\newenvironment{DefWithTitle}[1]{%
	\begin{Def}[\textbf{#1}]}{\end{Def}}
\newenvironment{RemarkWithTitel}[1]{%
	\begin{remark}[\textbf{#1}]}{\end{remark}}

%format of paragraph 
\renewcommand\paragraph{\@startsection{paragraph}{4}{\z@}%
	{-2.5ex\@plus -1ex \@minus -.25ex}%
	{1.25ex \@plus .25ex}%
	{\normalfont\normalsize\bfseries}}
\makeatother
\setcounter{secnumdepth}{5} % how many sectioning levels to assign numbers to
\setcounter{tocdepth}{4}    % how many sectioning levels to show in ToC

\newcounter{row} 
\renewcommand\therow{\alph{row}} %hier a,b,c etc. def und mit therow abrufbar

\newenvironment{aufz}
{\setcounter{row}{0}%
	\par\noindent\tabularx{\linewidth}[t]
	{\cdot{20}{>{\stepcounter{row}\makebox[1.5em][l]{\therow)\hfill}}X}} %bis max 20 Elemente nebeinander
}
{\endtabularx}


%biblio
\usepackage[]{biblatex}
\addbibresource{referenzenma.bib} 

%glossary
\usepackage{glossaries}
\usepackage{import}

\newglossaryentry{AdenylateCyclase}{
	name={Adenylate Cyclase},
	description={Also called adenylyl cyclase, this membrane-bound enzyme converts ATP to cAMP in response to stimulation by G-proteins. It is a key player in many GPCR-mediated pathways.}
}



\newglossaryentry{Allornothingsignal}{
	name={All or nothing signal},
	description={A type of cellular response that occurs only after a threshold level of signal is reached, resulting in a binary, digital-like outcome.}
}


\newglossaryentry{AlphaHelix}{
	name={Alpha Helix (AH)},
	description={A common structural motif in proteins, including in G-proteins and GPCRs, where it can play a role in conformational change upon activation.}
}



\newglossaryentry{AlphaSubunit}{
	name={Alpha Subunit},
	description={The component of a heterotrimeric G-protein that binds GDP/GTP and dissociates upon activation to regulate downstream effectors such as adenylate cyclase or phospholipase C.}
}



\newglossaryentry{Arrestins}{
	name={Arrestins},
	description={Proteins that bind phosphorylated GPCRs, blocking further G-protein activation and targeting receptors for internalization or alternate signaling.}
}



\newglossaryentry{AutocrineSignaling}{
	name={Autocrine Signaling},
	description={A form of signaling in which a cell secretes signaling molecules that bind to receptors on its own surface, allowing it to regulate itself.}
}



\newglossaryentry{Autophosphorylation}{
	name={Autophosphorylation},
	description={A process in which a kinase adds a phosphate group to itself, often leading to sustained activation independent of the original signal, as seen in CaM-Kinase II.}
}



\newglossaryentry{BetaComplex}{
	name={Beta Complex},
	description={Part of the G-protein beta-gamma dimer, it remains membrane-associated and contributes to the regulation of ion channels and other signaling proteins.}
}



\newglossaryentry{Calcium}{
	name={Ca\textsuperscript{2+}},
	description={A ubiquitous intracellular second messenger that regulates a wide range of cellular processes including muscle contraction, secretion, metabolism, and gene expression. Its release is often triggered by IP\textsubscript{3} in response to upstream signaling events.}
}


\newglossaryentry{Calmodulin}{
	name={Calmodulin},
	description={A small calcium-binding protein that undergoes conformational change upon Ca\textsuperscript{2+} binding, enabling it to activate target enzymes such as CaM-Kinase II.}
}



\newglossaryentry{CaMKII}{
	name={CaM-Kinase II},
	description={Short for calcium/calmodulin-dependent protein kinase II, an important serine/threonine kinase that decodes calcium oscillations via autophosphorylation and regulates memory, gene expression, and metabolism.}
}



\newglossaryentry{cAMP}{
	name={cAMP},
	description={Short for cyclic adenosine monophosphate, a second messenger synthesized by adenylate cyclase that activates downstream targets like PKA and regulates cellular responses.}
}



\newglossaryentry{cAMPPhosphodiesterase}{
	name={cAMP Phosphodiesterase},
	description={An enzyme that degrades cAMP into AMP, thereby terminating the cAMP signaling pathway. It is a key modulator of signal duration.}
}



\newglossaryentry{CellSignaling}{
	name={Cell Signaling},
	description={A fundamental process by which cells detect, interpret, and respond to external or internal cues through molecular signals. It involves extracellular signaling molecules binding to specific receptors, triggering intracellular signaling cascades that regulate cellular functions such as gene expression, metabolism, division, or apoptosis. Cell signaling enables coordination in multicellular organisms and is essential for development, immune response, and homeostasis.}
}



\newglossaryentry{CellSurfaceReceptor}{
	name={Cell-Surface Receptor},
	description={A transmembrane protein located on the cell membrane that binds extracellular signaling molecules (ligands), such as hormones or neurotransmitters. Upon ligand binding, it initiates an intracellular signaling cascade without the ligand entering the cell. Major classes include ion-channel-coupled receptors, G-protein-coupled receptors, and enzyme-coupled receptors.}
}



\newglossaryentry{cGMP}{
	name={cGMP},
	description={Short for cyclic guanosine monophosphate, a second messenger similar to cAMP that is produced by guanylate cyclase and regulates processes like phototransduction and vasodilation.}
}



\newglossaryentry{cGMPPhosphodiesterase}{
	name={cGMP Phosphodiesterase},
	description={An enzyme activated in visual transduction that hydrolyzes cGMP to GMP, leading to the closing of ion channels in photoreceptor cells.}
}



\newglossaryentry{ConstitutivelyActive}{
	name={Constitutively Active},
	description={Describes a receptor or signaling protein that is active without the need for ligand binding, often due to mutations or abnormal expression.}
}


\newglossaryentry{ContactDependentSignaling}{
	name={Contact Dependent Signaling},
	description={A signaling mechanism that requires direct membrane-to-membrane contact between cells, typically involving membrane-bound ligands and receptors.}
}


\newglossaryentry{CREB}{
	name={CREB},
	description={Short for cAMP response element-binding protein, a transcription factor phosphorylated by PKA that regulates genes involved in memory, survival, and metabolism.}
}



\newglossaryentry{DAG}{
	name={DAG},
	description={Short for diacylglycerol, a lipid-derived second messenger produced by PLC that activates protein kinase C and regulates membrane-associated signaling.}
}



\newglossaryentry{DownstreamCascade}{
	name={Downstream Cascade},
	description={A sequence of biochemical events triggered by receptor activation, involving multiple intermediates and amplifying the original signal to produce a cellular response.}
}



\newglossaryentry{Effectorproteins}{
	name={Effector proteins},
	description={Proteins that execute the final cellular response to a signal, such as changes in gene expression, metabolism, or cytoskeletal structure.}
}


\newglossaryentry{EndocrineSignaling,hormonalsignaling}{
	name={Endocrine Signaling},
	description={A.k.a. hormonal signaling, Long-range signaling in which hormones are secreted into the bloodstream and act on distant target cells.}
}


\newglossaryentry{Enzymecoupledreceptors}{
	name={Enzyme coupled receptors},
	description={Transmembrane receptors that have intrinsic enzymatic activity or are associated with enzymes activated by ligand binding.}
}


\newglossaryentry{ExtracellularSignalingMolecule}{
	name={Extracellular Signaling Molecule},
	description={A molecule, such as a hormone or neurotransmitter, that is released from one cell to bind receptors on another and initiate signaling.}
}


\newglossaryentry{G-protein-coupledreceptors}{
	name={G-protein-coupled receptors},
	description={A large family of membrane receptors (GPCRs) that activate intracellular G-proteins upon ligand binding to transmit signals.}
}


\newglossaryentry{GammaComplex}{
	name={Gamma Complex},
	description={Forms a functional dimer with the beta subunit in heterotrimeric G-proteins, anchoring the complex to membranes and participating in downstream signaling.}
}



\newglossaryentry{GAP}{
	name={GAP},
	description={GTPase-activating proteins that enhance the intrinsic GTPase activity of G-proteins, leading to signal termination.}
}


\newglossaryentry{GAPJunctions}{
	name={GAP Junctions},
	description={Specialized intercellular connections that allow direct chemical communication between adjacent cells via diffusion of small molecules and ions.}
}



\newglossaryentry{GEF}{
	name={GEF},
	description={Guanine nucleotide exchange factors, proteins that activate GTP-binding proteins by promoting the exchange of GDP for GTP.}
}


\newglossaryentry{GPCR}{
	name={GPCR},
	description={Short for G-protein-coupled receptor, a family of 7-pass transmembrane receptors that activate G-proteins in response to extracellular ligands. They are among the most abundant and versatile signaling receptors in eukaryotic cells.}
}



\newglossaryentry{GPCRKinases}{
	name={GPCR Kinases (GRKs)},
	description={A family of kinases that phosphorylate activated GPCRs, initiating their desensitization by promoting arrestin binding.}
}



\newglossaryentry{GProteins}{
	name={G-Proteins},
	description={Short for guanine nucleotide-binding proteins, these molecular switches relay signals from receptors (like GPCRs) to intracellular effectors by cycling between GDP-bound (inactive) and GTP-bound (active) states.}
}



\newglossaryentry{Gq}{
	name={Gq},
	description={A subclass of heterotrimeric G-proteins that activates phospholipase C, leading to intracellular calcium release and activation of protein kinase C.}
}



\newglossaryentry{GTPases}{
	name={GTPases},
	description={Enzymes that hydrolyze GTP to GDP and phosphate, acting as molecular switches in signaling pathways.}
}


\newglossaryentry{GTPbinding}{
	name={GTP binding},
	description={A regulatory mechanism by which proteins, especially G-proteins, toggle between active and inactive states depending on GTP or GDP binding.}
}


\newglossaryentry{GuanylateCyclase}{
	name={Guanylate Cyclase},
	description={An enzyme that converts GTP to cGMP upon activation by nitric oxide or natriuretic peptides, initiating cGMP-mediated signaling pathways.}
}


		
		
		
\newglossaryentry{HeterotrimericGProtein}{
	name={Heterotrimeric G Protein},
	description={A type of G-protein composed of three distinct subunits—alpha, beta, and gamma—that relay signals from GPCRs to downstream effectors.}
}



\newglossaryentry{Hyperbolicsignal}{
	name={Hyperbolic signal},
	description={A graded signal response that increases steadily with ligand concentration and eventually plateaus, resembling Michaelis-Menten kinetics.}
}


\newglossaryentry{Inhibitorysignals}{
	name={Inhibitory signals},
	description={Signals that suppress or diminish cellular responses, often balancing excitatory pathways for proper cell regulation.}
}


\newglossaryentry{InsulinReceptorSubstrate(IRS)}{
	name={Insulin Receptor Substrate (IRS)},
	description={A docking protein phosphorylated by the insulin receptor, serving as a scaffold for downstream signaling molecules.}
}



\newglossaryentry{IntracellularReceptor}{
	name={Intracellular Receptor},
	description={A receptor located within the cytoplasm or nucleus that binds small, hydrophobic signaling molecules (e.g., steroid hormones) that cross the plasma membrane. Upon activation, many intracellular receptors function as transcription factors that directly modulate gene expression.}
}



\newglossaryentry{Ion-channel-coupledreceptors}{
	name={Ion-channel-coupled receptors},
	description={Receptors that open or close ion channels in response to ligand binding, converting chemical signals into electrical ones.}
}


\newglossaryentry{Lipidrecruitment}{
	name={Lipid recruitment},
	description={The process of signaling molecules being recruited to specific membrane lipids, such as PIP3, for spatial activation.}
}


\newglossaryentry{Longfeedbackdelay}{
	name={Long feedback delay},
	description={A feedback loop that acts over a longer time scale, potentially leading to oscillations or long-term regulation.}
}


\newglossaryentry{ModularInteractionDomain}{
	name={Modular Interaction Domain},
	description={Protein domains that mediate specific interactions with phosphorylated or lipid-modified partners in signaling complexes.}
}


\newglossaryentry{Molecularswitches}{
	name={Molecular switches},
	description={Molecules, often proteins, that toggle between 'on' and 'off' states to propagate or terminate signals.}
}


\newglossaryentry{NegativeFeedback}{
	name={Negative Feedback},
	description={A regulatory mechanism in which a signaling output inhibits an earlier step, stabilizing the pathway.}
}


\newglossaryentry{Neurotransmitter}{
	name={Neurotransmitter},
	description={A chemical messenger that transmits signals across synapses from one neuron to another.}
}

%Section - Cell Signaling: the World of G-Proteins


\newglossaryentry{Oscillation}{
	name={Oscillation},
	description={In cell signaling, a periodic fluctuation in the concentration or activity of signaling molecules (such as Ca\textsuperscript{2+}) that conveys dynamic information to control gene expression or cellular responses.}
}



\newglossaryentry{ParacrineSignaling}{
	name={Paracrine Signaling},
	description={Short-range signaling where secreted molecules affect nearby target cells without entering the bloodstream.}
}


\newglossaryentry{Phosphatidylserine}{
	name={Phosphatidylserine},
	description={A negatively charged phospholipid found on the inner leaflet of the plasma membrane that helps localize signaling proteins like PKC through electrostatic interactions.}
}



\newglossaryentry{PhospholipaseC}{
	name={Phospholipase C (PLC)},
	description={A membrane-associated enzyme activated by certain G-proteins (like Gq), which hydrolyzes phosphoinositides to generate DAG and IP\textsubscript{3}, initiating calcium signaling and PKC activation.}
}



\newglossaryentry{Phosphorylation}{
	name={Phosphorylation},
	description={The addition of a phosphate group to a protein or other molecule, often regulating activity or interactions.}
}


\newglossaryentry{PhosphotyrosineBinding(PTB)}{
	name={Phosphotyrosine Binding (PTB)},
	description={A domain that binds phosphorylated tyrosines on target proteins, mediating recruitment in signaling pathways.}
}


\newglossaryentry{PI}{
	name={PI},
	description={Short for phosphatidylinositol, a membrane phospholipid that can be phosphorylated to form various signaling lipids like PI(4,5)P\textsubscript{2}, which are substrates for PLC and involved in many signaling pathways.}
}



\newglossaryentry{PKA}{
	name={PKA},
	description={Short for protein kinase A, a serine/threonine kinase activated by cAMP that phosphorylates various substrates to regulate metabolism, gene expression, and other cellular processes.}
}



\newglossaryentry{PKC}{
	name={PKC},
	description={Short for protein kinase C, a family of serine/threonine kinases activated by DAG and calcium that phosphorylate a variety of cellular proteins involved in growth, metabolism, and differentiation.}
}



\newglossaryentry{PLC}{
	name={PLC},
	description={Short for phospholipase C, a key enzyme in Gq-mediated signaling that generates second messengers DAG and IP\textsubscript{3} from membrane phospholipids. See also Phospholipase C.}
}



\newglossaryentry{PleckstrinHomology(PH)}{
	name={Pleckstrin Homology (PH)},
	description={A protein domain that binds phosphoinositides in membranes, targeting proteins to specific locations.}
}


\newglossaryentry{PositiveFeedback}{
	name={Positive Feedback},
	description={A mechanism in which a signaling output enhances an earlier step, amplifying the signal.}
}


\newglossaryentry{ProteinRecruitment}{
	name={Protein Recruitment},
	description={The assembly of signaling complexes at specific membrane sites or proteins through binding domains.}
}


\newglossaryentry{ProteinResponse}{
	name={Protein Response},
	description={The cellular outcome of a signaling event, typically involving activation or repression of specific proteins.}
}


\newglossaryentry{RasDomain}{
	name={Ras Domain},
	description={A conserved GTP-binding domain found in small GTPases like Ras, involved in signal transduction and downstream activation of pathways such as MAPK.}
}



\newglossaryentry{Receptor}{
	name={Receptor},
	description={A protein, usually on the cell surface or in the cytoplasm, that binds a specific signaling molecule and initiates a response.}
}


\newglossaryentry{Rhodopsin}{
	name={Rhodopsin},
	description={A light-sensitive GPCR found in photoreceptor cells of the retina that activates the visual transduction pathway via transducin and cGMP breakdown.}
}



\newglossaryentry{Scaffoldingprotein}{
	name={Scaffolding protein},
	description={A protein that binds multiple signaling components, organizing them into functional complexes to enhance efficiency and specificity.}
}


\newglossaryentry{Shortfeedbackdelay}{
	name={Short feedback delay},
	description={A feedback loop that acts rapidly after signal initiation, often stabilizing or fine-tuning the signal.}
}


\newglossaryentry{SigmoidalSignal}{
	name={Sigmoidal Signal},
	description={A type of cellular response curve characterized by a slow initiation, followed by a steep increase, and then saturation—forming an “S” shape. It often reflects cooperative binding or multi-step signaling cascades, allowing cells to respond sensitively to threshold changes in stimulus concentration.}
}



\newglossaryentry{Signalingcascade}{
	name={Signaling cascade},
	description={A series of biochemical events, often involving sequential activation of enzymes, leading to a cellular response.}
}


\newglossaryentry{SignalIntegration}{
	name={Signal Integration},
	description={The cellular process of combining inputs from multiple signaling pathways to generate a unified response.}
}


\newglossaryentry{SrcHomology(SH)}{
	name={Src Homology (SH)},
	description={A family of protein domains (e.g., SH2, SH3) involved in recognizing phosphorylated tyrosines or proline-rich motifs.}
}


\newglossaryentry{SynapticSignaling}{
	name={Synaptic Signaling},
	description={A specialized form of signaling in neurons where neurotransmitters are released at synapses to stimulate adjacent cells.}
}


\newglossaryentry{Transcriptionalresponse}{
	name={Transcriptional response},
	description={Changes in gene expression triggered by signaling pathways reaching the nucleus.}
}


\newglossaryentry{Transducin}{
	name={Transducin},
	description={A heterotrimeric G-protein specifically involved in visual signaling, activated by rhodopsin to stimulate cGMP phosphodiesterase.}
}

\newglossaryentry{HalfLife}{
	name={Half-Life},
	description={The time required for the concentration of a substance—such as a signaling molecule, mRNA, or protein—to decrease to half of its initial value.}
}


\newglossaryentry{AKT}{
	name={AKT},
	description={A serine/threonine-specific protein kinase also known as Protein Kinase B, involved in promoting cell survival and growth through downstream effects of PI3K signaling.}
}

\newglossaryentry{Cholesterol}{
	name={Cholesterol},
	description={A lipid molecule essential for membrane structure and function; it also serves as a precursor for steroid hormones and plays a role in modulating signaling pathways such as Hedgehog.}
}

\newglossaryentry{CK1}{
	name={CK1},
	description={Short for Casein Kinase 1, a serine/threonine kinase that phosphorylates signaling components in the Wnt and Hedgehog pathways.}
}

\newglossaryentry{Costal2}{
	name={Costal2},
	description={A kinesin-like protein in the Hedgehog pathway that forms a complex with Smoothened and regulates the processing of Cubitus Interruptus.}
}

\newglossaryentry{CubitusInterruptus}{
	name={Cubitus Interruptus (Ci)},
	description={A transcription factor regulated by the Hedgehog pathway in Drosophila; acts as a repressor or activator depending on Hedgehog signal presence.}
}

\newglossaryentry{Delta}{
	name={Delta},
	description={A membrane-bound ligand for the Notch receptor that plays a critical role in lateral inhibition during development.}
}

\newglossaryentry{Disheveled}{
	name={Disheveled},
	description={A cytoplasmic protein activated by Frizzled in the Wnt pathway; it inhibits the degradation complex to stabilize β-catenin.}
}

\newglossaryentry{Dimerization}{
	name={Dimerization},
	description={The process by which two receptor molecules associate, often as a prerequisite for activation, especially in receptor tyrosine kinases (RTKs).}
}

\newglossaryentry{EGFKinase}{
	name={EGF Kinase},
	description={A kinase domain found in the Epidermal Growth Factor Receptor (EGFR), involved in autophosphorylation upon ligand binding and dimerization.}
}

\newglossaryentry{EGFR}{
	name={EGF-R},
	description={Epidermal Growth Factor Receptor, a receptor tyrosine kinase (RTK) that activates downstream pathways like Ras/MAPK upon EGF binding.}
}

\newglossaryentry{Endocytosis}{
	name={Endocytosis},
	description={A cellular process by which extracellular materials are internalized via vesicles; in signaling, it helps regulate receptor availability and pathway duration.}
}

\newglossaryentry{epithelialcells}{
	name={Epithelial cells},
	description={Cells that line surfaces and cavities of organs, involved in barrier function, absorption, and signaling processes like lateral inhibition.}
}

\newglossaryentry{Erk}{
	name={Erk},
	description={Extracellular signal-regulated kinase, part of the MAPK signaling pathway downstream of Ras, involved in cell proliferation and differentiation.}
}

\newglossaryentry{FRET}{
	name={FRET},
	description={Short for Förster Resonance Energy Transfer, a technique used to study molecular interactions based on energy transfer between two fluorescent molecules.}
}

\newglossaryentry{FGFR}{
	name={FGFR},
	description={Fibroblast Growth Factor Receptor, a type of RTK that triggers signaling cascades involved in development and cell differentiation.}
}

\newglossaryentry{frizzled}{
	name={Frizzled},
	description={A family of G-protein-coupled receptors that bind Wnt proteins and initiate the Wnt/β-catenin signaling pathway.}
}

\newglossaryentry{Grb2}{
	name={Grb2},
	description={An adaptor protein that links RTKs like EGFR to Ras activation via interaction with SOS, part of the Ras/MAPK pathway.}
}

\newglossaryentry{Groucho}{
	name={Groucho},
	description={A transcriptional co-repressor that inhibits Wnt target gene expression by binding LEF1 in the absence of β-catenin.}
}

\newglossaryentry{GSK3}{
	name={GSK3},
	description={Short for Glycogen Synthase Kinase 3 (often GSK3), involved in phosphorylating β-catenin to promote its degradation in the Wnt pathway.}
}

\newglossaryentry{iHog}{
	name={iHog},
	description={Short for Interference Hedgehog, a co-receptor in the Hedgehog signaling pathway that facilitates ligand reception and signaling.}
}

\newglossaryentry{IKKalpha}{
	name={IKK alpha},
	description={A kinase that is part of the IκB kinase complex; phosphorylates IκB, promoting its degradation and thereby activating NFκB signaling.}
}

\newglossaryentry{IKKbeta}{
	name={IKK beta},
	description={A key catalytic subunit of the IKK complex, required for the canonical NFκB pathway activation through phosphorylation of IκB.}
}

\newglossaryentry{JAK}{
	name={JAK},
	description={Janus Kinase, a family of non-receptor tyrosine kinases associated with cytokine receptors, activating STAT proteins upon phosphorylation.}
}

\newglossaryentry{kinasecytokinereceptors}{
	name={Kinase cytokine receptors},
	description={Receptors that lack intrinsic kinase activity but associate with tyrosine kinases like JAK to transduce signals from extracellular cytokines.}
}

\newglossaryentry{kinaseinsertregion}{
	name={Kinase insert region},
	description={A flexible loop within the kinase domain of some RTKs involved in substrate specificity or regulation of kinase activity.}
}

\newglossaryentry{LEF1}{
	name={LEF1},
	description={Lymphoid enhancer-binding factor 1, a transcription factor activated by β-catenin in the Wnt signaling pathway.}
}

\newglossaryentry{LateralInhibition}{
	name={Lateral Inhibition},
	description={A process during development where a cell inhibits its neighbors from adopting the same fate, often mediated by Notch-Delta signaling.}
}

\newglossaryentry{LRP}{
	name={LRP},
	description={Low-density lipoprotein receptor-related protein, a Wnt co-receptor that cooperates with Frizzled to activate downstream signaling.}
}

\newglossaryentry{MAP}{
	name={MAP},
	description={Short for Mitogen-Activated Protein; involved in cascades such as Ras-MAPK that control gene expression, cell division, and survival.}
}

\newglossaryentry{mTORC1}{
	name={mTORC1},
	description={Mechanistic target of rapamycin complex 1, regulates protein synthesis, metabolism, and cell growth; activated downstream of AKT.}
}

\newglossaryentry{mTORC2}{
	name={mTORC2},
	description={A signaling complex that phosphorylates AKT and regulates the cytoskeleton and survival pathways.}
}

\newglossaryentry{NEMO}{
	name={NEMO},
	description={NFκB essential modulator, a regulatory subunit of the IKK complex that is crucial for NFκB activation.}
}

\newglossaryentry{Nemo}{
	name={Nemo},
	description={Alternative name for NEMO, required for assembling the IKK complex and regulating NFκB signaling.}
}

\newglossaryentry{NFKB}{
	name={NFkB},
	description={A transcription factor that regulates immune and inflammatory responses, activated upon degradation of its inhibitor IκB.}
}

\newglossaryentry{Notch}{
	name={Notch},
	description={A membrane-bound receptor that, upon binding Delta, undergoes proteolytic cleavage releasing the Notch intracellular domain (NICD) to regulate gene transcription.}
}

\newglossaryentry{Notchtail}{
	name={Notch tail},
	description={Also known as NICD (Notch Intracellular Domain), it translocates to the nucleus and associates with Rbpsuh to influence transcription.}
}

\newglossaryentry{Patched}{
	name={Patched},
	description={A receptor in the Hedgehog pathway that inhibits Smoothened in the absence of Hedgehog ligand.}
}

\newglossaryentry{PDK1}{
	name={PDK1},
	description={Phosphoinositide-dependent kinase-1, activates AKT by phosphorylation following PI3K signaling.}
}

\newglossaryentry{Phosphotyrosin}{
	name={Phosphotyrosine},
	description={A phosphorylated tyrosine residue that serves as a binding site for SH2 domain-containing proteins in signaling pathways.}
}

\newglossaryentry{PI3K}{
	name={PI3K},
	description={Phosphoinositide 3-kinase, an enzyme activated by RTKs that produces PIP3, leading to AKT activation.}
}

\newglossaryentry{protooncogene}{
	name={Proto-oncogene},
	description={A normal gene that can become an oncogene through mutation or overexpression, promoting cell proliferation or survival.}
}

\newglossaryentry{PTEN}{
	name={PTEN},
	description={A phosphatase that antagonizes PI3K signaling by dephosphorylating PIP3 to PIP2, acting as a tumor suppressor.}
}

\newglossaryentry{Ras}{
	name={Ras},
	description={A small GTPase that transmits signals from RTKs to MAPK cascades, promoting proliferation and differentiation.}
}

\newglossaryentry{RasMPK}{
	name={Ras MPK},
	description={Refers to the Ras-MAPK pathway, a cascade where Ras activates RAF, MEK, and ERK, leading to gene regulation and cell proliferation.}
}

\newglossaryentry{Rbpsuh}{
	name={Rbpsuh},
	description={Recombination signal-binding protein for immunoglobulin kappa J region, a transcription factor that partners with NICD in Notch signaling.}
}

\newglossaryentry{RTK}{
	name={RTK},
	description={Short for Receptor Tyrosine Kinase, a class of cell-surface receptors that activate intracellular signaling via tyrosine phosphorylation.}
}

\newglossaryentry{Smad}{
	name={Smad},
	description={Intracellular proteins that transmit signals from TGF-beta receptors to the nucleus to regulate transcription.}
}

\newglossaryentry{Smoothened}{
	name={Smoothened},
	description={A transmembrane protein activated in the Hedgehog pathway upon relief of Patched inhibition, triggering downstream signaling.}
}

\newglossaryentry{STAT}{
	name={STAT},
	description={Signal Transducer and Activator of Transcription; phosphorylated by JAKs and then dimerizes to regulate gene expression.}
}

\newglossaryentry{TGFbeta}{
	name={TGF beta},
	description={Transforming Growth Factor beta, a family of cytokines involved in cell proliferation, differentiation, and immune regulation via Smad signaling.}
}

\newglossaryentry{TNF}{
	name={TNF},
	description={Tumor Necrosis Factor, a cytokine involved in systemic inflammation, apoptosis, and immune system signaling through receptors like TNFR.}
}

\newglossaryentry{typeIreceptor}{
	name={Type I receptor},
	description={A component of the TGF-beta receptor complex that is phosphorylated by type II receptors to propagate Smad signaling.}
}

\newglossaryentry{typeIIreceptor}{
	name={Type II receptor},
	description={A receptor that binds TGF-beta ligand and phosphorylates the associated type I receptor to initiate downstream signaling.}
}

\newglossaryentry{Wnt}{
	name={Wnt},
	description={A family of secreted glycoproteins that activate Frizzled receptors and regulate β-catenin stabilization, crucial for development and cell fate.}
}

\newglossaryentry{androgen}{
	name={Androgen},
	description={A group of steroid hormones like testosterone that regulate male traits and reproductive activity.}
}

\newglossaryentry{androgenreceptor}{
	name={Androgen receptor (AR)},
	description={A type of intracellular receptor that binds androgens, then translocates to the nucleus to regulate target gene transcription.}
}

\newglossaryentry{Transphosphorylation}{
	name={Transphosphorylation},
	description={A process in which one kinase phosphorylates another kinase, often occurring during receptor activation, such as with receptor tyrosine kinases (RTKs) where two adjacent receptors phosphorylate each other upon dimerization.}
}

\newglossaryentry{TumorSuppressor}{
	name={Tumor Suppressor},
	description={A gene or protein that prevents uncontrolled cell growth by regulating the cell cycle, promoting apoptosis, or repairing DNA. Loss-of-function mutations in tumor suppressors can contribute to cancer.}
}

\newglossaryentry{ProteolyticCleavage}{
	name={Proteolytic Cleavage},
	description={A biochemical process where specific peptide bonds in a protein are broken by proteases, activating or deactivating signaling molecules or receptors, such as in Notch or Hedgehog signaling pathways.}
}

\newglossaryentry{BetaCatenin}{
	name={Beta Catenin},
	description={A multifunctional protein involved in the Wnt signaling pathway and in cell adhesion. In Wnt signaling, its stabilization leads to nuclear translocation and activation of Wnt target genes.}
}

\newglossaryentry{Axin}{
	name={Axin},
	description={A scaffold protein that forms part of the destruction complex in Wnt signaling. It promotes degradation of beta-catenin in the absence of Wnt signals.}
}

\newglossaryentry{Hedgehog}{
	name={Hedgehog},
	description={A secreted signaling molecule that regulates cell growth and patterning during development. Binding of Hedgehog to the Patched receptor activates Smoothened, initiating downstream signaling through proteins like Cubitus interruptus (Ci).}
}
\newglossaryentry{MEK}{
	name={MEK},
	description={Mitogen-activated protein kinase kinase (MAPKK), a dual-specificity kinase that phosphorylates and activates ERK in the MAPK signaling cascade. MEK acts downstream of Raf and plays a key role in transmitting growth signals from the cell membrane to the nucleus.}
}

\newglossaryentry{Raf}{
	name={Raf},
	description={A serine/threonine-specific protein kinase (MAPKKK) that is activated by Ras in the MAPK pathway. Raf phosphorylates and activates MEK, initiating a kinase cascade involved in cell division and differentiation.}
}



\makeglossaries

\begin{document}
	

\section{Cellular Signaling}

\subsection{Principles of Cell Signaling}

\subsubsection{The Basic Vocabulary of Cell Signaling}

\begin{figure}[h]
	\centering
	\includegraphics[width=0.45\textwidth]{Basic_overview}
	\caption{Basic Vocabulary of Cell Signaling}
\end{figure}

There are some key terms in cell signaling. Here's a run down of how they connect in cell signaling:
\begin{enumerate}
	\item An \textbf{extracellular signaling molecule} binds to a \textbf{signal receiving protein or receptor}.
	\item That receptor is generally a transmembrane protein but can also be intracellular. This receptor is activated through the binding (generally some sort of conformational change).
	\item This causes a \textbf{signaling cascade} through a chain of \textbf{intracellular signaling proteins} activating each other.
	\item The final protein in that cascade will then change the activity of the effector protein, which launches the cellular response.
	\item These \textbf{effector proteins} can be metabolic enzymes, transcription regulators, or cytoskeletal proteins.
\end{enumerate}

\subsubsection{Cell-surface vs. Intracellular Receptors}

There are two main differences between Cell-Surface and Intracellular receptors:

First, the location of the receptors (surprised you with that am I right):
\begin{itemize}
	\item Cell-Surface: generally transmembrane protein, where the signaling molecule binds extracellularly.
	\item Intracellular: The receptor protein will be close or even inside the nucleus.
\end{itemize}

Then, accordingly the signaling molecule will also be different, in the case of:
\begin{itemize}
	\item Cell-Surface, it is generally a hydrophilic signaling molecule. This means the molecule can't enter the cell, so we need the receptor to have some extracellular component.
	\item Intracellular, it is a small hydrophilic signaling molecule, which can transfer the cell membrane. This is necessary as it needs to reach the receptor in the nucleus. They are carrier through the blood by carrier proteins (hydrophilic).
\end{itemize}


\subsubsection{The Four Subtypes of Cell Signaling}

\begin{figure}[H]
	\centering
	\includegraphics[width=0.45\textwidth]{4_horsemen}
	\caption{The Four types of Signaling}
\end{figure}

\textbf{Contact Dependent Signaling}:
\begin{itemize}
	\item Area of signal: Cells are in contac6t.
	\item Form of communication: Proteins which are attached to the cells interact. One protein serves as the signal and the other as receptor.
	\item Variation: cell with extracellular matrix (e.g., collagen), for more details see section on ECM.
\end{itemize}


\textbf{Paracrine Signaling}:
\begin{itemize}
	\item Area of signal: Cells are not in contact. This is usually a local signal, just a few cells away.
	\item Form of communication: One protein secreted by a cell, is the signal or ligand and attaches to the receptor of a different cell.
	\item Variation: If a cell receives its own signal it is called autocrine signaling.
\end{itemize}


\textbf{Synaptic Signaling}:
\begin{itemize}
	\item Area of signal: Cells are not in contact. Small distance between releaser of ligand and receptor, called the synapse. Very local signal.
	\item Form of communication: Secretion of a ligand or neurotransmitter. Released by one cell and recepted by another.
	\item Variation: If a cell receives its own signal it is also called autocrine signaling.
\end{itemize}


\textbf{Endocrine Signaling a.k.a. Hormonal Signaling}:
\begin{itemize}
	\item Area of signal: Cells are not in contact. Can be long distance and have effects from anywhere to anywhere
	\item Form of communication: A hormone is produced by cell A and then released into the bloodstream, where it can then leave at some point and serve as a signal to a receptor protein.
	\item Variation: If a cell receives its own signal it is called autocrine signaling.
\end{itemize}



\subsubsection{The diversity in Signals}

The same signal can cause a multitude of signals. This section will have a look of the consequences and opportunities of that. 

\textbf{Multiple signaling molecules}, better the combiantion can cause very different signals: Depending on the combination of signals received a cell can kill, proliferate, or differentiate itself. Further the same signaling ligand or protein can have very different consequences dependign on the receptor or cell it attaches to. See for example acetylcholine: 

\begin{figure}[H]
	\centering
	\includegraphics[width=0.45\textwidth]{Acetylcholine}
	\caption{Examples of Acetylcholine having vastly different responses to its signal.}
\end{figure}

\textbf{Speed of the response}: Depending on the response path, the response by the cell can be fast or slow. If the response alters a protein it will take seconds to minutes, while if the gene has to be transcribed it takes minutes to hours. These two types are called \textbf{protein response} or \textbf{transcriptional response}. Some receptors also cause both the fast and slow response path. 

\begin{figure}[H]
	\centering
	\includegraphics[width=0.45\textwidth]{response_speed}
	\caption{An overview of protein (fast) vs. transcriptional (slow) response.}
\end{figure}

Examples or a fast response: change in movement, secretion, or metabolism, caused by e.g., phosphorylation. Concretely the recruitment of GLUT transporters from recycling endosomes, has to occur very rapidly once insulin docks onto the receptor.

\begin{figure}[H]
	\centering
	\includegraphics[width=0.45\textwidth]{GLUT_fast}
	\caption{An example of a fast protein response with the GLUT transporter.}
\end{figure}

\subsubsection{Classes of Cell-Surface Receptors}

There are three main classes of cell-surface receptors, which we will all be diving into later on:
\begin{enumerate}
	\item Ion-Channel-Coupled Receptors a.k.a. transmitter-gated ion channels;
	\item G-Protein-Couple Receptors;
	\item Enzyme-Couple Receptors;
\end{enumerate}

\begin{figure}[h]
	\centering
	\subfigure[Ion class]{
		\includegraphics[width=0.45\textwidth]{ion_class}
	}
	\hfill
	% Second subfigure
	\subfigure[G class]{
		\includegraphics[width=0.5\textwidth]{g_class}
	}
		\hfill
	% Second subfigure
	\subfigure[Enzyme class]{
		\includegraphics[width=0.5\textwidth]{enzyme_class}
	}
	\caption{The three classes of cell-surface receptors}
\end{figure}
Note on the enzyme-coupled one: There are two options here: one where the enzyme is part of the receptor and another where the enzyme is recruited. Ligands activate the receptors by promoting their dimerization though, regardless if the enzyme is directly attached or not.

\subsubsection{Regulation of Intracellular Signaling Proteins}
\label{sec:GTP}

There are two main molecular switches for intracellular signaling proteins:
\begin{enumerate}
	\item \textbf{Phosphorylation} 
	\begin{itemize}
		\item based on a phosphate group being attached to the protein (attached means active).
		\item phosphorylation or dephosphorylation often leads to change in formation and to activation.
		\item Addition by kinases.
		\item Removal by phosphatases.
		\item This group can be added to three amino acids: Tyrosine, Threonine, or Serine. This is because they have an alcohol which works for substitution.
	\end{itemize}
	\item \textbf{GTP binding}
	\begin{itemize}
		\item We have GTPases, whose activity is controlled through GTP/GDP This is also a type of G-Protein, but a different class of monomeric "small" GTPases.
		\item A phosphate is removed from GTP to make GDP, this deactivates the molecule. GDP stays bound.
		\item With an incoming signal, this GDP can be exchanged for a GTP.
		\item GEF activate GTPases, by exchanging GDP for GTP.
		\item GAPs inactivate GTPases by hydrolyzing GTP and yes GAP stands for GTPase-activating protein, as it activates the inactivation.
		\item going from GTP to GDP is more favorable due to the ratio GTP:GDP.
	\end{itemize}
\end{enumerate}
	
\begin{figure}[h]
	\centering
	\subfigure[Phosphorylation as molecular switch]{
		\includegraphics[width=0.45\textwidth]{intra_phospy}
	}
	\hfill
	% Second subfigure
	\subfigure[GTP as molecular switch]{
		\includegraphics[width=0.5\textwidth]{intra_GTP}
	}
	\caption{Two types of molecular switches in intracellular signaling}
\end{figure}
 


\subsubsection{Inhibitory Signals as Activators}

Signal transduction isn't always a positive signal, sometimes a inhibitory signal can lead to activation. Basically the idea is to inhibit the inhibitor.
\begin{figure}[H]
	\centering
	\includegraphics[width=0.45\textwidth]{inhib_ex}
	\caption{Example pathway of an inhibitory signal leads to activation. Note that the left and right path show the same pathway. }
\end{figure}

\subsubsection{Initiating the Signal}

A signal starts through a protein being in \textbf{close proximity} to the signaling compound. This proximity is key and the minimum for a signal to be start, something ATP is also required.

There are three main types of starts to signaling:
\begin{enumerate}
	\item \textbf{Preassembled signaling complex}
	\begin{itemize}
		\item The signal complex is already assembled with all its intracellular signaling proteins.
	\end{itemize}
	\item\textbf{ Protein recruitment}
	\begin{itemize}
		\item the signaling proteins are in close proximity. Once the the signal molcule attaches they attach to the receptor
		\item Sometimes the signaling proteins don't need to actually be attached, but just being in close proximity is enough.
	\end{itemize}
	\item \textbf{Lipid recruitment}
	\begin{itemize}
		\item Instead of having the signaling proteins attach to the receptor they attach to a Phosphoinositides (PI, a type of phospholipid) in the proximity.
		\item These special phospholipids a.k.a. PIs a.k.a. Phospatidylinositol are part of the membrane and in close proximity to the receptor. These phospholipids can be phosphorylated in the cell, which them allows the signaling proteins to attach.
	\end{itemize}
\end{enumerate}

\begin{figure}[H]
	\centering
	\subfigure[Preassembled complex]{
		\includegraphics[width=0.45\textwidth]{start_pre}
	}
	\hfill
	% Second subfigure
	\subfigure[Protein recruitment]{
		\includegraphics[width=0.5\textwidth]{start_protein}
	}
	\hfill
	% Second subfigure
	\subfigure[Lipid recruitment]{
		\includegraphics[width=0.5\textwidth]{start_lipid}
	}
	\caption{The three classes of cell-surface receptors}
\end{figure}

These signaling complex's got their name because they can get very complex. They are formed using \textbf{modular interaction domains}. Here is an example of an insulin receptor:

\begin{figure}[H]
	\centering
	\includegraphics[width=0.45\textwidth]{start_ins}
	\caption{insulin signaling complex as an example for the complexity and modularity of a signaling complex.}
\end{figure}
The shortcuts of the molecules in the picture:
\begin{itemize}
	\item PH = Pleckstrin Homology - binds to phosphorylated PI's
	\item PTB = Phosphotyrosine Binding - binds phosphotyrosine
	\item SH = Src Homology, Src is on the first signaling proteins identified in a viral induced chicken sarcoma.
	\item IRS = Insulin Receptor Substrate
\end{itemize}


\subsubsection{Regulating and Dampening the Signal}

One way a cell can add extra regulation to a pathway is to require multiple independent pathways to integrate for them to signal downstream, called \textbf{signal integration}.
\begin{figure}[H]
	\centering
	\includegraphics[width=0.45\textwidth]{reg_int}
	\caption{An example pathway showing how multiple streams need to come together to allow downstream signaling.}
\end{figure}

Now, this need for multiple proteins gives the cell the power to change the response duration and strength depending on how it changes the production and degradation rate of each protein. 

\begin{figure}[H]
	\centering
	\subfigure[Equal half life and production (responsive)]{
		\includegraphics[width=0.3\textwidth]{reg_nor}
	}
	\hfill
	% Second subfigure
	\subfigure[shorter half life, same production (possibly unresponsive)]{
		\includegraphics[width=0.3\textwidth]{reg_deg}
	}
	\hfill
	% Second subfigure
	\subfigure[shorter half life and faster production (responsive)]{
		\includegraphics[width=0.3\textwidth]{reg_prod}
	}
	\caption{An example of how influencing the half life and production of certain proteins can seriously influence the responsiveness of a protein complex.}
\end{figure}

To further show the role and importance of degradation in protein complexes here are some graphs we're gonna digest: 
\begin{figure}[H]
	\centering
	\includegraphics[width=0.45\textwidth]{reg_graph}
	\caption{Shows the connection between half-life (blue time), with it degradation, and synthesis rate (y-axis). The x-axis is the time after the synthesis rate is increased by a factor 10. A protein with a short half-life will probably also have a high production rate (red line) as otherwise it will be unresponsive (see picture above). Hence it will react much more strongly to a signal than a protein with a long half-life and slow production rate (green line).}
\end{figure}



The next key concept is \textbf{negative and positive feedback}. This is when a downstream molecule will signal upwards in the pathway to either increase or decrease its activity.
\begin{figure}[H]
	\centering
	\includegraphics[width=0.45\textwidth]{reg_feed}
	\caption{Shows both positive and negative feedback. Note that B is often not directly after A but somewhere down the chain (so T not B)}
\end{figure}

\begin{figure}[H]
	\centering
	\includegraphics[width=0.45\textwidth]{reg_delay}
	\caption{Shows both a positive and negative feedback loop. Below the pathway it also shows the effects of the feedback in a time vs. enzyme activity graph.}
\end{figure}
Discussion of the image: in feedback the positive one is pretty straightforward. You add positive feedback the signal gets extended. For negative feedback the delay with which the delay arrives plays a major role, as the dampening gets stronger the more is being produced upstream. So, the reaction is stronger there more upstream there is.
\begin{itemize}
	\item Short delay: in this case after a short pretty strong response it finds a stable dampened state pretty quickly.
	\item Long delay: here the signal becomes strong, meanign we get a strong but delayed feedback reaction, once that feedback hits, it kills the signal too strongly, so the feedback gets turned back really strongly. That again allows the signal to become strong and we start over again. \\
\end{itemize}

Next up is \textbf{adapting the extracellular signal molecule}. This will lead to the desensitization or sensitization of the signal molecule. This happens mainly by messing around with the receptor protein, it's quantity and function. It is often done through phosphorylation or ubiquitylation of the receptor proteins. Some are also cases of feedback:

\begin{figure}[H]
	\centering
	\includegraphics[width=0.5\textwidth]{reg_ext}
	\caption{A bunch of ways the extracellular signal molecule's strength on the pathway can be adapted. This happens mainly by messing around with the receptor protein, it's quantity and function.}
\end{figure}

\subsubsection{All or nothing, Hyperbolic, Sigmoidal Signals}

There are three main shapes a signal response will take on: 
\begin{itemize}
	\item \textbf{hyperbolic}: a gradually increasing cell response to a gradually increasing signal, eventually reaching a plateau.
	\item \textbf{sigmoidal}: it takes a while for the signal to take effect, but then results in a steeper reaction at some intermediate concentration
	\item \textbf{All or nothing}: extreme form; nothing happens until a certain concentration threshold is reached and then we get a full signal.
\end{itemize}
\begin{figure}[H]
	\centering
	\includegraphics[width=0.5\textwidth]{resp_types}
	\caption{The three shapes a response tends to take in reaction to the signal. This is determined by how it is processed.}
\end{figure} 

When we analyze cells it is important to remember that we are taking an average response of all cells. While a hyperbolic response average is probably hyperbolic in all cells, what appears to be sigmoidal could actually be a all or nothing response with some cells firing and others doing nothing. Hence, it is important to analyze the individual cells too. Here is a visualization:
\begin{figure}[H]
	\centering
	\includegraphics[width=0.5\textwidth]{resp_trick}
	\caption{What appears to be sigmoidal may actually be all or nothing.}
\end{figure} 




\section{The World of G-proteins}

\subsection{The Components of a G-Protein Pathway}

Guanine nucleotide-binding proteins or G-proteins are a major type of cell-surface receptor. There are many different types of G proteins.

\subsubsection{G-Protein-Coupled Receptor or GPCR}

the G-protein is the place the ligand attaches too. Then the GPCR will activate the G-protein. GPCR uses trimeric G-proteins.

\textbf{Structure}: A GPCR has seven transmembrane regions, composed of 7 alpha helices and 6 loops. It has a N-terminal extracellular region and a C-terminal intracellular region. The alpha helices form a pocket for the ligand to bind. Depending on the size of the ligand GPCR will have a differently sized extracellular domain to accomadate for the ligand, while remaining specific. There are over 700 different GPCR in humans.

\begin{figure}[H]
	\centering
	\includegraphics[width=0.45\textwidth]{GPCR}
	\caption{Simplified image of GPCR in membrane and its 3D structure.}
\end{figure}

\subsubsection{Heterotrimeric G Protein}

\begin{figure}[H]
	\centering
	\includegraphics[width=0.45\textwidth]{G_structure}
	\caption{Simplified image of G-Protein in membrane and its 3D structure.}
\end{figure}

Some quick facts:
\begin{itemize}
	\item 3 proteins that make up the complex that makes up the G-protein.
	\item GPCR uses trimeric G-proteins.
	\item at least 20 different alpha subunits exist.
	\item are there are numerous different beta and gamma complexes, meaning we have quite a number of different G-proteins out there.
\end{itemize}

There are a bunch of different trimeric G-proteins, which are split into four families, which all have different functions:
\begin{figure}[H]
	\centering
	\includegraphics[width=0.45\textwidth]{G_families}
	\caption{Shows the four major families of trimeric G-proteins.}
\end{figure}

\textbf{Function}: The Ras domain is part of the alpha subunit and is related to GTPases and provides a face for GDP/GTP to bind too. The alpha helix (AH) domain binds it in place. Activation from a GPCR triggers the release of GDP from the alpha subunit followed by the binding of GTP
 
\subsubsection{Activation of G-Protein by GPCR}

Here is how GPCR activates a G-protein:
\begin{enumerate}
	\item An extracellular signal molecule binds to the GPCR molecule;
	\item The GPCR molecule changes conformation, which allows it to bind to the Ras domain of the G-protein;
	\item This alters the conformation of the alpha subunit, specifically the alpha helix subunit, releasing the GDP.
	\item The binding of GTP then promotes the closing of the subunit
	\item This triggers conformational changes causing the alpha subunit to dissociate from both the GPCR as well as the beta-gamma subunit.
	\item Both the alpha and the beta-gamma subunit then become active in downstream pathways.
	\item GPCR stays active as long as the ligand is bound to it, meaning it can activate many G-proteins.
\end{enumerate}

How an activated G-protein starts the downstream cascade:
\begin{figure}[H]
	\centering
	\includegraphics[width=0.45\textwidth]{G_usage}
	\caption{Once we have activated the G-protein and the subunits split, once they are used and the GTP cpnverted to GDP the inactive subunits merge back together and the cycle can start again.}
\end{figure}

The main \textbf{downstream targets}: 
\begin{itemize}
	\item Adenylate (adenylyl) cyclase, which in turn increases or decreases cAMP (very common target);
	\item Channels;
	\item Phosopholipase C, in turn generates IP3 and diacylglycerol.
\end{itemize}

\subsubsection{Stopping GPCR signaling}

The signalling of GPCR can be stopped through GPCR kinases (GRKs) and arrestins, as they cause desensitization of the GPCR. The process:
\begin{enumerate}
	\item \textbf{Negative feedback}: Activated GPCR stimulates GRKs which phosphorylate the GPCR on multiple sites. 
	\item This then allows the arrestin to bind to he GPCR desensitizing it. 
\end{enumerate}
\begin{figure}[H]
	\centering
	\includegraphics[width=0.45\textwidth]{GRk}
	\caption{How GRK is a negative regulator, through negative feedback, for GPCR.} 
\end{figure}




\subsection{GPCR signaling through Cyclic AMP a.k.a. cAMP}
\subsubsection{cAMP}

Cyclic AMP a.k.a. cAMP is a derivate of ATP. Two phosphates are replaced by a sugar bond (with enzyme adenylyl cyclase). cAMP is a shortlived molecule which is "uncycled" to 5'-AMP (with enzyme cAMP phosphodiesterase). The fact the molecule is so shortlived makes it great as a signaling molecule.
\begin{figure}[H]
	\centering
	\includegraphics[width=0.45\textwidth]{cAMP_prod}
	\caption{The production of cAMP with the enzymes adenylyl cyclase and cAMP phosphodiesterase.}
\end{figure}

\subsubsection{cAMP as a signaling molecule}

The pathway is as follows:

\begin{figure}[H]
	\centering
	\includegraphics[width=0.45\textwidth]{cAMP_path}
	\caption{The production of cAMP with the enzymes adenylyl cyclase and cAMP phosphodiesterase.}
\end{figure}


\begin{enumerate}
	\item \textbf{Activation of GPCR}: GPCR gets activated, which in turn activates the G-protein.
	\item \textbf{cAMP produced}: The activated adenylyl cyclase converts ATP into cAMP.
	\item \textbf{ Activation of PKA}: The main role of cAMP is the activation cAMP-dependent protein kinases (PKAs). By binding to the regulatory subunits of the PKA tetramer induces a conformational change, which makes the regulatory subunits to dissociate from the catalytic subunits activating them. This release requires multiple cAMPs per regulatory unit. This means a lot of cAMP is required, as cAMP quickly decays, so we get a pretty sharp response.
	\begin{figure}[H]
		\centering
		\includegraphics[width=0.3\textwidth]{cAMP_PKA}
		\caption{Shows the activation of PKAs by cAMP.}
	\end{figure}
	\item The active PKA is then translocated to the nucelus, where it activates a transcription factor CREB (cAMP response binding protein) through phosphorylation.
	\item CREB interacts with CREB-binding protein and activates transciption on the cAMP response element.
\end{enumerate}

In a cell it will respond to activation of GPCR by a ligand, say serotonin, increasing the concentration 20fold in a matter of seconds. Depending on the cell and the ligand we will get very different cell responses. Here are some examples with hormones:
\begin{figure}[H]
	\centering
	\includegraphics[width=0.45\textwidth]{cAMP_horm}
	\caption{The expression using different hormones in different cells. Vasopressin is also a "love" hormone, meanign that when you are in love GPCR is active}
\end{figure}

\subsubsection{cGMP}

Cyclic-Guanine-Mono-Phosphate or cGMP is an alternative in the cAMP pathway. So, sometimes cGMP is activated by GPCR not cAMP. The only difference is the guanine instead of adenosine. The enzyme is called guanylate cyclase. 
\begin{figure}[H]
	\centering
	\includegraphics[width=0.45\textwidth]{cGmp}
	\caption{The structure of cGMP, where the only difference to cAMP is the guanine for adenosine.}
\end{figure}

\subsubsection{Case study with response to light}

The process of recognition is as follows:
\begin{enumerate}
	\item a \textbf{rodopsin} molecule absorbs a photon;
	\item 500 G-proteins molecules (transducin) are activated (signal is amplified);
	\item 500 cGMP phosphodiesterase molecules are activated;
	\item $10^{5}$ cGMPs are hydrolyzed (signal has been amplified);
	\item They block 250 cation channels;
	\item $10^{6} - 10^{7} Na^{+}$-ions per second are prevented from entering the cell for a period of around a second (signal has been amplified);
	\item The membrane potential is altered by 1mV, which in turn relays a signal to the brain. 
\end{enumerate}

\begin{figure}[H]
	\centering
	\includegraphics[width=0.45\textwidth]{cGMP_rhod}
	\caption{How the cell changes when light is received. Note that the signal sent to the brain is inversed of how a normal neuron is fired (channels close instead of open).}
\end{figure}

\subsection{GPCR Signaling through phosphlipase C}

Here are some example cell responses where GPCRs activate PLC$\beta$
\begin{figure}[H]
	\centering
	\includegraphics[width=0.45\textwidth]{PLC_ex}
	\caption{Some cell responses where GPCR activates PLC$\beta$, shoutout to Vasopressin for all the lovin'.}
\end{figure}

\subsubsection{Case Study: GPCRs activating Cytosolic Ca2+ and activating protein Kinase C}
\begin{figure}[H]
	\centering
	\includegraphics[width=0.45\textwidth]{PLC_Ca}
	\caption{Some cell responses where GPCR activates PLC$\beta$, shoutout to Vasopressin for all the lovin'.}
\end{figure}
Rundown of the pathway:
\begin{enumerate}
	\item The GPCR activates the PLC$\beta$ via a G-protein called Gq. The Gq Beta-gamma complex and the alpha-complex activate with the PLC$\beta$. 
	\item PLC$\beta$ hydrolyzes PI(4,5)P2, causing it to split into two messengers
	\item IP3 diffuses through the cytosol and releases Ca2+ from the ER by binding to the IP-gated Ca2+ channels.
	\item Then the diacyglycerol (other part of PI(4,5)P2), remaining in the membrane, together with the Ca2+ and phophatidylserine activate the protein Kinase C (PKC). Of the min. 10 forms of PKC at least 4 are activated by diacylglycerol.
\end{enumerate}

\subsubsection{Ca2+ feedback waves and oscillations}

The concentration of Ca2+ plays a big role in activating or inactivating. So, givign itself positive or negative feedback. Here's how:
\begin{itemize}
	\item \textbf{Activation}: At low concentrations Ca2+ goes to neighboring channels and activates them, causing the release of more Ca2+ and a wave like reaction across receptors (first couple pics). This means that the channels can stay active even without any IP3 being present.
	\item \textbf{Inactivation}: When Ca2+ is present at very high concentrations it inactivates the channels. That means that now we create a wave of inactivation.
	\item \textbf{Oscillation}: In the continued presence of the ligand activator, or even without, this mix of feedback can cause oscillations in Ca2+ excretion.
\end{itemize}
\begin{figure}[H]
	\centering
	\includegraphics[width=0.45\textwidth]{Ca2_waves}
	\caption{Shows how Ca2+ influences the activity of its own channels}
\end{figure}

\subsubsection{How Ca2+ plays an important role in regulating and relaying signals}

\textbf{Ca2+ and calmodulin}: With the help of Ca2+/calmodulin, Ca2+ is able to bind to target proteins and with that relay the signal. The dumbbell shape of calmodulin and alpha-helix allows it to take on numerous different conformations.
\begin{figure}[H]
	\centering
	\includegraphics[width=0.45\textwidth]{Ca2_cal}
	\caption{On the left the structure of calmodulin and on the right an example of how it can bind to target proteins (this move is called the jackknife).}
\end{figure}

\textbf{CaM-Kinase II} is regulated by calmodulin. Here's how that runs down:
\begin{enumerate}
	\item 6 CaM-KII (green) form a ring. 
	\item The kinase domains pop in and out naturally.
	\item Calmodulin can bind the popped-out domain in place when it is bound to Ca2+
	\item Then that kinase domain gets phosphorylated, making it active.
	\item In the continued presence of calmodulin it is even more active.
	\item Becomes inactive through dephosphorylation
	\item The more domains are active, the more active the enzyme as a whole.
\end{enumerate}
\begin{figure}[H]
	\centering
	\includegraphics[width=0.45\textwidth]{Ca2_cam}
	\caption{Shows how Ca2+/calmodulin activate enzymes, in this case CaM-KII.}
\end{figure}

Depending on the frequency of the Ca2+ oscillations, the activity of the enzyme is influenced in major ways. The more frequent the oscillations the more the activity as a whole will rise. Hence CaM-KII is a good mechanism of decondign the frequencies of oscillations in a cell. Here's a figure to visualize:
\begin{figure}[H]
	\centering
	\includegraphics[width=0.45\textwidth]{Ca2_osc}
	\caption{How different frequencies of oscillations cause major differences in enzyme activity.}
\end{figure}




\section{Other Types of Signaling}
\subsection{Receptor Tyrosine Kinase a.k.a. RTK signaling}

Receptor Tyrosine Kinases are a large group, here are some of them:
\begin{figure}[H]
	\centering
	\includegraphics[width=0.45\textwidth]{RTk_ex}
	\caption{A bunch of different RTK groups}
\end{figure}

RTKs are connected by the fact that they all have an intracellular kinase domain which can phosphorylate a Tyrosine. The extracellular domain on the other hand is completely variable. the kinase region can also have a kinase insert region emerging from it, which is relevant for interactions with other proteins.

\begin{figure}[H]
	\centering
	\includegraphics[width=0.45\textwidth]{RTk_types}
	\caption{A bunch of different RTK types, with the core features.}
\end{figure}

\subsubsection{Activation of RTKs by Dimerization}

\begin{enumerate}
	\item Two RTKs are initially inactive until some type of ligand arrives to bring them together.
	\item In this proximity the RTKs dimerize and make an initial Tyrosine auto-phosphorylation.
	\item Once the first phosphorylation has happened that initiates trans-phosphorylation of several Tyrosines.
	\item Phospho-Tyrosin sites recruit and/or activate downstream signaling proteins.
\end{enumerate}
\begin{figure}[H]
	\centering
	\includegraphics[height=0.3\textwidth]{RTk_act}
	\caption{Activation of a RTK dimer}
\end{figure}

\subsubsection{An exception to the rule: Activation of EGF Kinase}

Compared to the regular activation the kinase domains are not both auto-phosphorylated to be activated. We still have two identical domains, but one takes on the role of activator, while the other is the receiver.

\begin{enumerate}
	\item Both domains are activated through EGF.
	\item Then the activator pushes on the receiver, causing a conformational change in the receiver domain, activating it. 
	\item The activated Kinase domain then phosphorylates the tyrosines on both receptors.
\end{enumerate}

\subsection{Binding to the Receptor}

The phospho-tyrosines are docking sites for proteins containing:
\begin{enumerate}
	\item SH2 (Src Homology, the sarcoma), cancer bro
	\item PTB (PhosphoTyrosine Binding), cancer bro
	\item PLC 
\end{enumerate}

Because of the multitude of phospho-tyrosines many different proteins, and consequently different pathways, can interact with the receptors.

\subsubsection{Ras signaling}

Ras is essentialy a monomeric GTPase. Ras is anchored to the membrane through a lipid modification. For a refresher on how GTP can be regualted please refer to Section \ref{sec:GTP}.

Here are some groups:
\begin{figure}[H]
	\centering
	\includegraphics[height=0.3\textwidth]{Ras_ex}
	\caption{Ras groups in our body.}
\end{figure}

\textbf{Activation of Ras by an  RTk}: 
\begin{enumerate}
	\item Adaptor protein Grb2 docks to RTk with SH2
	\item Ras-GEF then interacts with Grb2
	\item Ras-GEF then exchanges the GDP for a GTP
	\item Ras is activated.
\end{enumerate}

\textbf{Detecting Ras activity}: We use FRET (Fluorescence resonance energy  transfer), by attaching a yellow fluorescent protein (YFP) to the gene of Ras. Then we add a red fluorescent dye to GTP. That way when no GTP is there (Ras inactive), it emits yellow light, but when GTP is attached to the Ras (active) red light is emitted.
\begin{figure}[H]
	\centering
	\includegraphics[height=0.3\textwidth]{Ras_FRET}
	\caption{How FRET can be used to detect the activity of Ras.}
\end{figure}

\textbf{Case study: MAP kinase module} is a module activated by Ras. This is done the following way: Ras activates Raf to the membrane, which in turn activates Mek, which in turn activates Erk, which phosphorylates a bunch of downstream proteins, such as further kinases, and transciption regulators. The resulting activations cause complex changes in the cell.
\begin{figure}[H]
	\centering
	\includegraphics[height=0.3\textwidth]{Ras_MAP}
	\caption{How the MAp kinase module is activated by Ras.}
\end{figure}

\textbf{How cancer changes Ras}: By changing certain amino acids (G12, G13, Q61), the mutants show impaired GTPase activity, leading to a gain-of-function. So, the GAP proteins no longer work as well. Of the three types of famous Ras (K,H,N-Ras) it seems mostly the mutant K-Ras is found in cancers. Mutant Ras probably important in the initiation of tumors.
\begin{figure}[H]
	\centering
	\includegraphics[height=0.3\textwidth]{Ras_canc}
	\caption{Showing how mutating the GAP messes with the regulation of Ras.}
\end{figure}

\subsubsection{PI3K signaling}

In order to kick off the splitting PLC to diaglycerol and IP3, it first needs to phosphorylated at the 3-carbon of the PI, by a PI 3-Kinase (hmmmm I wonder where that name comes from). It can also create docking sites for downstream proteins.

\begin{figure}[H]
	\centering
	\includegraphics[height=0.3\linewidth]{PI3_phos}
	\caption{Phosphorylation of the 3-carbon activates the PLC.}
\end{figure}

\textbf{PI 3-Kinase activates AKT}:
\begin{enumerate}
	\item PI3K is recruited by RTK
	\item PI3K creates docking sites where proteins with a PH domain can dock.
	\item PDK1 activates AKT by phosphorylation
	\item p-AKT activates many cellular programs including cell growth and anti-apostosis (hinting that cancer may be interested here).
\end{enumerate}


\textbf{Negative regulation}: Phosphatase PTEN removes phosphate from PI(3,4,5)P3, making it a negative regulator of the PI3K. \\

\textbf{Case study: how PI 3-kinase promotes cell survival}:
\begin{figure}[H]
	\centering
	\includegraphics[height=0.4\textheight]{PI3_surv}
	\caption{How PI3K inihbits cell death.}
\end{figure}

The figures below shows on the left the entire pathway which promotes cell growth. The one on the left shows how the chain on the left continues leading to cell growth. The MAP kinase can also go down the pathway on the right, meaning bot can promote cell growth.
\begin{figure}[H]
	\centering
	\subfigure[cell growth whole pathway]{
		\includegraphics[width=0.65\textwidth]{PI3_surv}
	}
	\hfill
	% Second subfigure
	\subfigure[cell growth comparision of PI3K active vs inactive]{
		\includegraphics[width=0.3\textwidth]{PI3_comp}
	}
\end{figure}

\subsubsection{Checking out how GPCRs and RTKs are intertwined}
\begin{figure}[H]
	\centering
	\includegraphics[width=0.5\linewidth]{GPCR_RTK}
	\caption{Compares the pathways caused by GPCRs and RTKs, also shows which are shared. All of them end with a Kinase, which then causes a reaction chain downstream}
\end{figure}

Both GPCR and RTK have a PLC pathway, called beta and gamma respectively. The effect is very similar.


\subsection{EGF Receptors in cancer}

In cancer EGF-R can become over-activated.

\begin{figure}[H]
	\centering
	\includegraphics[width=0.4\linewidth]{EGF_canc}
	\caption{Different ways EGFR can become overexpressed in cancer}
\end{figure}

Mutations in cancer occur at the kinase domain of the receptor, leading to the domain being permanently active. Thus Ras-MAP Kinase and PI3K are always active, leading to uncontrolled cell growth a.k.a. cancer. These mutations will be deletions which keep the reading frame or missense mutations:
\begin{figure}[H]
	\centering
	\includegraphics[width=0.4\linewidth]{EGF_muta}
	\caption{Showing the options of both missense and frame-keeping deletions, found through Sanger Sequencing.}
\end{figure}

Inhibitors can target always active EGFR. Through this some pathways are now slowed down, leading to more normalized cell growth.
\begin{figure}[H]
	\centering
	\includegraphics[width=0.4\linewidth]{EGF_inhi}
	\caption{How an inhibtor restore normal cell growth.}
\end{figure}



\end{document}