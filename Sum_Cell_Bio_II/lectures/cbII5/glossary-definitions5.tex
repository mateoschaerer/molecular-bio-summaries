\newglossaryentry{AdenylateCyclase}{
	name={Adenylate Cyclase},
	description={Also called adenylyl cyclase, this membrane-bound enzyme converts ATP to cAMP in response to stimulation by G-proteins. It is a key player in many GPCR-mediated pathways.}
}



\newglossaryentry{Allornothingsignal}{
	name={All or nothing signal},
	description={A type of cellular response that occurs only after a threshold level of signal is reached, resulting in a binary, digital-like outcome.}
}


\newglossaryentry{AlphaHelix}{
	name={Alpha Helix (AH)},
	description={A common structural motif in proteins, including in G-proteins and GPCRs, where it can play a role in conformational change upon activation.}
}



\newglossaryentry{AlphaSubunit}{
	name={Alpha Subunit},
	description={The component of a heterotrimeric G-protein that binds GDP/GTP and dissociates upon activation to regulate downstream effectors such as adenylate cyclase or phospholipase C.}
}



\newglossaryentry{Arrestins}{
	name={Arrestins},
	description={Proteins that bind phosphorylated GPCRs, blocking further G-protein activation and targeting receptors for internalization or alternate signaling.}
}



\newglossaryentry{AutocrineSignaling}{
	name={Autocrine Signaling},
	description={A form of signaling in which a cell secretes signaling molecules that bind to receptors on its own surface, allowing it to regulate itself.}
}



\newglossaryentry{Autophosphorylation}{
	name={Autophosphorylation},
	description={A process in which a kinase adds a phosphate group to itself, often leading to sustained activation independent of the original signal, as seen in CaM-Kinase II.}
}



\newglossaryentry{BetaComplex}{
	name={Beta Complex},
	description={Part of the G-protein beta-gamma dimer, it remains membrane-associated and contributes to the regulation of ion channels and other signaling proteins.}
}



\newglossaryentry{Calcium}{
	name={Ca\textsuperscript{2+}},
	description={A ubiquitous intracellular second messenger that regulates a wide range of cellular processes including muscle contraction, secretion, metabolism, and gene expression. Its release is often triggered by IP\textsubscript{3} in response to upstream signaling events.}
}


\newglossaryentry{Calmodulin}{
	name={Calmodulin},
	description={A small calcium-binding protein that undergoes conformational change upon Ca\textsuperscript{2+} binding, enabling it to activate target enzymes such as CaM-Kinase II.}
}



\newglossaryentry{CaMKII}{
	name={CaM-Kinase II},
	description={Short for calcium/calmodulin-dependent protein kinase II, an important serine/threonine kinase that decodes calcium oscillations via autophosphorylation and regulates memory, gene expression, and metabolism.}
}



\newglossaryentry{cAMP}{
	name={cAMP},
	description={Short for cyclic adenosine monophosphate, a second messenger synthesized by adenylate cyclase that activates downstream targets like PKA and regulates cellular responses.}
}



\newglossaryentry{cAMPPhosphodiesterase}{
	name={cAMP Phosphodiesterase},
	description={An enzyme that degrades cAMP into AMP, thereby terminating the cAMP signaling pathway. It is a key modulator of signal duration.}
}



\newglossaryentry{CellSignaling}{
	name={Cell Signaling},
	description={A fundamental process by which cells detect, interpret, and respond to external or internal cues through molecular signals. It involves extracellular signaling molecules binding to specific receptors, triggering intracellular signaling cascades that regulate cellular functions such as gene expression, metabolism, division, or apoptosis. Cell signaling enables coordination in multicellular organisms and is essential for development, immune response, and homeostasis.}
}



\newglossaryentry{CellSurfaceReceptor}{
	name={Cell-Surface Receptor},
	description={A transmembrane protein located on the cell membrane that binds extracellular signaling molecules (ligands), such as hormones or neurotransmitters. Upon ligand binding, it initiates an intracellular signaling cascade without the ligand entering the cell. Major classes include ion-channel-coupled receptors, G-protein-coupled receptors, and enzyme-coupled receptors.}
}



\newglossaryentry{cGMP}{
	name={cGMP},
	description={Short for cyclic guanosine monophosphate, a second messenger similar to cAMP that is produced by guanylate cyclase and regulates processes like phototransduction and vasodilation.}
}



\newglossaryentry{cGMPPhosphodiesterase}{
	name={cGMP Phosphodiesterase},
	description={An enzyme activated in visual transduction that hydrolyzes cGMP to GMP, leading to the closing of ion channels in photoreceptor cells.}
}



\newglossaryentry{ConstitutivelyActive}{
	name={Constitutively Active},
	description={Describes a receptor or signaling protein that is active without the need for ligand binding, often due to mutations or abnormal expression.}
}


\newglossaryentry{ContactDependentSignaling}{
	name={Contact Dependent Signaling},
	description={A signaling mechanism that requires direct membrane-to-membrane contact between cells, typically involving membrane-bound ligands and receptors.}
}


\newglossaryentry{CREB}{
	name={CREB},
	description={Short for cAMP response element-binding protein, a transcription factor phosphorylated by PKA that regulates genes involved in memory, survival, and metabolism.}
}



\newglossaryentry{DAG}{
	name={DAG},
	description={Short for diacylglycerol, a lipid-derived second messenger produced by PLC that activates protein kinase C and regulates membrane-associated signaling.}
}



\newglossaryentry{DownstreamCascade}{
	name={Downstream Cascade},
	description={A sequence of biochemical events triggered by receptor activation, involving multiple intermediates and amplifying the original signal to produce a cellular response.}
}



\newglossaryentry{Effectorproteins}{
	name={Effector proteins},
	description={Proteins that execute the final cellular response to a signal, such as changes in gene expression, metabolism, or cytoskeletal structure.}
}


\newglossaryentry{EndocrineSignaling,hormonalsignaling}{
	name={Endocrine Signaling},
	description={A.k.a. hormonal signaling, Long-range signaling in which hormones are secreted into the bloodstream and act on distant target cells.}
}


\newglossaryentry{Enzymecoupledreceptors}{
	name={Enzyme coupled receptors},
	description={Transmembrane receptors that have intrinsic enzymatic activity or are associated with enzymes activated by ligand binding.}
}


\newglossaryentry{ExtracellularSignalingMolecule}{
	name={Extracellular Signaling Molecule},
	description={A molecule, such as a hormone or neurotransmitter, that is released from one cell to bind receptors on another and initiate signaling.}
}


\newglossaryentry{G-protein-coupledreceptors}{
	name={G-protein-coupled receptors},
	description={A large family of membrane receptors (GPCRs) that activate intracellular G-proteins upon ligand binding to transmit signals.}
}


\newglossaryentry{GammaComplex}{
	name={Gamma Complex},
	description={Forms a functional dimer with the beta subunit in heterotrimeric G-proteins, anchoring the complex to membranes and participating in downstream signaling.}
}



\newglossaryentry{GAP}{
	name={GAP},
	description={GTPase-activating proteins that enhance the intrinsic GTPase activity of G-proteins, leading to signal termination.}
}


\newglossaryentry{GAPJunctions}{
	name={GAP Junctions},
	description={Specialized intercellular connections that allow direct chemical communication between adjacent cells via diffusion of small molecules and ions.}
}



\newglossaryentry{GEF}{
	name={GEF},
	description={Guanine nucleotide exchange factors, proteins that activate GTP-binding proteins by promoting the exchange of GDP for GTP.}
}


\newglossaryentry{GPCR}{
	name={GPCR},
	description={A family of seven-pass transmembrane receptors that activate intracellular G-proteins in response to extracellular ligands. They are among the most abundant and versatile signaling receptors in eukaryotic cells, triggering signaling cascades including those that lead to integrin activation.}
}



\newglossaryentry{GPCRKinases}{
	name={GPCR Kinases (GRKs)},
	description={A family of kinases that phosphorylate activated GPCRs, initiating their desensitization by promoting arrestin binding.}
}



\newglossaryentry{GProteins}{
	name={G-Proteins},
	description={Short for guanine nucleotide-binding proteins, these molecular switches relay signals from receptors (like GPCRs) to intracellular effectors by cycling between GDP-bound (inactive) and GTP-bound (active) states.}
}



\newglossaryentry{Gq}{
	name={Gq},
	description={A subclass of heterotrimeric G-proteins that activates phospholipase C, leading to intracellular calcium release and activation of protein kinase C.}
}



\newglossaryentry{GTPases}{
	name={GTPases},
	description={Enzymes that hydrolyze GTP to GDP and phosphate, acting as molecular switches in signaling pathways.}
}


\newglossaryentry{GTPbinding}{
	name={GTP binding},
	description={A regulatory mechanism by which proteins, especially G-proteins, toggle between active and inactive states depending on GTP or GDP binding.}
}


\newglossaryentry{GuanylateCyclase}{
	name={Guanylate Cyclase},
	description={An enzyme that converts GTP to cGMP upon activation by nitric oxide or natriuretic peptides, initiating cGMP-mediated signaling pathways.}
}


		
		
		
\newglossaryentry{HeterotrimericGProtein}{
	name={Heterotrimeric G Protein},
	description={A type of G-protein composed of three distinct subunits—alpha, beta, and gamma—that relay signals from GPCRs to downstream effectors.}
}



\newglossaryentry{Hyperbolicsignal}{
	name={Hyperbolic signal},
	description={A graded signal response that increases steadily with ligand concentration and eventually plateaus, resembling Michaelis-Menten kinetics.}
}


\newglossaryentry{Inhibitorysignals}{
	name={Inhibitory signals},
	description={Signals that suppress or diminish cellular responses, often balancing excitatory pathways for proper cell regulation.}
}


\newglossaryentry{InsulinReceptorSubstrate(IRS)}{
	name={Insulin Receptor Substrate (IRS)},
	description={A docking protein phosphorylated by the insulin receptor, serving as a scaffold for downstream signaling molecules.}
}



\newglossaryentry{IntracellularReceptor}{
	name={Intracellular Receptor},
	description={A receptor located within the cytoplasm or nucleus that binds small, hydrophobic signaling molecules (e.g., steroid hormones) that cross the plasma membrane. Upon activation, many intracellular receptors function as transcription factors that directly modulate gene expression.}
}



\newglossaryentry{Ion-channel-coupledreceptors}{
	name={Ion-channel-coupled receptors},
	description={Receptors that open or close ion channels in response to ligand binding, converting chemical signals into electrical ones.}
}


\newglossaryentry{Lipidrecruitment}{
	name={Lipid recruitment},
	description={The process of signaling molecules being recruited to specific membrane lipids, such as PIP3, for spatial activation.}
}


\newglossaryentry{Longfeedbackdelay}{
	name={Long feedback delay},
	description={A feedback loop that acts over a longer time scale, potentially leading to oscillations or long-term regulation.}
}


\newglossaryentry{ModularInteractionDomain}{
	name={Modular Interaction Domain},
	description={Protein domains that mediate specific interactions with phosphorylated or lipid-modified partners in signaling complexes.}
}


\newglossaryentry{Molecularswitches}{
	name={Molecular switches},
	description={Molecules, often proteins, that toggle between 'on' and 'off' states to propagate or terminate signals.}
}


\newglossaryentry{NegativeFeedback}{
	name={Negative Feedback},
	description={A regulatory mechanism in which a signaling output inhibits an earlier step, stabilizing the pathway.}
}


\newglossaryentry{Neurotransmitter}{
	name={Neurotransmitter},
	description={A chemical messenger that transmits signals across synapses from one neuron to another.}
}

%Section - Cell Signaling: the World of G-Proteins


\newglossaryentry{Oscillation}{
	name={Oscillation},
	description={In cell signaling, a periodic fluctuation in the concentration or activity of signaling molecules (such as Ca\textsuperscript{2+}) that conveys dynamic information to control gene expression or cellular responses.}
}



\newglossaryentry{ParacrineSignaling}{
	name={Paracrine Signaling},
	description={Short-range signaling where secreted molecules affect nearby target cells without entering the bloodstream.}
}


\newglossaryentry{Phosphatidylserine}{
	name={Phosphatidylserine},
	description={A negatively charged phospholipid found on the inner leaflet of the plasma membrane that helps localize signaling proteins like PKC through electrostatic interactions.}
}



\newglossaryentry{PhospholipaseC}{
	name={Phospholipase C (PLC)},
	description={A membrane-associated enzyme activated by certain G-proteins (like Gq), which hydrolyzes phosphoinositides to generate DAG and IP\textsubscript{3}, initiating calcium signaling and PKC activation.}
}



\newglossaryentry{Phosphorylation}{
	name={Phosphorylation},
	description={The addition of a phosphate group to a protein or other molecule, often regulating activity or interactions.}
}


\newglossaryentry{PhosphotyrosineBinding(PTB)}{
	name={Phosphotyrosine Binding (PTB)},
	description={A domain that binds phosphorylated tyrosines on target proteins, mediating recruitment in signaling pathways.}
}


\newglossaryentry{PI}{
	name={PI},
	description={Short for phosphatidylinositol, a membrane phospholipid that can be phosphorylated to form various signaling lipids like PI(4,5)P\textsubscript{2}, which are substrates for PLC and involved in many signaling pathways.}
}



\newglossaryentry{PKA}{
	name={PKA},
	description={Short for protein kinase A, a serine/threonine kinase activated by cAMP that phosphorylates various substrates to regulate metabolism, gene expression, and other cellular processes.}
}



\newglossaryentry{PKC}{
	name={PKC},
	description={Short for protein kinase C, a family of serine/threonine kinases activated by DAG and calcium that phosphorylate a variety of cellular proteins involved in growth, metabolism, and differentiation.}
}



\newglossaryentry{PLC}{
	name={PLC},
	description={Short for phospholipase C, a key enzyme in Gq-mediated signaling that generates second messengers DAG and IP\textsubscript{3} from membrane phospholipids. See also Phospholipase C.}
}



\newglossaryentry{PleckstrinHomology(PH)}{
	name={Pleckstrin Homology (PH)},
	description={A protein domain that binds phosphoinositides in membranes, targeting proteins to specific locations.}
}


\newglossaryentry{PositiveFeedback}{
	name={Positive Feedback},
	description={A mechanism in which a signaling output enhances an earlier step, amplifying the signal.}
}


\newglossaryentry{ProteinRecruitment}{
	name={Protein Recruitment},
	description={The assembly of signaling complexes at specific membrane sites or proteins through binding domains.}
}


\newglossaryentry{ProteinResponse}{
	name={Protein Response},
	description={The cellular outcome of a signaling event, typically involving activation or repression of specific proteins.}
}


\newglossaryentry{RasDomain}{
	name={Ras Domain},
	description={A conserved GTP-binding domain found in small GTPases like Ras, involved in signal transduction and downstream activation of pathways such as MAPK.}
}



\newglossaryentry{Receptor}{
	name={Receptor},
	description={A protein, usually on the cell surface or in the cytoplasm, that binds a specific signaling molecule and initiates a response.}
}


\newglossaryentry{Rhodopsin}{
	name={Rhodopsin},
	description={A light-sensitive GPCR found in photoreceptor cells of the retina that activates the visual transduction pathway via transducin and cGMP breakdown.}
}



\newglossaryentry{Scaffoldingprotein}{
	name={Scaffolding protein},
	description={A protein that binds multiple signaling components, organizing them into functional complexes to enhance efficiency and specificity.}
}


\newglossaryentry{Shortfeedbackdelay}{
	name={Short feedback delay},
	description={A feedback loop that acts rapidly after signal initiation, often stabilizing or fine-tuning the signal.}
}


\newglossaryentry{SigmoidalSignal}{
	name={Sigmoidal Signal},
	description={A type of cellular response curve characterized by a slow initiation, followed by a steep increase, and then saturation—forming an “S” shape. It often reflects cooperative binding or multi-step signaling cascades, allowing cells to respond sensitively to threshold changes in stimulus concentration.}
}



\newglossaryentry{Signalingcascade}{
	name={Signaling cascade},
	description={A series of biochemical events, often involving sequential activation of enzymes, leading to a cellular response.}
}


\newglossaryentry{SignalIntegration}{
	name={Signal Integration},
	description={The cellular process of combining inputs from multiple signaling pathways to generate a unified response.}
}


\newglossaryentry{SrcHomology(SH)}{
	name={Src Homology (SH)},
	description={A family of protein domains (e.g., SH2, SH3) involved in recognizing phosphorylated tyrosines or proline-rich motifs.}
}


\newglossaryentry{SynapticSignaling}{
	name={Synaptic Signaling},
	description={A specialized form of signaling in neurons where neurotransmitters are released at synapses to stimulate adjacent cells.}
}


\newglossaryentry{Transcriptionalresponse}{
	name={Transcriptional response},
	description={Changes in gene expression triggered by signaling pathways reaching the nucleus.}
}


\newglossaryentry{Transducin}{
	name={Transducin},
	description={A heterotrimeric G-protein specifically involved in visual signaling, activated by rhodopsin to stimulate cGMP phosphodiesterase.}
}

\newglossaryentry{HalfLife}{
	name={Half-Life},
	description={The time required for the concentration of a substance—such as a signaling molecule, mRNA, or protein—to decrease to half of its initial value.}
}


\newglossaryentry{AKT}{
	name={AKT},
	description={A serine/threonine-specific protein kinase also known as Protein Kinase B, involved in promoting cell survival and growth through downstream effects of PI3K signaling.}
}

\newglossaryentry{Cholesterol}{
	name={Cholesterol},
	description={A lipid molecule essential for membrane structure and function; it also serves as a precursor for steroid hormones and plays a role in modulating signaling pathways such as Hedgehog.}
}

\newglossaryentry{CK1}{
	name={CK1},
	description={Short for Casein Kinase 1, a serine/threonine kinase that phosphorylates signaling components in the Wnt and Hedgehog pathways.}
}

\newglossaryentry{Costal2}{
	name={Costal2},
	description={A kinesin-like protein in the Hedgehog pathway that forms a complex with Smoothened and regulates the processing of Cubitus Interruptus.}
}

\newglossaryentry{CubitusInterruptus}{
	name={Cubitus Interruptus (Ci)},
	description={A transcription factor regulated by the Hedgehog pathway in Drosophila; acts as a repressor or activator depending on Hedgehog signal presence.}
}

\newglossaryentry{Delta}{
	name={Delta},
	description={A membrane-bound ligand for the Notch receptor that plays a critical role in lateral inhibition during development.}
}

\newglossaryentry{Disheveled}{
	name={Disheveled},
	description={A cytoplasmic protein activated by Frizzled in the Wnt pathway; it inhibits the degradation complex to stabilize β-catenin.}
}

\newglossaryentry{Dimerization}{
	name={Dimerization},
	description={The process by which two receptor molecules associate, often as a prerequisite for activation, especially in receptor tyrosine kinases (RTKs).}
}

\newglossaryentry{EGFKinase}{
	name={EGF Kinase},
	description={A kinase domain found in the Epidermal Growth Factor Receptor (EGFR), involved in autophosphorylation upon ligand binding and dimerization.}
}

\newglossaryentry{EGFR}{
	name={EGF-R},
	description={Epidermal Growth Factor Receptor, a receptor tyrosine kinase (RTK) that activates downstream pathways like Ras/MAPK upon EGF binding.}
}

\newglossaryentry{Endocytosis}{
	name={Endocytosis},
	description={A cellular process by which extracellular materials are internalized via vesicles; in signaling, it helps regulate receptor availability and pathway duration.}
}

\newglossaryentry{epithelialcells}{
	name={Epithelial cells},
	description={Cells that line surfaces and cavities of organs, involved in barrier function, absorption, and signaling processes like lateral inhibition.}
}

\newglossaryentry{Erk}{
	name={Erk},
	description={Extracellular signal-regulated kinase, part of the MAPK signaling pathway downstream of Ras, involved in cell proliferation and differentiation.}
}

\newglossaryentry{FRET}{
	name={FRET},
	description={Short for Förster Resonance Energy Transfer, a technique used to study molecular interactions based on energy transfer between two fluorescent molecules.}
}

\newglossaryentry{FGFR}{
	name={FGFR},
	description={Fibroblast Growth Factor Receptor, a type of RTK that triggers signaling cascades involved in development and cell differentiation.}
}

\newglossaryentry{frizzled}{
	name={Frizzled},
	description={A family of G-protein-coupled receptors that bind Wnt proteins and initiate the Wnt/β-catenin signaling pathway.}
}

\newglossaryentry{Grb2}{
	name={Grb2},
	description={An adaptor protein that links RTKs like EGFR to Ras activation via interaction with SOS, part of the Ras/MAPK pathway.}
}

\newglossaryentry{Groucho}{
	name={Groucho},
	description={A transcriptional co-repressor that inhibits Wnt target gene expression by binding LEF1 in the absence of β-catenin.}
}

\newglossaryentry{GSK3}{
	name={GSK3},
	description={Short for Glycogen Synthase Kinase 3 (often GSK3), involved in phosphorylating β-catenin to promote its degradation in the Wnt pathway.}
}

\newglossaryentry{iHog}{
	name={iHog},
	description={Short for Interference Hedgehog, a co-receptor in the Hedgehog signaling pathway that facilitates ligand reception and signaling.}
}

\newglossaryentry{IKKalpha}{
	name={IKK alpha},
	description={A kinase that is part of the IκB kinase complex; phosphorylates IκB, promoting its degradation and thereby activating NFκB signaling.}
}

\newglossaryentry{IKKbeta}{
	name={IKK beta},
	description={A key catalytic subunit of the IKK complex, required for the canonical NFκB pathway activation through phosphorylation of IκB.}
}

\newglossaryentry{JAK}{
	name={JAK},
	description={Janus Kinase, a family of non-receptor tyrosine kinases associated with cytokine receptors, activating STAT proteins upon phosphorylation.}
}

\newglossaryentry{kinasecytokinereceptors}{
	name={Kinase cytokine receptors},
	description={Receptors that lack intrinsic kinase activity but associate with tyrosine kinases like JAK to transduce signals from extracellular cytokines.}
}

\newglossaryentry{kinaseinsertregion}{
	name={Kinase insert region},
	description={A flexible loop within the kinase domain of some RTKs involved in substrate specificity or regulation of kinase activity.}
}

\newglossaryentry{LEF1}{
	name={LEF1},
	description={Lymphoid enhancer-binding factor 1, a transcription factor activated by β-catenin in the Wnt signaling pathway.}
}

\newglossaryentry{LateralInhibition}{
	name={Lateral Inhibition},
	description={A process during development where a cell inhibits its neighbors from adopting the same fate, often mediated by Notch-Delta signaling.}
}

\newglossaryentry{LRP}{
	name={LRP},
	description={Low-density lipoprotein receptor-related protein, a Wnt co-receptor that cooperates with Frizzled to activate downstream signaling.}
}

\newglossaryentry{MAP}{
	name={MAP},
	description={Short for Mitogen-Activated Protein; involved in cascades such as Ras-MAPK that control gene expression, cell division, and survival.}
}

\newglossaryentry{mTORC1}{
	name={mTORC1},
	description={Mechanistic target of rapamycin complex 1, regulates protein synthesis, metabolism, and cell growth; activated downstream of AKT.}
}

\newglossaryentry{mTORC2}{
	name={mTORC2},
	description={A signaling complex that phosphorylates AKT and regulates the cytoskeleton and survival pathways.}
}

\newglossaryentry{NEMO}{
	name={NEMO},
	description={NFκB essential modulator, a regulatory subunit of the IKK complex that is crucial for NFκB activation.}
}

\newglossaryentry{Nemo}{
	name={Nemo},
	description={Alternative name for NEMO, required for assembling the IKK complex and regulating NFκB signaling.}
}

\newglossaryentry{NFKB}{
	name={NFkB},
	description={A transcription factor that regulates immune and inflammatory responses, activated upon degradation of its inhibitor IκB.}
}

\newglossaryentry{Notch}{
	name={Notch},
	description={A membrane-bound receptor that, upon binding Delta, undergoes proteolytic cleavage releasing the Notch intracellular domain (NICD) to regulate gene transcription.}
}

\newglossaryentry{Notchtail}{
	name={Notch tail},
	description={Also known as NICD (Notch Intracellular Domain), it translocates to the nucleus and associates with Rbpsuh to influence transcription.}
}

\newglossaryentry{Patched}{
	name={Patched},
	description={A receptor in the Hedgehog pathway that inhibits Smoothened in the absence of Hedgehog ligand.}
}

\newglossaryentry{PDK1}{
	name={PDK1},
	description={Phosphoinositide-dependent kinase-1, activates AKT by phosphorylation following PI3K signaling.}
}

\newglossaryentry{Phosphotyrosin}{
	name={Phosphotyrosine},
	description={A phosphorylated tyrosine residue that serves as a binding site for SH2 domain-containing proteins in signaling pathways.}
}

\newglossaryentry{PI3K}{
	name={PI3K},
	description={Phosphoinositide 3-kinase, an enzyme activated by RTKs that produces PIP3, leading to AKT activation.}
}

\newglossaryentry{protooncogene}{
	name={Proto-oncogene},
	description={A normal gene that can become an oncogene through mutation or overexpression, promoting cell proliferation or survival.}
}

\newglossaryentry{PTEN}{
	name={PTEN},
	description={A phosphatase that antagonizes PI3K signaling by dephosphorylating PIP3 to PIP2, acting as a tumor suppressor.}
}

\newglossaryentry{Ras}{
	name={Ras},
	description={A small GTPase that transmits signals from RTKs to MAPK cascades, promoting proliferation and differentiation.}
}

\newglossaryentry{RasMPK}{
	name={Ras MPK},
	description={Refers to the Ras-MAPK pathway, a cascade where Ras activates RAF, MEK, and ERK, leading to gene regulation and cell proliferation.}
}

\newglossaryentry{Rbpsuh}{
	name={Rbpsuh},
	description={Recombination signal-binding protein for immunoglobulin kappa J region, a transcription factor that partners with NICD in Notch signaling.}
}

\newglossaryentry{RTK}{
	name={RTK},
	description={Short for Receptor Tyrosine Kinase, a class of cell-surface receptors that activate intracellular signaling via tyrosine phosphorylation.}
}

\newglossaryentry{Smad}{
	name={Smad},
	description={Intracellular proteins that transmit signals from TGF-beta receptors to the nucleus to regulate transcription.}
}

\newglossaryentry{Smoothened}{
	name={Smoothened},
	description={A transmembrane protein activated in the Hedgehog pathway upon relief of Patched inhibition, triggering downstream signaling.}
}

\newglossaryentry{STAT}{
	name={STAT},
	description={Signal Transducer and Activator of Transcription; phosphorylated by JAKs and then dimerizes to regulate gene expression.}
}

\newglossaryentry{TGFbeta}{
	name={TGF beta},
	description={Transforming Growth Factor beta, a family of cytokines involved in cell proliferation, differentiation, and immune regulation via Smad signaling.}
}

\newglossaryentry{TNF}{
	name={TNF},
	description={Tumor Necrosis Factor, a cytokine involved in systemic inflammation, apoptosis, and immune system signaling through receptors like TNFR.}
}

\newglossaryentry{typeIreceptor}{
	name={Type I receptor},
	description={A component of the TGF-beta receptor complex that is phosphorylated by type II receptors to propagate Smad signaling.}
}

\newglossaryentry{typeIIreceptor}{
	name={Type II receptor},
	description={A receptor that binds TGF-beta ligand and phosphorylates the associated type I receptor to initiate downstream signaling.}
}

\newglossaryentry{Wnt}{
	name={Wnt},
	description={A family of secreted glycoproteins that activate Frizzled receptors and regulate β-catenin stabilization, crucial for development and cell fate.}
}

\newglossaryentry{androgen}{
	name={Androgen},
	description={A group of steroid hormones like testosterone that regulate male traits and reproductive activity.}
}

\newglossaryentry{androgenreceptor}{
	name={Androgen receptor (AR)},
	description={A type of intracellular receptor that binds androgens, then translocates to the nucleus to regulate target gene transcription.}
}

\newglossaryentry{Transphosphorylation}{
	name={Transphosphorylation},
	description={A process in which one kinase phosphorylates another kinase, often occurring during receptor activation, such as with receptor tyrosine kinases (RTKs) where two adjacent receptors phosphorylate each other upon dimerization.}
}

\newglossaryentry{TumorSuppressor}{
	name={Tumor Suppressor},
	description={A gene or protein that prevents uncontrolled cell growth by regulating the cell cycle, promoting apoptosis, or repairing DNA. Loss-of-function mutations in tumor suppressors can contribute to cancer.}
}

\newglossaryentry{ProteolyticCleavage}{
	name={Proteolytic Cleavage},
	description={A biochemical process where specific peptide bonds in a protein are broken by proteases, activating or deactivating signaling molecules or receptors, such as in Notch or Hedgehog signaling pathways.}
}

\newglossaryentry{BetaCatenin}{
	name={Beta Catenin},
	description={A multifunctional protein involved in the Wnt signaling pathway and in cell adhesion. In Wnt signaling, its stabilization leads to nuclear translocation and activation of Wnt target genes.}
}

\newglossaryentry{Axin}{
	name={Axin},
	description={A scaffold protein that forms part of the destruction complex in Wnt signaling. It promotes degradation of beta-catenin in the absence of Wnt signals.}
}

\newglossaryentry{Hedgehog}{
	name={Hedgehog},
	description={A secreted signaling molecule that regulates cell growth and patterning during development. Binding of Hedgehog to the Patched receptor activates Smoothened, initiating downstream signaling through proteins like Cubitus interruptus (Ci).}
}
\newglossaryentry{MEK}{
	name={MEK},
	description={Mitogen-activated protein kinase kinase (MAPKK), a dual-specificity kinase that phosphorylates and activates ERK in the MAPK signaling cascade. MEK acts downstream of Raf and plays a key role in transmitting growth signals from the cell membrane to the nucleus.}
}

\newglossaryentry{Raf}{
	name={Raf},
	description={A serine/threonine-specific protein kinase (MAPKKK) that is activated by Ras in the MAPK pathway. Raf phosphorylates and activates MEK, initiating a kinase cascade involved in cell division and differentiation.}
}

