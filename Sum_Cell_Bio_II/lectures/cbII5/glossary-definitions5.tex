\newglossaryentry{Enzyme}{
	name={Enzyme},
	description={A macromolecular biological catalyst that significantly accelerates chemical reactions, often with high specificity. They can be proteins or catalytically active RNA molecules:contentReference[oaicite:0]{index=0}.}
}

\newglossaryentry{Cofactor}{
	name={Cofactor},
	description={An inorganic chemical component required for enzyme activity. Often metal ions such as Fe\textsuperscript{2+}, Mg\textsuperscript{2+}, or Zn\textsuperscript{2+}:contentReference[oaicite:1]{index=1}.}
}

\newglossaryentry{Coenzyme}{
	name={Coenzyme},
	description={A complex organic or metallorganic molecule required for enzyme function, often derived from vitamins. Acts as a transient carrier of specific functional groups:contentReference[oaicite:2]{index=2}.}
}

\newglossaryentry{Systematic name}{
	name={Systematic name},
	description={A precise name given to an enzyme that indicates the substrates and type of reaction it catalyzes (e.g., ATP:glucose phosphotransferase for hexokinase):contentReference[oaicite:3]{index=3}.}
}

\newglossaryentry{Classification number}{
	name={Classification number},
	description={A four-part code assigned to enzymes based on the type of reaction they catalyze (e.g., 2.7.1.1 for hexokinase):contentReference[oaicite:4]{index=4}.}
}

\newglossaryentry{Activation Energy}{
	name={Activation Energy},
	description={The energy barrier ($$\Delta G^{\ddagger}$$) between the ground state and the transition state that must be overcome for a reaction to proceed:contentReference[oaicite:5]{index=5}.}
}

\newglossaryentry{Transition State Theory}{
	name={Transition State Theory},
	description={A model describing the top of the energy hill in a reaction coordinate where reactants are equally likely to proceed to products or return to reactants. Not to be confused with reaction intermediates:contentReference[oaicite:6]{index=6}.}
}

\newglossaryentry{Transition State}{
	name={Transition State},
	description={The highest-energy configuration along the reaction coordinate. It is the point at which the system is equally likely to proceed toward products or return to reactants. It is not a stable intermediate and is denoted as $\ddagger$:contentReference[oaicite:0]{index=0}.}
}


\newglossaryentry{Reaction Intermediate}{
	name={Reaction Intermediate},
	description={A short-lived, chemically distinct species that occurs during the transformation of reactants to products, such as enzyme-substrate (ES) complexes:contentReference[oaicite:7]{index=7}.}
}

\newglossaryentry{K_{eq}}{
	name={$K_{eq}$},
	description={The equilibrium constant of a reaction, given by the ratio $$\frac{[P]}{[S]}$$. Not affected by enzymes:contentReference[oaicite:8]{index=8}.}
}

\newglossaryentry{Delta G}{
	name={$\Delta G$},
	description={The change in free energy during a reaction. $\Delta G°$ refers to standard conditions, and $\Delta G°^{\prime}$ includes biochemical conditions (pH = 7):contentReference[oaicite:9]{index=9}.}
}

\newglossaryentry{Delta G dagger}{
	name={$\Delta$ $G^{\ddagger}$},
	description={The activation free energy, the difference in energy between the ground state and the transition state:contentReference[oaicite:10]{index=10}.}
}

\newglossaryentry{Return Rate}{
	name={Return Rate},
	description={The rate of the reaction based on substrate concentration and a rate constant $$k$$, where $$v = k[S]$$:contentReference[oaicite:11]{index=11}.}
}

\newglossaryentry{Catalytic Power}{
	name={Catalytic Power},
	description={The ability of enzymes to lower the activation energy and provide alternate reaction pathways, enhancing the reaction rate:contentReference[oaicite:12]{index=12}.}
}

\newglossaryentry{Active Site}{
	name={Active Site},
	description={The region of the enzyme where substrate binding and catalysis occur, often through specific amino acid residues:contentReference[oaicite:13]{index=13}.}
}

\newglossaryentry{Transition State Complementarity}{
	name={Transition State Complementarity},
	description={The enzyme binds more tightly to the transition state than to the substrate, thereby lowering the activation energy and enhancing catalysis:contentReference[oaicite:14]{index=14}.}
}

\newglossaryentry{Rate Enhancement}{
	name={Rate Enhancement},
	description={The increase in reaction speed due to the lowering of the activation energy by enzyme catalysis:contentReference[oaicite:15]{index=15}.}
}

\newglossaryentry{Binding}{
	name={Binding},
	description={The specific interaction between enzyme and substrate, often involving multiple weak noncovalent interactions to drive catalysis:contentReference[oaicite:16]{index=16}.}
}

\newglossaryentry{Specificity}{
	name={Specificity},
	description={The enzyme's ability to select exact substrates and catalyze a specific reaction, influenced by binding energy and structural complementarity:contentReference[oaicite:17]{index=17}.}
}

\newglossaryentry{Acid-Base Catalysis}{
	name={Acid-Base Catalysis},
	description={A catalytic mechanism in which amino acid residues act as proton donors or acceptors to stabilize charged intermediates, increasing reaction rates when water alone is insufficient:contentReference[oaicite:0]{index=0}.}
}

\newglossaryentry{Covalent Catalysis}{
	name={Covalent Catalysis},
	description={A mechanism where the enzyme forms a transient covalent bond with the substrate, creating an alternative reaction pathway with lower activation energy. This requires nucleophilic groups on the enzyme:contentReference[oaicite:1]{index=1}.}
}

\newglossaryentry{Metal Ion Catalysis}{
	name={Metal Ion Catalysis},
	description={Catalysis involving metal ions that can stabilize negative charges on reaction intermediates, help substrate orientation, or mediate redox reactions through reversible changes in oxidation state:contentReference[oaicite:2]{index=2}.}
}

\newglossaryentry{Enzyme Kinetics}{
	name={Enzyme Kinetics},
	description={The study of the rate of enzyme-catalyzed reactions and how they change in response to experimental parameters like substrate concentration:contentReference[oaicite:3]{index=3}.}
}

\newglossaryentry{v_0}{
	name={$v_0$},
	description={The initial velocity of an enzyme-catalyzed reaction, measured at the very beginning before product accumulates. At low $[S]$, $v_0$ increases almost linearly with $[S]$:contentReference[oaicite:4]{index=4}.}
}

\newglossaryentry{Enzyme Saturation}{
	name={Enzyme Saturation},
	description={The condition where increasing substrate concentration no longer increases the reaction rate because all active sites are occupied, resulting in a plateau at $v_{max}$:contentReference[oaicite:5]{index=5}.}
}

\newglossaryentry{v_{max}}{
	name={$v_{max}$},
	description={The maximal velocity of an enzyme-catalyzed reaction when all enzyme molecules are saturated with substrate. Defined by $v_{max} = k_{cat}[E]_T$:contentReference[oaicite:6]{index=6}.}
}

\newglossaryentry{Michaelis-Menten}{
	name={Michaelis-Menten},
	description={A kinetic model describing enzyme activity: $v_0 = \frac{v_{max}[S]}{K_m + [S]}$, where $K_m$ is the substrate concentration at half-maximal velocity:contentReference[oaicite:7]{index=7}.}
}

\newglossaryentry{Lineweaver-Burk}{
	name={Lineweaver-Burk},
	description={A double reciprocal plot of enzyme kinetics: $\frac{1}{v_0} = \frac{K_m}{v_{max}} \cdot \frac{1}{[S]} + \frac{1}{v_{max}}$, used to linearize the Michaelis-Menten equation for easier interpretation:contentReference[oaicite:8]{index=8}.}
}

\newglossaryentry{Enzyme Efficiency}{
	name={Enzyme Efficiency},
	description={Measured by the specificity constant $\frac{k_{cat}}{K_m}$, which reflects both substrate binding and catalytic turnover. The theoretical upper limit is $10^8$ to $10^9 \text{ M}^{-1}\text{s}^{-1}$:contentReference[oaicite:9]{index=9}.}
}

\newglossaryentry{Inhibition}{
	name={Inhibition},
	description={The decrease or complete loss of enzyme activity due to the binding of a molecule (an inhibitor) that interferes with catalysis. Inhibition is fundamental to drug action and metabolic regulation:contentReference[oaicite:0]{index=0}.}
}

\newglossaryentry{Reversible Inhibition}{
	name={Reversible Inhibition},
	description={Inhibition where the inhibitor binds non-covalently and can dissociate from the enzyme. Includes competitive, non-competitive, and uncompetitive mechanisms:contentReference[oaicite:1]{index=1}.}
}

\newglossaryentry{Competitive Inhibitor}{
	name={Competitive Inhibitor},
	description={A molecule that competes with the substrate for binding at the active site. It increases $K_m$ but does not affect $v_{max}$:contentReference[oaicite:2]{index=2}.}
}

\newglossaryentry{Non-competitive Inhibitor}{
	name={Non-competitive Inhibitor},
	description={Binds to a site other than the active site and reduces the effective concentration of active enzyme. It decreases $v_{max}$ without changing $K_m$:contentReference[oaicite:3]{index=3}.}
}

\newglossaryentry{Uncompetitive Inhibitor}{
	name={Uncompetitive Inhibitor},
	description={Binds only to the enzyme-substrate complex, preventing product formation. It decreases both $v_{max}$ and $K_m$:contentReference[oaicite:4]{index=4}.}
}

\newglossaryentry{Irreversible Inhibition}{
	name={Irreversible Inhibition},
	description={Inhibition where the inhibitor covalently binds or permanently inactivates the enzyme, preventing further catalytic activity:contentReference[oaicite:5]{index=5}.}
}

\newglossaryentry{Suicide Inhibition}{
	name={Suicide Inhibition},
	description={A special type of irreversible inhibition where the enzyme converts the inhibitor into a reactive intermediate that covalently modifies and inactivates the enzyme:contentReference[oaicite:6]{index=6}.}
}

\newglossaryentry{Transition-State Analogs}{
	name={Transition-State Analogs},
	description={Stable molecules that resemble the transition state of a substrate and bind tightly to the enzyme, acting as potent inhibitors by exploiting transition state complementarity:contentReference[oaicite:7]{index=7}.}
}

