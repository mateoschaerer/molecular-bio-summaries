\newglossaryentry{actinfilaments}{
	name={actin filaments},
	description={Thin protein filaments that shape the cell surface, enable cell movement, and drive cell division}
}

\newglossaryentry{microtubules}{
	name={microtubule},
	description={Hollow tubes that position organelles, guide intracellular transport, and form the mitotic spindle for chromosome segregation}
}

\newglossaryentry{intermediatefilaments}{
	name={intermediate filaments},
	description={Rope-like fibers that provide mechanical strength and structural support to the cell}
}

\newglossaryentry{motorproteins}{
	name={motor proteins},
	description={ATP-powered proteins that move organelles along cytoskeletal filaments or shift the filaments themselves}
}

\newglossaryentry{microvilli}{
	name={microvilli},
	description={Finger-like cell surface projections supported by actin filaments that increase surface area for absorption}
}

\newglossaryentry{thymosin}{
	name=thymosin,
	description={A small, abundant actin monomer-binding protein that sequesters actin subunits in an inactive form, preventing filament assembly}
}

\newglossaryentry{profilin}{
	name=profilin,
	description={An actin monomer-binding protein that promotes filament assembly by facilitating the addition of actin to the plus end and recycling itself afterward}
}

\newglossaryentry{arpcomplex}{
	name={Arp2/3 complex},
	description={A protein complex that includes two actin-related proteins (Arp2 and Arp3). It nucleates new actin filaments by mimicking the plus end of actin, thus it promotes the formation of branched actin networks}
}

\newglossaryentry{formin}{
	name={formin},
	description={A actin-nucleating protein that promotes the formation of straight, unbranched actin filaments. Formins remain attached to the plus end of the filament during elongation, enabling continuous subunit addition.}
}

\newglossaryentry{gelsolin}{
	name=Gelsolin,
	description={A Ca\textsuperscript{2+}-activated actin-severing protein that binds the side of filaments and caps the newly formed plus ends after cleavage}
}

\newglossaryentry{cofilin}{
	name=Cofilin,
	description={A small actin-binding protein that promotes disassembly by twisting ADP-actin filaments, making them more prone to severing}
}

\newglossaryentry{tropomyosin}{
	name=Tropomyosin,
	description={Protein that binds along actin filaments, stabilizing and blocking interactions with other proteins}
}

\newglossaryentry{CapZ}{
	name=CapZ,
	description={Plus-end capping protein that stabilizes actin filaments by blocking subunit exchange}
}

\newglossaryentry{tropomodulin}{
	name=Tropomodulin,
	description={Minus-end capping protein that regulates actin filament length and stability}
}

\newglossaryentry{tubulin}{
	name={tubulin},
	description={A globular protein that polymerizes to form microtubules}
}

\newglossaryentry{stathmin}{
	name={stathmin},
	description={A microtubule-destabilizing protein that binds to tubulin dimers and prevents their polymerization into microtubules. Stathmin sequesters free $\alpha/\beta$-tubulin heterodimers, reducing the available pool for microtubule assembly. It plays a crucial role in regulating microtubule dynamics, particularly during cell division and migration. Its activity is controlled by phosphorylation.}
}


\newglossaryentry{kinesin13}{
	name={kinesin-13},
	description={It binds to the ends of microtubules and promotes depolymerization (Catastrophe factor), increasing microtubule catastrophe frequency}
}

\newglossaryentry{xmap215}{
	name={XMAP215},
	description={A MAP (Xenopus Microtubule-Associated Protein 215) that functions as a microtubule polymerase. It binds to tubulin dimers and delivers them to the growing plus end of microtubules, thereby promoting microtubule growth}
}

\newglossaryentry{gammaturc}{
	name={$\gamma$-tubulin ring complex},
	sort={gamma-tubulin ring complex},
	description={A multi-protein complex that acts as a nucleation template for microtubule polymerization. Located primarily at the centrosome, the $\gamma$-tubulin ring complex ($\gamma$-TuRC) mimics the microtubule minus end, allowing the rapid and spatially controlled nucleation of microtubules}
}

\newglossaryentry{pcm}{
	name={Pericentriolar Material (PCM)},
	description={A dense, protein-rich matrix that surrounds the centrioles in the centrosome. The PCM anchors $\gamma$-tubulin ring complexes ($\gamma$-TuRCs), which nucleate microtubules}
}

\newglossaryentry{centriole}{
	name={Centriole},
	description={A cylindrical cellular structure composed of nine triplet microtubules arranged with ninefold symmetry. They are found in pairs within the centrosome (mother and daughter)}
}

\newglossaryentry{centrosome}{
	name={Centrosome},
	description={The major microtubule-organizing center (MTOC) in most animal cells}
}


\newglossaryentry{tau}{
	name={Tau},
	description={MAP with a short projecting domain that cross-links microtubules closely together.}
}

\newglossaryentry{MAP2}{
	name={MAP2},
	description={MAP with a long projecting domain, resulting in widely spaced microtubule bundles}
}

\newglossaryentry{SAS-6}{
	name={SAS-6},
	description={Forms coiled-coil dimers that self-assemble into a ninefold symmetrical ring, providing the scaffold for centriole cartwheel formation}
}


\newglossaryentry{kinesin}{
	name={kinesin},
	plural={kinesins},
	description={A class of motor proteins that move along microtubules towards the plus end}
}

\newglossaryentry{dynein}{
	name={dynein},
	plural={dyneins},
	description={A class of motor proteins that move along microtubules towards the minus end}
}

\newglossaryentry{dynactin}{
	name={dynactin},
	description={Dynactin links dynein to its cargo  and enhances dynein's processivity and binding to microtubules.}
}

\newglossaryentry{primarycilium}{
	name={Primary cilium},
	description={A non-motile, microtubule-based organelle present on most vertebrate cells. It functions as a sensory antenna, detecting environmental signals.}
}

\newglossaryentry{keratins}{
	name=Keratins,
	description={A large and diverse family of intermediate filament proteins found mainly in epithelial cells, hair, and nails. They provide structural support and mechanical strength, especially in skin and other tissues under stress}
}

\newglossaryentry{Cdc42}{
	name=Cdc42,
	description={A small GTPase of the Rho family involved in actin cytoskeleton regulation. Its activation promotes the formation of filopodia (thin, finger-like protrusions.)}
}

\newglossaryentry{Rac}{
	name=Rac,
	description={A small Rho-family GTPase that regulates actin dynamics. Activation of Rac leads to the formation of lamellipodia—broad, sheet-like membrane protrusions.}
}

\newglossaryentry{Rho}{
	name=Rho,
	description={A Rho-family GTPase that promotes the formation of stress fibers and focal adhesions. Activated Rho stimulates the formation of thick stress fibers (contractile actin bundles)}
}

\newglossaryentry{axoneme}{
	name=axoneme,
	description={Inside a cilium and a flagellum is a microtubule-based cytoskeleton called the axoneme}
}






