\newglossaryentry{necrosis}{
	name=necrosis,
	description={A form of uncontrolled cell death resulting from acute cellular injury, leading to the rupture of the plasma membrane and inflammation of the surrounding tissue. Cell looks like it exploded}
}

\newglossaryentry{apoptosis}{
	name=apoptosis,
	description={A regulated, energy-dependent form of programmed cell death characterized by cell shrinkage, DNA fragmentation, membrane blebbing, and the absence of inflammation}
}

\newglossaryentry{XIAP}{
	name=XIAP,
	description={X-linked inhibitor of apoptosis protein; a member of the IAP family that binds to and inhibits caspases, particularly caspase-3, -7, and -9, thereby blocking apoptosis}
}

\newglossaryentry{FLIP}{
	name=FLIP,
	description={FLICE-like inhibitory protein; a regulator that inhibits caspase-8 activation at the death-inducing signaling complex (DISC), thereby preventing the initiation of extrinsic apoptosis}
}

\newglossaryentry{initiatorcaspases}{
	name=initiator caspases,
	description={A class of caspases (e.g., caspase-8 and caspase-9) that are activated early in the apoptotic signaling cascade and initiate apoptosis by activating executioner caspases}
}

\newglossaryentry{executionercaspases}{
	name=executioner caspases,
	description={A class of caspases (e.g., caspase-3 and caspase-7) that carry out apoptosis by cleaving a wide range of cellular substrates, leading to controlled cellular dismantling}
}

\newglossaryentry{CAD}{
	name=CAD,
	description={Caspase-Activated DNase; an endonuclease that, upon release from its inhibitor iCAD by executioner caspases, cleaves chromosomal DNA during apoptosis, producing characteristic nucleosome-sized fragments}
}

\newglossaryentry{caspases}{
	name=caspases,
	description={Caspases are a family of proteases that use a cysteine residue in their active site and specifically cleave peptide bonds after aspartic acid residues in substrate proteins}
}

\newglossaryentry{fasreceptor}{
	name=Fas receptor,
	description={A cell-surface death receptor also known as CD95 or APO-1, part of the TNF receptor family. Upon binding its ligand (FasL), it triggers the extrinsic apoptotic signaling pathway}
}

\newglossaryentry{fadd}{
	name=FADD,
	description={Fas-Associated protein with Death Domain; an adaptor protein that binds to death receptors such as Fas. It recruits and activates initiator caspases (e.g., caspase-8), playing a key role in the extrinsic apoptotic pathway}
}

\newglossaryentry{deathsreceptors}{
	name=cell-surface death receptors,
	description={Transmembrane proteins that belong to the TNF (tumor necrosis factor) receptor family and can initiate apoptosis when bound by their ligands. Examples include Fas receptor and TNF receptor}
}

\newglossaryentry{tnf}{
	name=TNF,
	description={Tumor Necrosis Factor; a cytokine involved in systemic inflammation. It can bind to TNF receptors, including death receptors, and promote cell death or survival depending on cellular context}
}

\newglossaryentry{disc}{
	name=DISC,
	description={Death-Inducing Signaling Complex; a multiprotein complex formed upon activation of cell-surface death receptors (such as Fas or TNF receptors). It includes adaptor proteins like FADD and initiator caspases (e.g., caspase-8), and initiates the extrinsic apoptosis pathway}
}

\newglossaryentry{cytochromec}{
	name=Cytochrome~c,
	description={A mitochondrial protein that, when released into the cytosol, helps activate apoptosis by binding Apaf1}
}

\newglossaryentry{apaf1}{
	name=Apaf1,
	description={Apoptotic protease-activating factor 1; binds cytochrome~c and forms the apoptosome to activate caspase-9}
}

\newglossaryentry{apoptosome}{
	name=Apoptosome,
	description={A multiprotein complex formed by Apaf1 and cytochrome~c that activates initiator caspase-9}
}

\newglossaryentry{bcl2family}{
	name=Bcl-2~family,
	description={A group of proteins that regulate mitochondrial outer membrane permeabilization and apoptosis}
}

\newglossaryentry{bax}{
	name=Bax,
	description={A pro-apoptotic Bcl-2 family protein that promotes cytochrome~c release from mitochondria}
}

\newglossaryentry{bak}{
	name=Bak,
	description={A pro-apoptotic Bcl-2 family member that cooperates with Bax to permeabilize the mitochondrial membrane}
}

\newglossaryentry{BH3domain}{
	name={BH3 domain},
	description={A short conserved sequence found in Bcl-2 family proteins. It enables pro-apoptotic proteins to bind to and inhibit anti-apoptotic Bcl-2 proteins, promoting apoptosis}
}


\newglossaryentry{survivalfactor}{
	name={survival factor},
	description={An extracellular signaling molecule that promotes cell survival by inhibiting apoptosis. Survival factors typically act by activating cell-surface receptors, which trigger intracellular pathways that suppress pro-apoptotic proteins}
}

\newglossaryentry{IAP}{
	name={IAP},
	description={Short for \textit{Inhibitor of Apoptosis Protein}. A family of proteins that inhibit caspase activity and thereby block apoptosis. IAPs typically contain BIR (baculovirus IAP repeat) domains and may also promote the degradation of caspases via ubiquitylation}
}

\newglossaryentry{eat-me-signal}{
	name={Eat-me signal},
	description={A molecular marker exposed on the surface of apoptotic cells that signals phagocytes to recognize and engulf them. A common eat-me signal is phosphatidylserine (PS), a phospholipid normally located on the inner leaflet of the plasma membrane but externalized during apoptosis to promote clearance of dying cells.}
}

