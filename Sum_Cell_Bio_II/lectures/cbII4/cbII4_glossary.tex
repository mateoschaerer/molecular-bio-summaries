\newglossaryentry{RNP}{
    name={RNP},
    description={Ribonucleoprotein complex; a molecular complex composed of RNA and proteins that plays key roles in various biological processes such as RNA processing, transport, stability, and translation regulation}
}

\newglossaryentry{P-bodies}{
    name={P-bodies},
    description={Processing bodies; cytoplasmic, membrane-less ribonucleoprotein (RNP) granules involved in mRNA degradation, storage, and translational repression}
}

\newglossaryentry{StressGranules}{
    name={Stress Granules},
    description={Cytoplasmic, membrane-less ribonucleoprotein (RNP) granules that form in response to cellular stress; they store untranslated mRNAs and associated proteins, helping regulate translation and protect RNA during stress conditions}
}

\newglossaryentry{EndoplasmicReticulum}{
    name={Endoplasmic Reticulum (ER)},
    description={A membrane-bound organelle in eukaryotic cells involved in protein and lipid synthesis, calcium storage, and detoxification. It exists in two forms: rough ER (with ribosomes, involved in protein synthesis) and smooth ER (lacking ribosomes, involved in lipid metabolism and detoxification)}
}

\newglossaryentry{BetaOxidation}{
    name={\(\beta\)-oxidation},
    description={A metabolic process that breaks down fatty acids in the mitochondria (and peroxisomes) to generate acetyl-CoA, NADH, and FADH\(_2\), which are used in energy production via the citric acid cycle and oxidative phosphorylation}
}

\newglossaryentry{translocators}{
    name=translocators,
    description={Proteins or protein complexes that transport molecules across biological membranes, such as the cell membrane or organelle membranes. They play a critical role in processes like nutrient uptake, waste export, and protein trafficking}
}

\newglossaryentry{NLS}{
    name={NLS},
    description={Nuclear Localization Signal, a short amino acid sequence that directs the transport of a protein into the nucleus of a cell},
    sort=NLS
}

\newglossaryentry{NPC}{
    name={NPC},
    description={Nuclear Pore Complex, a large protein assembly embedded in the nuclear envelope that regulates the transport of molecules between the nucleus and cytoplasm},
    sort=NPC
}

\newglossaryentry{importin}{
    name={importin},
    description={A transport protein that binds to nuclear localization signals (NLS) on cargo proteins and mediates their transport into the nucleus through the nuclear pore complex (NPC)},
    sort=importin
}

\newglossaryentry{exportin}{
    name={Exportin},
    description={A transport receptor that binds cargo proteins with nuclear export signals (NES) and facilitates their export from the nucleus to the cytosol in a Ran-GTP-dependent manner.}
}

\newglossaryentry{ranGTP}{
    name={Ran-GTP},
    description={The active, GTP-bound form of the Ran protein, primarily found in the nucleus, which facilitates nuclear export by binding to export receptors and releasing import receptors.}
}

\newglossaryentry{ranGDP}{
    name={Ran-GDP},
    description={The inactive, GDP-bound form of the Ran protein, primarily found in the cytosol, which results from the hydrolysis of Ran-GTP and is necessary for recycling import/export receptors.}
}

\newglossaryentry{gef}{
    name={Ran-GEF (Guanine nucleotide exchange factor)},
    description={A nuclear protein that facilitates the exchange of GDP for GTP on Ran, maintaining the high concentration of Ran-GTP in the nucleus.}
}

\newglossaryentry{gap}{
    name={Ran-GAP (GTPase-activating protein)},
    description={A cytosolic protein that stimulates the GTP hydrolysis activity of Ran, converting Ran-GTP to Ran-GDP. This process ensures a high concentration of Ran-GDP in the cytosol, maintaining the Ran-GTP gradient necessary for nuclear transport.}
}

\newglossaryentry{calcineurin}{
    name={Calcineurin},
    description={A calcium/calmodulin-dependent serine/threonine phosphatase that plays a key role in signal transduction. It dephosphorylates nuclear factor of activated T cells (NFAT), enabling its nuclear translocation and regulating immune responses, as well as other cellular processes such as nuclear transport and synaptic plasticity.}
}


\newglossaryentry{cotranslocation}{
    name={co-translational translocation},
    description={A process in which a nascent protein is simultaneously synthesized and translocated into the endoplasmic reticulum (ER) through the Sec61 translocon complex. This occurs co-translationally, meaning the translation of the protein occurs in parallel with its translocation into the ER.}
}

\newglossaryentry{postrtranslocation}{
    name={post-translational translocation},
    description={The process where a protein is synthesized and fully translated in the cytoplasm before being translocated into the endoplasmic reticulum (ER). Unlike co-translocation, post-translocation involves the completion of translation before the protein enters the ER.}
}

\newglossaryentry{translocator}{
    name={translocator (Sec61 complex)},
    description={A protein translocator complex embedded in the membrane of the endoplasmic reticulum (ER). It forms a channel through which nascent polypeptides are co-translationally or post-translationally translocated into the ER lumen or integrated into the membrane. The complex consists of three core subunits: Sec61α, Sec61β, and Sec61γ.}
}

