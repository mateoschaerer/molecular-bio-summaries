\newglossaryentry{electrochemicalgradient}{
	name={electrochemical gradient},
	description={A gradient formed by the combined effect of an ion's concentration gradient and the electrical potential across a membrane. It determines the direction and force with which ions move across biological membranes},
	sort=electrochemicalgradient
}
\newglossaryentry{uniporter}{
	name=Uniporter,
	description={Transporter that facilitates the movement of a single type of molecule or ion across the membrane down its concentration gradient (passive transport), without coupling to the movement of any other substance}
}

\newglossaryentry{symporter}{
	name=Symporter,
	description={A cotransporter that moves two (or more) different substances across a membrane in the same direction. One substance typically moves down its electrochemical gradient, providing the energy to transport the other substance against its gradient}
}

\newglossaryentry{antiporter}{
	name=Antiporter,
	description={A cotransporter that exchanges two (or more) substances across a membrane in opposite directions. Movement of one substance down its electrochemical gradient powers the transport of another substance against its gradient}
}

\newglossaryentry{primaryactivetransport}{
	name=Primary active transport,
	description={A form of active transport that directly uses energy, usually from the hydrolysis of ATP, to move molecules or ions against their electrochemical gradient across a membrane. Example: the Na\textsuperscript{+}/K\textsuperscript{+}-ATPase pump}
}

\newglossaryentry{secondaryactivetransport}{
	name=Secondary active transport,
	description={A type of active transport that does not use ATP directly. Instead, it relies on the electrochemical gradient established by primary active transport to drive the movement of other substances against their gradients. Can be in the form of symport or antiport}
}

\newglossaryentry{SGLT}{
	name=SGLT family,
	description={Sodium-Glucose Linked Transporters are secondary active transporters that couple the uptake of glucose with the inward movement of Na\textsuperscript{+} ions, enabling glucose absorption against its concentration gradient. Prominent in intestinal and renal epithelial cells}
}

\newglossaryentry{osmolarity}{
	name=Osmolarity,
	description={The total concentration of solute particles in a solution, measured in osmoles per liter (Osm/L). It governs water movement across membranes and is regulated in cells in part by ion pumps like the Na\textsuperscript{+}/K\textsuperscript{+} pump}
}

\newglossaryentry{isotonic}{
	name=Isotonic,
	description={Describes a solution that has the same osmolarity as the inside of a cell, resulting in no net movement of water across the cell membrane}
}

\newglossaryentry{hypertonic}{
	name=Hypertonic,
	description={Describes a solution with higher osmolarity than the cell interior, causing water to leave the cell and leading to cell shrinkage}
}

\newglossaryentry{hypotonic}{
	name=Hypotonic,
	description={Describes a solution with lower osmolarity than the cell interior, causing water to enter the cell and potentially leading to cell swelling or lysis}
}

\newglossaryentry{sarcoplasmicreticulum}{
	name={sarcoplasmic reticulum (SR)}, 
	description={The SR is a specialized type of endoplasmic reticulum that forms a network of tubular sacs in the muscle cell cytoplasm, and it serves as an intracellular store of Ca\textsuperscript{2+}.}
}


\newglossaryentry{multidrugresistance}{
	name=Multidrug resistance (MDR),
	description={A phenomenon where cells become resistant to a wide range of structurally unrelated drugs, often due to the activity of ATP-binding cassette transporters that actively export toxic substances and therapeutic drugs out of the cell, reducing their intracellular concentrations and effectiveness}
}

\newglossaryentry{vasopressin}{
	name=Vasopressin,
	description={Also known as antidiuretic hormone (ADH); a peptide hormone released by the posterior pituitary in response to dehydration or increased plasma osmolarity. It promotes water reabsorption in the kidneys by stimulating the insertion of aquaporin-2 channels in the collecting ducts, thereby reducing urine output and conserving body water}
}

\newglossaryentry{patchclamp}{
	name={patch-clamp},
	description={A laboratory technique in electrophysiology used to study ionic currents in individual isolated living cells, tissues, or patches of cell membrane by measuring the current through a small patch of membrane}
}