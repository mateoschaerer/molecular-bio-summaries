\newglossaryentry{anabolic}{
    name=anabolic,
    description={Relating to or promoting anabolism, the set of metabolic pathways that construct molecules from smaller units, typically requiring energy}
}


\newglossaryentry{phosphoenolpyruvate}{
    name=phosphoenolpyruvate (PEP),
    description={A high-energy intermediate in glycolysis that donates a phosphate group to ADP to form ATP and pyruvate; also involved in gluconeogenesis and other metabolic pathways},
    sort=phosphoenolpyruvate
}

\newglossaryentry{pyruvatecarboxylase}{
    name=pyruvate carboxylase,
    description={A mitochondrial enzyme that catalyzes the carboxylation of pyruvate to form oxaloacetate, playing a key role in gluconeogenesis and anaplerotic reactions}
}

\newglossaryentry{PEPcarboxykinase}{
    name=PEP carboxykinase,
    description={An enzyme that catalyzes the conversion of oxaloacetate to phosphoenolpyruvate (PEP); exists in both mitochondrial and cytosolic forms and is important in gluconeogenesis},
    sort=PEP carboxykinase
}

\newglossaryentry{malatedehydrogenase}{
    name=malate dehydrogenase,
    description={An enzyme that catalyzes the reversible conversion between malate and oxaloacetate, present in both the mitochondria and cytosol},
    sort=malate dehydrogenase
}


\newglossaryentry{acetylcoa}{
    name=Acetyl-CoA,
    description={A central metabolic intermediate formed from the breakdown of carbohydrates, fats, and proteins. It delivers the acetyl group to the citric acid cycle (Krebs cycle) for energy production and also acts as an allosteric cofactor that activates pyruvate carboxylase. Acetyl-CoA is an allosteric inhibitor of pyruvate dehydrogenase complex, while it activates pyruvate carboxylase.
    sort=acetylcoa
    }
}

\newglossaryentry{glc6ptransporter}{
    name=glucose-6-phosphate transporter (T1),
    description={A transport protein that shuttles glucose-6-phosphate (Glc(6)P) from the cytosol into the lumen of the endoplasmic reticulum (ER)},
    sort=glucose6phosphatetransporter
}

\newglossaryentry{glc6ptase}{
    name=glucose-6-phosphatase (Glc(6)Ptase),
    description={An ER-membrane-bound enzyme that hydrolyzes glucose-6-phosphate into glucose and inorganic phosphate (Pi)},
    sort=glucose6phosphatase
}

\newglossaryentry{sp}{
    name=stabilizing protein (SP),
    description={A calcium-binding protein that stabilizes the glucose-6-phosphatase complex in the ER lumen},
    sort=stabilizingprotein
}

\newglossaryentry{phosphatetransporter}{
    name=phosphate transporter (T2),
    description={A transporter that exports inorganic phosphate (Pi) from the ER lumen to the cytosol after glucose-6-phosphate hydrolysis},
    sort=phosphatetransporter
}

\newglossaryentry{glucosetransporter}{
    name=glucose transporter (T3),
    description={A transporter that moves glucose from the ER lumen to the cytosol following its generation from glucose-6-phosphate},
    sort=glucosetransporter
}
\newglossaryentry{pfk1}{
    name=phosphofructokinase-1 (PFK-1),
    description={A key regulatory enzyme in glycolysis that catalyzes the conversion of fructose-6-phosphate to fructose-1,6-bisphosphate using ATP; activated by AMP and fructose-2,6-bisphosphate, and inhibited by ATP and citrate},
    sort=pfk1
}

\newglossaryentry{fbpase1}{
    name={fructose-16-bisphosphatase-1 (FBPase-1)},
    description={A key enzyme in gluconeogenesis that catalyzes the hydrolysis of fructose-1,6-bisphosphate to fructose-6-phosphate, bypassing the irreversible step of phosphofructokinase-1 (PFK-1) in glycolysis; inhibited by fructose-2,6-bisphosphate and AMP},
    sort=fbpase1
}

\newglossaryentry{pfk2domain}{
    name=PFK-2 domain,
    description={The kinase domain of the bifunctional enzyme PFK-2/FBPase-2. It catalyzes the phosphorylation of fructose-6-phosphate to fructose-2,6-bisphosphate, which activates PFK-1 and stimulates glycolysis. Its activity is enhanced by insulin signaling},
    sort=pfk2domain
}


\newglossaryentry{fbpase2domain}{
    name=FBPase-2 domain,
    description={The phosphatase domain of the bifunctional enzyme PFK-2/FBPase-2. It hydrolyzes fructose-2,6-bisphosphate back to fructose-6-phosphate, reducing glycolytic flux and promoting gluconeogenesis. Its activity is stimulated by glucagon signaling},
    sort=fbpase2domain
}

\newglossaryentry{udp-glucose-pyrophosphorylase}{
    name={UDP-glucose pyrophosphorylase},
    description={An enzyme that catalyzes the formation of UDP-glucose from glucose-1-phosphate and UTP. UDP-glucose is a key precursor for glycogen and polysaccharide biosynthesis in many organisms},
    sort=UDPglucosepyrophosphorylase
}

\newglossaryentry{glycogenin}{
    name={glycogenin},
    description={A self-glucosylating glycosyltransferase enzyme that initiates glycogen synthesis by catalyzing the attachment of glucose residues to a specific tyrosine residue on itself, forming the primer required for glycogen synthase to elongate the glycogen chain},
    sort=glycogenin
}

\newglossaryentry{glycogen-synthase}{
    name={glycogen synthase},
    description={The key enzyme responsible for elongating the glycogen chain by adding glucose residues from UDP-glucose to the non-reducing ends of an existing primer via \(\alpha(1\rightarrow4)\) glycosidic bonds},
    sort=glycogensynthase
}

\newglossaryentry{glycogen-branching-enzyme}{
    name={glycogen branching enzyme},
    description={An enzyme that introduces \(\alpha(1\rightarrow6)\) glycosidic branches into the linear glycogen chain, increasing its solubility and accessibility for rapid synthesis and degradation},
    sort=glycogenbranchingenzyme
}

\newglossaryentry{phosphoglucomutase}{
    name={phosphoglucomutase},
    description={An enzyme that catalyzes the reversible conversion of glucose-1-phosphate to glucose-6-phosphate, a key step in both glycogen synthesis and glycogenolysis},
    sort=phosphoglucomutase
}









