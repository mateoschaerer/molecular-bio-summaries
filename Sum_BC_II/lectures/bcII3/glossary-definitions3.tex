\newglossaryentry{anabolism}{
	name={Anabolism},
	description={Metabolic pathways that build complex molecules from simpler ones, requiring an input of energy.}
}

\newglossaryentry{atp}{
	name={ATP (Adenosine Triphosphate)},
	description={The primary energy currency of the cell, used to power many cellular processes. Its hydrolysis releases heat but not energy, while the transfer of phosphate leads to a higher state in free energy of the substrate.}
}

\newglossaryentry{catabolism}{
	name={Catabolism},
	description={Metabolic pathways that break down complex molecules into simpler ones, releasing energy.}
}

\newglossaryentry{coupledreactions}{
	name={Coupled Reactions},
	description={Two chemical reactions linked together, where an energetically favourable reaction (e.g., ATP hydrolysis) provides the energy to drive an energetically unfavourable reaction.}
}

\newglossaryentry{electroncarrier}{
	name={Electron Carrier},
	description={Molecules that can accept and donate electrons, facilitating the transfer of energy in redox reactions (e.g., NAD+, FAD).}
}

\newglossaryentry{glycolysis}{
	name={Glycolysis},
	description={A universal metabolic pathway that breaks down glucose into pyruvate, producing a net gain of ATP and NADH.}
}

\newglossaryentry{metabolism}{
	name={Metabolism},
	description={The sum of all chemical reactions that occur within a living organism to maintain life.}
}

\newglossaryentry{oxidation}{
	name={Oxidation},
	description={The loss of electrons or an increase in oxidation state of a molecule. Often involves the addition of oxygen or removal of hydrogen.}
}

\newglossaryentry{phosphorylation}{
	name={Phosphorylation},
	description={The addition of a phosphate group to a molecule, often increasing its energy or altering its activity.}
}


\newglossaryentry{reduction}{
	name={Reduction},
	description={The gain of electrons or a decrease in oxidation state of a molecule. Often involves the addition of hydrogen or removal of oxygen.}
}

\newglossaryentry{secondlaw}{
	name={Second Law of Thermodynamics},
	description={States that the total entropy of an isolated system can never decrease over time.}
}

\newglossaryentry{enthalpy}{
	name={Enthalpy (H)},
	description={A thermodynamic quantity representing the total heat content of a system. In biochemistry, enthalpy changes (\(\Delta H\)) are associated with bond formation and breaking, influencing biochemical reactions and energy transfer.}
}

\newglossaryentry{entropy}{
	name={Entropy},
	description={A measure of the disorder or randomness of a system. The second law of thermodynamics states that the total entropy of an isolated system tends to increase over time.}
}

\newglossaryentry{exergonicreaction}{
	name={Exergonic Reaction},
	description={A chemical reaction that releases energy (has a negative \( \Delta G \)) and is therefore favorable. }
}

\newglossaryentry{freeenergy}{
	name={Free Energy (G)},
	description={The portion of a system's energy that is available to do useful work.}
}


\newglossaryentry{kinase}{
	name={Kinase},
	description={An enzyme that catalyzes the transfer of phosphate groups from high-energy donor molecules, such as ATP, to specific substrates, a process known as phosphorylation.}
}

\newglossaryentry{phosphatase}{
	name={Phosphatase},
	description={An enzyme that removes phosphate groups from proteins or other molecules, a process known as dephosphorylation, which often regulates cellular activity.}
}

\newglossaryentry{dehydrogenase}{
	name={Dehydrogenase},
	description={An enzyme that catalyzes the removal of two hydrogen atoms from a substrate, typically transferring them to an electron acceptor such as NAD$^+$ or FAD. Dehydrogenases play a crucial role in metabolic pathways like glycolysis and the citric acid cycle.}
}

\newglossaryentry{oxidase}{
	name={Oxidase},
	description={An enzyme that catalyzes oxidation reactions, using molecular oxygen (O$_2$) as the electron acceptor without incorporating it into the substrate. Oxidases are involved in various biological oxidation processes, including those in the electron transport chain.}
}

\newglossaryentry{oxygenase}{
	name={Oxygenase},
	description={An enzyme that catalyzes the incorporation of oxygen atoms from molecular oxygen (O$_2$) into a substrate. Oxygenases are classified into monooxygenases and dioxygenases, which incorporate one or two oxygen atoms, respectively, and are essential in metabolic pathways like drug metabolism and biosynthesis.}
}

\newglossaryentry{reducingequivalent}{
	name={Reducing Equivalent},
	description={A unit of reducing power in biochemical redox reactions, referring to the transfer of one electron (or its equivalent as a hydrogen atom or hydride ion). Reducing equivalents are carried by molecules such as NADH, NADPH, and FADH$_2$, playing a crucial role in cellular respiration and biosynthetic pathways.}
}

\newglossaryentry{niacin}{
	name={Niacin (Vitamin B3)},
	description={A water-soluble B vitamin essential for energy metabolism, DNA repair, and cell signaling. Niacin is a precursor to the coenzymes NAD$^+$ and NADP$^+$, which are crucial for redox reactions in cellular respiration and biosynthetic pathways}
}

\newglossaryentry{arsenatepoisoning}{
	name={Arsenate Poisoning},
	description={A toxic condition caused by exposure to arsenate (AsO$_4^{3-}$), which disrupts cellular metabolism by mimicking phosphate. Arsenate can uncouple oxidative phosphorylation by substituting for inorganic phosphate glycolisys or ATP synthesis, leading to decreased ATP production and cellular toxicity. Symptoms include nausea, vomiting, neurological disturbances, and multi-organ failure in severe cases.}
}

\newglossaryentry{nad}{
	name={NAD$^+$ (Nicotinamide Adenine Dinucleotide)},
	description={A coenzyme involved in redox reactions, serving as an electron carrier in cellular respiration. NAD$^+$ is reduced to NADH, which donates electrons to the electron transport chain for ATP production.}
}

\newglossaryentry{nadp}{
	name={NADP$^+$ (Nicotinamide Adenine Dinucleotide Phosphate)},
	description={A phosphorylated form of NAD$^+$ that functions as an electron carrier, primarily in anabolic pathways such as fatty acid and nucleotide biosynthesis. NADP$^+$ is reduced to NADPH, which provides reducing power for biosynthetic reactions.}
}

\newglossaryentry{fad}{
	name={FAD (Flavin Adenine Dinucleotide)},
	description={A redox-active coenzyme associated with various enzymes, particularly in the electron transport chain and fatty acid oxidation. FAD is reduced to FADH$_2$, which donates electrons to the respiratory chain.}
}

\newglossaryentry{fmn}{
	name={FMN (Flavin Mononucleotide)},
	description={A coenzyme derived from riboflavin (vitamin B2) that acts as a prosthetic group in various oxidoreductases, including NADH dehydrogenase in the electron transport chain. FMN is involved in redox reactions, cycling between oxidized and reduced states.}
}


\newglossaryentry{flavoprotein}{
	name={Flavoprotein},
	description={A protein that contains a flavin coenzyme, such as FAD or FMN, as a prosthetic group}
}

\newglossaryentry{ubiquinone}{
	name={Ubiquinone (Coenzyme Q)},
	description={A lipid-soluble electron carrier in the electron transport chain, transferring electrons between complex I/II and complex III. Ubiquinone exists in oxidized (Q), semiquinone (Q$^-$), and reduced (QH$_2$) forms.}
}

