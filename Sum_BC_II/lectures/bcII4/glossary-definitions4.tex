\newglossaryentry{hexokinase}{
	name={Hexokinase},
	description={An enzyme that catalyzes the phosphorylation of glucose to glucose-6-phosphate, the first step in glycolysis.}
}

\newglossaryentry{pyruvate}{
	name={Pyruvate},
	description={A three-carbon molecule that is the end product of glycolysis.}
}

\newglossaryentry{fanaerorg}{
	name={Facultative Anaerobic Organism},
	description={A organism that is able to produce ATP by anerobic respiration if oxygen is present, but is also capable of switching to fermentation if oxygen is absent}
}

\newglossaryentry{pdh}{
	name={pyruvate dehydrogenase complex (PDH complex or PDC)},
	description={a multi-enzyme complex that catalyzes the conversion of pyruvate into acetyl-CoA, linking glycolysis to the citric acid cycle},
	sort=pyruvatedehydrogenase
}

\newglossaryentry{pyruvatetranslocase}{
	name={pyruvate translocase},
	description={a transport protein located in the inner mitochondrial membrane that facilitates the import of pyruvate from the cytosol into the mitochondrial matrix for further metabolic processing},
	sort=pyruvatetranslocase
}

\newglossaryentry{lipoyllysine}{
	name={lipoyllysine},
	description={a covalent complex of lipoic acid attached via an amide bond to the $\epsilon$-amino group of a lysine residue. It serves as a swinging arm in multi-enzyme complexes like the pyruvate dehydrogenase complex, transferring reaction intermediates between active sites. It has \textbf{two thiol groups} that can undergo reversible oxidation to a disulfid bond.},
	sort=lipoyllysine
}

\newglossaryentry{tpp}{
	name={thiamine pyrophosphate (TPP)},
	description={a coenzyme derived from vitamin B1, essential in decarboxylation reactions such as those in the pyruvate dehydrogenase complex. It stabilizes carbanion intermediates via its thiazolium ring},
	sort=thiaminepyrophosphate
}

\newglossaryentry{succinatedehydrogenase}{
	name={succinate dehydrogenase},
	description={an enzyme that catalyzes the oxidation of succinate to fumarate in the citric acid cycle (step 6). It is also part of Complex II in the electron transport chain, linking the TCA cycle with oxidative phosphorylation by transferring electrons from FADH2 to ubiquinone (coenzyme Q)},
	sort=succinatedehydrogenase
}

\newglossaryentry{bileacids}{
	name={bile acids},
	description={amphipathic molecules derived from cholesterol that aid in the digestion and absorption of dietary fats by emulsifying lipids and facilitating micelle formation. They are synthesized in the liver, stored in the gallbladder, and released into the small intestine. Bile acids are also involved in cholesterol excretion and undergo enterohepatic circulation},
	sort=bileacids
}

\newglossaryentry{carnitine}{
	name={Carnitine},
	description={ammonium compound that plays a crucial role in the transport of long-chain fatty acids into the mitochondrial matrix for $\beta$-oxidation. Carnitine enabling them to cross the inner mitochondrial membrane via the carnitine shuttle system},
	sort=carnitine
}


\newglossaryentry{plp}{
	name={pyridoxal phosphate (PLP)},
	description={A cofactor derived from vitamin B\textsubscript{6}, essential for amino acid metabolism. PLP is involved in transamination, decarboxylation, and deamination reactions}
}

\newglossaryentry{cytochromec}{
	name={Cytochrome c},
	description={A small heme protein located in the intermembrane space of mitochondria. It plays a key role in the electron transport chain by transferring electrons between Complex III (cytochrome bc\textsubscript{1} complex) and Complex IV (cytochrome c oxidase)}
}
