\newglossaryentry{malonylcoa}{
    name={Malonyl-CoA},
    description={A key intermediate in fatty acid synthesis, formed by the carboxylation of acetyl-CoA via the enzyme acetyl-CoA carboxylase. Malonyl-CoA serves as the two-carbon donor in chain elongation steps catalyzed by fatty acid synthase (FAS)},
    sort=malonylcoa
}

\newglossaryentry{acetylcoacarboxylase}{
    name={Acetyl-CoA Carboxylase (ACC)},
    description={The rate-limiting enzyme in fatty acid synthesis. It catalyzes the carboxylation of acetyl-CoA to form malonyl-CoA, using biotin as a cofactor. ACC is regulated by phosphorylation, allosteric effectors (e.g., citrate activation, palmitoyl-CoA inhibition), and hormonal signals such as insulin and glucagon},
    sort=acetylcoacarboxylase
}

\newglossaryentry{citratesynthase}{
    name={Citrate Synthase},
    description={A key enzyme of the citric acid (TCA) cycle that catalyzes the condensation of acetyl-CoA and oxaloacetate to form citrate. This reaction initiates the TCA cycle and also provides citrate for export to the cytosol, where it can be used for fatty acid synthesis after conversion back to acetyl-CoA},
    sort=citratesynthase
}

\newglossaryentry{citrate_transporter}{
    name={TCA transporter (Citrate Transporter)},
    description={A mitochondrial membrane transporter that exports citrate from the mitochondrial matrix to the cytosol in exchange for malate. This shuttle is essential for transferring acetyl-CoA equivalents into the cytosol, where citrate is cleaved by ATP-citrate lyase to regenerate acetyl-CoA for fatty acid synthesis},
    sort=citratetransporter
}



\newglossaryentry{malicenzyme}{
    name={Malic Enzyme (ME)},
    description={An enzyme that catalyzes the oxidative decarboxylation of malate to pyruvate, producing NADPH in the cytosol. This NADPH is essential for the reductive steps of fatty acid synthesis. Malic enzyme links the citrate-malate shuttle with the cell's need for reducing power},
    sort=malicenzyme
}

\newglossaryentry{FAsynthase}{
    name={Fatty Acid Synthase},
    description={A multifunctional enzyme complex that catalyzes the de novo synthesis of long-chain saturated fatty acids from acetyl-CoA and malonyl-CoA},
    text={FAS},
    first={Fatty Acid Synthase (FAS)}
}

\newglossaryentry{MalonylAcetyltransferase}{
    name={Malonyl/Acetyltransferase domain (MAT)},
    description={Transfers malonyl and acetyl groups to the acyl carrier protein (ACP), initiating fatty acid synthesis},
    text={MAT domain},
    first={Malonyl/Acetyltransferase (MAT) domain}
}

\newglossaryentry{AcylCarrierProtein}{
    name={Acyl Carrier Protein domain (ACP)},
    description={A flexible domain that shuttles the growing fatty acid chain between catalytic sites during synthesis},
    text={ACP domain},
    first={Acyl Carrier Protein (ACP) domain}
}

\newglossaryentry{BetaKetoacylSynthase}{
    name={$\beta$-Ketoacyl Synthase domain (KS)},
    description={Catalyzes the condensation of acyl and malonyl groups to form $\beta$-ketoacyl intermediates},
    text={KS domain},
    first={$\beta$-Ketoacyl Synthase (KS) domain}
}

\newglossaryentry{BetaKetoacylReductase}{
    name={$\beta$-Ketoacyl Reductase domain (KR)},
    description={Reduces $\beta$-ketoacyl intermediates to $\beta$-hydroxyacyl derivatives using NADPH},
    text={KR Domain},
    first={$\beta$-Ketoacyl Reductase (KR) domain}
}

\newglossaryentry{Dehydratase}{
    name={$\beta$-hydroxyacyl-ACP 
dehydratase(DH)},
    description={Dehydrates $\beta$-hydroxyacyl intermediates to form enoyl derivatives},
    text={DH domain},
    first={$\beta$-hydroxyacyl-ACP 
dehydratase(DH)}
}

\newglossaryentry{EnoylReductase}{
    name={Enoyl Reductase domain (ER)},
    description={Reduces enoyl intermediates to saturated acyl chains using NADPH},
    text={Enoyl Reductase domain},
    first={enoyl-ACP reductase (ER) domain}
}

\newglossaryentry{Thioesterase}{
    name={Thioesterase domain (TE)},
    description={Hydrolyzes and releases the final fatty acid product from the acyl carrier protein},
    text={TE domain},
    first={Thioesterase (TE) domain}
}

\newglossaryentry{acetoacetyl}{
  name={acetoacetyl},
  description={A functional group derived from acetoacetic acid. It is intermediate of FA sythesis}
}

\newglossaryentry{bhbcoa}{
  name={\ensuremath{\beta}-hydroxybutyryl-CoA},
  description={A coenzyme A (CoA) thioester of $\beta$-hydroxybutyric acid. It is an intermediate in FA sythesis}
}

\newglossaryentry{transD2ButenoylACP}{
  name={\textit{trans}-D\textsuperscript{2}-butenoyl-ACP},
  description={intermediate of FA sythesis}
}

\newglossaryentry{butyrylACP}{
  name={butyryl-ACP},
  description={A saturated four-carbon acyl carrier protein (ACP) thioester intermediate in fatty acid biosynthesis.}
}


\newglossaryentry{fattyAcylCoADesaturase}{
  name={fatty acyl-CoA desaturase},
  description={An enzyme that catalyzes the insertion of a cis double bond into a saturated fatty acyl-CoA molecule, converting it into an unsaturated fatty acid.}
}

\newglossaryentry{glycerol3phosphateDehydrogenase}{
  name={glycerol-3-phosphate dehydrogenase},
  description={An enzyme that catalyzes the reversible redox conversion between dihydroxyacetone phosphate (DHAP) and glycerol-3-phosphate.}
}

\newglossaryentry{glycerolKinase}{
  name={glycerol kinase},
  description={An enzyme that catalyzes the phosphorylation of glycerol to glycerol-3-phosphate using ATP. Used in the FA sythesis pathway}
}

\newglossaryentry{acylCoASynthetase}{
  name={acyl-CoA synthase},
  description={An enzyme that activates free fatty acids by catalyzing their conversion into fatty acyl-CoA using ATP and coenzyme A. This activation is required for fatty acid metabolism, including $\beta$-oxidation and lipid biosynthesis.}
}

\newglossaryentry{acylTransferases}{
  name={acyl transferases},
  description={A class of enzymes that catalyze the transfer of acyl groups from one molecule to another. They play essential roles in lipid biosynthesis.}
}

\newglossaryentry{phosphatidicacid}{
    name={phosphatidic acid (PA)},
    description={A key intermediate in lipid metabolism, composed of a glycerol backbone with two fatty acid chains and a phosphate group; precursor for both triacylglycerol and phospholipid synthesis},
    text={PA},
    first={phosphatidic acid (PA)}
}


\newglossaryentry{phosphatidicacidphosphatase}{
    name={phosphatidic acid phosphatase (PAP)},
    description={An enzyme that catalyzes the dephosphorylation of phosphatidic acid to form diacylglycerol (DAG), a key step in lipid biosynthesis},
    text={PAP},
    first={phosphatidic acid phosphatase (PAP)}
}


\newglossaryentry{phosphatidylinositol_synthase}{
    name=phosphatidylinositol synthase,
    description={An enzyme responsible for catalyzing the synthesis of phosphatidylinositol, a phospholipid involved in cellular signaling and membrane structure.}
}


\newglossaryentry{choline_kinase}{
    name=choline kinase,
    description={An enzyme that catalyzes the phosphorylation of choline to form phosphocholine, a key step in the biosynthesis of phosphatidylcholine, an essential phospholipid in cell membranes}
}

\newglossaryentry{ctp_choline_cytidylyltransferase}{
    name=CTP\textendash choline cytidylyltransferase,
    description={A regulatory enzyme in the CDP-choline pathway that catalyzes the conversion of phosphocholine and CTP to CDP-choline, a precursor for phosphatidylcholine synthesis}
}
\newglossaryentry{cdp_choline_dag_phosphocholine_transferase}{
    name=CDP-choline diacylglycerol phosphocholine transferase,
    description={An enzyme that catalyzes the final step in the CDP-choline pathway, transferring the phosphocholine group from CDP-choline to diacylglycerol to form phosphatidylcholine}
}

\newglossaryentry{thiolase}{
    name=thiolase,
    description={An enzyme that catalyzes the condensation or cleavage of acetyl-CoA molecules, playing a key role in cholesterol sythesis}
}

\newglossaryentry{hmg_coa_synthase}{
    name=HMG\textendash CoA synthase,
    description={An enzyme that catalyzes the condensation of acetoacetyl-CoA with acetyl-CoA to form $\beta$-hydroxy-$\beta$-methylglutaryl-CoA (HMG-CoA), a key intermediate in the mevalonate pathway}
}

\newglossaryentry{mevalonate_5_phosphotransferase}{
    name=mevalonate 5-phosphotransferase,
    description={An enzyme that catalyzes the phosphorylation of mevalonate to 5-phosphomevalonate, using ATP as the phosphate donor}
}

\newglossaryentry{phosphomevalonate_kinase}{
    name=phosphomevalonate kinase,
    description={An enzyme that phosphorylates 5-phosphomevalonate to form 5-pyrophosphomevalonate in the mevalonate pathway}
}

\newglossaryentry{pyrophosphomevalonate_decarboxylase}{
    name=pyrophosphomevalonate decarboxylase,
    description={A bifunctional enzyme that catalyzes both the phosphorylation of 5-pyrophosphomevalonate to 3-phospho-5-pyrophosphomevalonate and its decarboxylation to yield activated isoprene units}
}

\newglossaryentry{isoprene}{
    name=isoprene,
    description={A volatile hydrocarbon molecule that serves as a basic building block for the synthesis of terpenes and sterols, including cholesterol and other isoprenoid compounds}
}

\newglossaryentry{squalene}{
    name=squalene,
    description={A triterpenoid compound formed from the condensation of two molecules of farnesyl pyrophosphate (FPP), which is an intermediate in the mevalonate pathway, and serves as a precursor for sterols such as cholesterol}
}

\newglossaryentry{mevalonate}{
    name=mevalonate,
    description={A key intermediate in the biosynthesis of sterols, isoprenoids, and other terpenoids, formed from acetoacetyl-CoA through a series of enzymatic reactions in the mevalonate pathway}
}

\newglossaryentry{isopentenyl_pp}{
    name=isopentenyl pyrophosphate (IPP),
    description={An activated isoprene unit and key intermediate in the biosynthesis of terpenoids and sterols via the mevalonate pathway}
}

\newglossaryentry{dimethylallyl_pp}{
    name=dimethylallyl pyrophosphate (DMAPP),
    description={An isomer of IPP and another activated isoprene unit that participates in head-to-tail condensations during isoprenoid biosynthesis}
}

\newglossaryentry{geranyl_pp}{
    name=geranyl pyrophosphate (GPP),
    description={A 10-carbon isoprenoid intermediate formed by the head-to-tail condensation of IPP and DMAPP}
}

\newglossaryentry{farnesyl_pp}{
    name=farnesyl pyrophosphate (FPP),
    description={A 15-carbon isoprenoid intermediate formed by the head-to-tail condensation of GPP and IPP; precursor for squalene and other isoprenoids}
}

\newglossaryentry{prenyl_transferase}{
    name=prenyl transferase,
    description={An enzyme that catalyzes head-to-tail condensations of activated isoprene units (e.g., IPP and DMAPP) to form larger isoprenoid intermediates like GPP and FPP}
}

\newglossaryentry{squalene_synthase}{
    name=squalene synthase,
    description={An enzyme that catalyzes the head-to-head condensation of two farnesyl pyrophosphate molecules to produce squalene, a key precursor to sterols}
}


\newglossaryentry{squalene_monooxygenase}{
    name=squalene monooxygenase,
    description={An enzyme that catalyzes the first oxygenation step in sterol biosynthesis by converting squalene to squalene 2,3-epoxide}
}

\newglossaryentry{squalene_epoxide}{
    name=squalene 2-3-epoxide,
    description={An oxygenated intermediate formed from squalene by squalene monooxygenase; undergoes cyclization to form lanosterol}
}

\newglossaryentry{lanosterol}{
    name=lanosterol,
    description={The first sterol intermediate formed from squalene 2,3-epoxide during cholesterol biosynthesis in animals}
}

\newglossaryentry{cholesterol}{
    name=cholesterol,
    description={A vital sterol molecule in animal cells, serving as a structural component of membranes and a precursor for steroid hormones and bile acids}
}

\newglossaryentry{spt}{
    name={serine-palmitoyltransferase (SPT)},
    description={An enzyme that catalyzes the first and rate-limiting step in sphingolipid biosynthesis, condensing serine and palmitoyl-CoA to form 3-ketosphinganine}
}
\newglossaryentry{ormdl}{
    name={ORMDL1/2/3},
    description={A family of ER-resident membrane proteins that negatively regulate serine-palmitoyltransferase (SPT) activity, contributing to sphingolipid homeostasis}
}

\newglossaryentry{kdsr}{
    name={KDSR ($\beta$-ketosphinganine reductase)},
    description={An enzyme that catalyzes the NADPH-dependent reduction of $\beta$-ketosphingosine to sphinganine during sphingolipid biosynthesis. It is an ER membrane protein with its active site facing the cytosol}
}

\newglossaryentry{ceramidesynthase}{
    name={ceramide synthase},
    description={An enzyme that catalyzes the N-acylation of sphinganine or sphingosine with a fatty acyl-CoA to produce ceramide, a central intermediate in sphingolipid metabolism. It is localized to the endoplasmic reticulum}
}

\newglossaryentry{des}
{
    name={Dihydroceramide Desaturase (DES)},
    description={An enzyme that catalyzes the introduction of a double bond into dihydroceramides, converting them into ceramides. This reaction is crucial in the biosynthesis of sphingolipids, which are important components of cell membranes, playing key roles in cell signaling, structure, and function.}
}

\newglossaryentry{sms}
{
    name={Sphingomyelin Synthase (SMS)},
    description={An enzyme that catalyzes the conversion of ceramide and phosphatidylcholine into sphingomyelin and diacylglycerol. This reaction is a key step in sphingolipid metabolism.}
}

\newglossaryentry{cert1}
{
    name={CERT1 (Ceramide Transfer Protein 1)},
    description={A lipid transfer protein that transports ceramide from the endoplasmic reticulum (ER) to the Golgi apparatus, where it is used for the synthesis of sphingomyelin. CERT1 plays a crucial role in sphingolipid metabolism and membrane organization. }
}

\newglossaryentry{phdomain}
{
    name={Pleckstrin Homology (PH) Domain},
    description={A protein domain of approximately 100 amino acids found in many proteins involved in intracellular signaling. The PH domain binds to phosphoinositides in membranes, facilitating the localization of proteins to specific membrane compartments. It plays key roles in signal transduction and membrane trafficking.}
}

\newglossaryentry{ffat}
{
    name={FFAT Motif},
    description={A short linear amino acid sequence (two phenylalanines in an acidic tract) found in certain proteins that bind to the VAP (vesicle-associated membrane protein-associated protein) family on the endoplasmic reticulum. The FFAT motif mediates protein targeting to membrane contact sites, particularly between the ER and other organelles.}
}

\newglossaryentry{vap}
{
    name={VAP (Vesicle-associated membrane protein-associated protein)},
    description={A family of ER-resident membrane proteins that serve as scaffolds for the recruitment of cytosolic proteins containing FFAT motifs. VAPs play a central role in organizing membrane contact sites and facilitating lipid transfer and signaling between the ER and other organelles.}
}

\newglossaryentry{startdomain}
{
    name={C-terminal START Domain},
    description={A lipid-binding domain located at the C-terminus of certain proteins, including ceramide transport proteins (CERT). The START (StAR-related lipid transfer) domain facilitates the selective binding and transfer of lipids such as ceramide between membranes, contributing to non-vesicular lipid transport and membrane homeostasis.}
}

\newglossaryentry{mcs}
{
    name={Membrane Contact Sites (MCS)},
    description={Specialized regions where the membranes of two organelles are closely apposed, typically within 10–30 nm, without fusing. MCSs facilitate direct inter-organelle communication, allowing the exchange of lipids, ions, and signaling molecules. They play key roles in cellular homeostasis, lipid metabolism, and organelle dynamics.}
}

\newglossaryentry{cerebroside}
{
    name={Cerebroside},
    description={A type of glycosphingolipid consisting of a ceramide backbone (sphingosine + fatty acid) attached to a single sugar moiety, typically glucose or galactose. Cerebrosides are important components of cell membranes, particularly in the nervous system, where they contribute to myelin formation and cell-cell communication. In the brain, the most common cerebroside is galactocerebroside, found predominantly in myelin.}
}

\newglossaryentry{gcsc}
{
    name={GlcCer Synthase (GCS)},
    description={An enzyme responsible for the synthesis of glucosylceramide (GlcCer) from ceramide and UDP-glucose. GlcCer is an important glycosphingolipid involved in cell signaling, membrane structure, and interactions. GCS plays a pivotal role in the synthesis of complex sphingolipids and is essential for the proper function of various cellular processes, including cell growth and differentiation.}
}

\newglossaryentry{fapp2}
{
    name={FaPP2 (Fatty Acid Phosphatase 2)},
    description={An enzyme involved in the hydrolysis of phospholipids and sphingolipids, specifically catalyzing the dephosphorylation of fatty acid phosphates. FaPP2 plays a role in lipid metabolism by regulating the levels of fatty acid phosphates, which are important intermediates in various signaling pathways.}
}

\newglossaryentry{gltp}
{
    name={GLTP (Glycosphingolipid Transfer Protein)},
    description={A protein that facilitates the non-vesicular transfer of glycosphingolipids between cellular membranes. It is used to transfer cerebroside to the luminal leaflet of the Golgi}
}
\newglossaryentry{pitp}
{
    name={Phosphatidylinositol Transfer Proteins (PITPs)},
    description={A family of lipid transfer proteins that mediate the transfer of phosphatidylinositol (PI) and other phospholipids between membrane compartments. PITPs are crucial for phosphoinositide signaling, membrane trafficking, and lipid homeostasis. They play essential roles in processes such as vesicle formation, cell signaling, and membrane identity.}
}

\newglossaryentry{FYVEdomain}{
    name={FYVE domain},
    description={A protein domain that binds phosphatidylinositol 3-phosphate and is involved in membrane trafficking and signal transduction}
}

\newglossaryentry{gelsolinhomologydomain}{
    name={gelsolin homology domain},
    description={A domain found in gelsolin and related proteins, involved in actin filament severing, capping, and nucleation. It recognizes phosphoinositides}
}

\newglossaryentry{SH2domain}{
    name={SH2 domain},
    description={Src Homology 2 domain, a protein domain that binds specifically to phosphorylated tyrosine residues and is critical in signal transduction pathways. It recognizes phosphoinositides}
}

\newglossaryentry{PTBdomain}{
    name={PTB domain},
    description={Phosphotyrosine-binding domain, a protein domain that recognizes and binds to phosphotyrosine-containing motifs, typically in receptor tyrosine kinases. It recognizes phosphoinositides}
}

\newglossaryentry{PI4Ks}{
    name={PI4Ks},
    description={Phosphatidylinositol 4-kinases, a family of enzymes that phosphorylate phosphatidylinositol at the D-4 position of the inositol ring to produce phosphatidylinositol 4-phosphate (PI4P), a key lipid signaling molecule involved in membrane trafficking and signaling}
}

\newglossaryentry{Sac1}{
    name={Sac1},
    description={A conserved phosphoinositide phosphatase that dephosphorylates phosphatidylinositol 4-phosphate (PI4P), playing a key role in lipid homeostasis and membrane trafficking}
}

\newglossaryentry{OSBP}{
    name={OSBP},
    description={Oxysterol-binding protein, a lipid transfer protein that shuttles cholesterol and phosphatidylinositol 4-phosphate (PI4P) between organelle membranes, particularly the endoplasmic reticulum and Golgi apparatus}
}





























