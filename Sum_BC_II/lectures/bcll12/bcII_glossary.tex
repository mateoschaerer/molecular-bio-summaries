\newglossaryentry{cori-cycle}{
    name={Cori cycle},
    description={A metabolic pathway in which lactate produced by anaerobic glycolysis in the muscles is transported to the liver, converted to glucose via gluconeogenesis, and then sent back to the muscles to be used as an energy source}
}

\newglossaryentry{lipoprotein-lipase}{
    name={lipoprotein lipase (LPL)},
    description={An enzyme that hydrolyzes triglycerides in lipoproteins, such as chylomicrons and very low-density lipoproteins (VLDL), into free fatty acids and glycerol for uptake by tissues such as muscle and adipose tissue}
}

\newglossaryentry{hypoglycemia}{
    name={hypoglycemia},
    description={A condition characterized by abnormally low levels of blood glucose, which can lead to symptoms such as shakiness, confusion, sweating, and in severe cases, loss of consciousness}
}

\newglossaryentry{hypothalamus}{
    name={hypothalamus},
    description={A brain region that produces hormones targeting the anterior pituitary and sends neuronal signals to the posterior pituitary in response to nervous stimuli}
}

\newglossaryentry{anterior-pituitary}{
    name={anterior pituitary},
    description={An endocrine gland that produces hormones in response to nervous stimuli, targeting downstream organs such as the adrenal cortex, thyroid, testes, and ovaries}
}

\newglossaryentry{posterior-pituitary}{
    name={posterior pituitary},
    description={A part of the pituitary gland containing axons of hypothalamic neurons that secrete oxytocin and vasopressin}
}

\newglossaryentry{oxytocin}{
    name={oxytocin},
    description={A hormone secreted by the posterior pituitary that induces uterine contractions during labor and milk ejection during lactation}
}

\newglossaryentry{vasopressin}{
    name={vasopressin},
    description={A hormone released by the posterior pituitary that helps regulate blood pressure}
}

\newglossaryentry{snarecomplex}{
    name={SNARE complex},
    description={A protein complex that mediates vesicle fusion by bringing membranes into close proximity, crucial for intracellular trafficking and neurotransmitter release. SNARE stands for \textbf{S}oluble \textbf{N}-ethylmaleimide-sensitive factor \textbf{A}ttachment protein \textbf{RE}ceptor}
}

\newglossaryentry{katpchannel}{
    name={K\textsubscript{ATP} channel},
    description={An ATP-sensitive potassium channel that links cellular metabolism to electrical activity by opening or closing in response to intracellular ATP levels. These channels are important in tissues such as pancreatic beta cells, cardiac muscle, and neurons}
}

\newglossaryentry{sur}{
    name={SUR},
    description={Sulfonylurea Receptor, a regulatory subunit of the K\textsubscript{ATP} channel that binds sulfonylurea drugs and modulates channel activity. There are different isoforms such as SUR1 and SUR2, which influence channel properties in different tissues}
}

\newglossaryentry{synaptobrevin2}{
    name={Synaptobrevin2},
    description={Also known as VAMP2 (Vesicle-Associated Membrane Protein 2), a vesicular SNARE protein critical for membrane fusion during exocytosis}
}

\newglossaryentry{syntaxin1a}{
    name={Syntaxin1A},
    description={A membrane-bound SNARE protein located on the plasma membrane that forms a complex with SNAP-25 and Synaptobrevin2/VAMP2 to mediate vesicle fusion}
}

\newglossaryentry{synaptotagmin}{
    name={Synaptotagmin},
    description={A calcium-sensing protein that binds calcium ions and regulates SNARE-mediated membrane fusion, especially in neurons and endocrine cells}
}

\newglossaryentry{pc1pc2}{
    name={PC1 and PC2},
    description={Prohormone convertases 1 and 2; endoproteases that cleave prohormone precursors at specific dibasic residues, enabling the production of mature, biologically active peptide hormones in neuroendocrine cells}
}

\newglossaryentry{cpe}{
    name={Carboxypeptidase E},
    description={An exoprotease that removes C-terminal basic amino acids from peptide precursors after initial cleavage by prohormone convertases such as PC1 and PC2, completing the maturation of many peptide hormones}
}




